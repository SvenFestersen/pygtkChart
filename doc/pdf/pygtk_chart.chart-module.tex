%
% API Documentation for pygtkChart
% Module pygtk_chart.chart
%
% Generated by epydoc 3.0.1
% [Mon Sep 28 11:06:52 2009]
%

%%%%%%%%%%%%%%%%%%%%%%%%%%%%%%%%%%%%%%%%%%%%%%%%%%%%%%%%%%%%%%%%%%%%%%%%%%%
%%                          Module Description                           %%
%%%%%%%%%%%%%%%%%%%%%%%%%%%%%%%%%%%%%%%%%%%%%%%%%%%%%%%%%%%%%%%%%%%%%%%%%%%

    \index{pygtk\_chart \textit{(package)}!pygtk\_chart.chart \textit{(module)}|(}
\section{Module pygtk\_chart.chart}

    \label{pygtk_chart:chart}
(section) Module Contents

  This is the main module. It contains the base classes for chart widgets.

  \begin{itemize}
  \setlength{\parskip}{0.6ex}
    \item class Chart: base class for all chart widgets.

    \item class Background: background of a chart widget.

    \item class Title: title of a chart.

  \end{itemize}

  (section) Colors

    All colors that pygtkChart uses are gtk.gdk.Colors as used by PyGTK.

    Author: Sven Festersen (sven@sven-festersen.de)


%%%%%%%%%%%%%%%%%%%%%%%%%%%%%%%%%%%%%%%%%%%%%%%%%%%%%%%%%%%%%%%%%%%%%%%%%%%
%%                               Functions                               %%
%%%%%%%%%%%%%%%%%%%%%%%%%%%%%%%%%%%%%%%%%%%%%%%%%%%%%%%%%%%%%%%%%%%%%%%%%%%

  \subsection{Functions}

    \label{pygtk_chart:chart:init_sensitive_areas}
    \index{pygtk\_chart \textit{(package)}!pygtk\_chart.chart \textit{(module)}!pygtk\_chart.chart.init\_sensitive\_areas \textit{(function)}}

    \vspace{0.5ex}

\hspace{.8\funcindent}\begin{boxedminipage}{\funcwidth}

    \raggedright \textbf{init\_sensitive\_areas}()

\setlength{\parskip}{2ex}
\setlength{\parskip}{1ex}
    \end{boxedminipage}

    \label{pygtk_chart:chart:add_sensitive_area}
    \index{pygtk\_chart \textit{(package)}!pygtk\_chart.chart \textit{(module)}!pygtk\_chart.chart.add\_sensitive\_area \textit{(function)}}

    \vspace{0.5ex}

\hspace{.8\funcindent}\begin{boxedminipage}{\funcwidth}

    \raggedright \textbf{add\_sensitive\_area}(\textit{type}, \textit{coords}, \textit{data})

\setlength{\parskip}{2ex}
\setlength{\parskip}{1ex}
    \end{boxedminipage}

    \label{pygtk_chart:chart:get_sensitive_areas}
    \index{pygtk\_chart \textit{(package)}!pygtk\_chart.chart \textit{(module)}!pygtk\_chart.chart.get\_sensitive\_areas \textit{(function)}}

    \vspace{0.5ex}

\hspace{.8\funcindent}\begin{boxedminipage}{\funcwidth}

    \raggedright \textbf{get\_sensitive\_areas}(\textit{x}, \textit{y})

\setlength{\parskip}{2ex}
\setlength{\parskip}{1ex}
    \end{boxedminipage}


%%%%%%%%%%%%%%%%%%%%%%%%%%%%%%%%%%%%%%%%%%%%%%%%%%%%%%%%%%%%%%%%%%%%%%%%%%%
%%                               Variables                               %%
%%%%%%%%%%%%%%%%%%%%%%%%%%%%%%%%%%%%%%%%%%%%%%%%%%%%%%%%%%%%%%%%%%%%%%%%%%%

  \subsection{Variables}

    \vspace{-1cm}
\hspace{\varindent}\begin{longtable}{|p{\varnamewidth}|p{\vardescrwidth}|l}
\cline{1-2}
\cline{1-2} \centering \textbf{Name} & \centering \textbf{Description}& \\
\cline{1-2}
\endhead\cline{1-2}\multicolumn{3}{r}{\small\textit{continued on next page}}\\\endfoot\cline{1-2}
\endlastfoot\raggedright C\-O\-L\-O\-R\-\_\-A\-U\-T\-O\- & \raggedright \textbf{Value:} 
{\tt 0}&\\
\cline{1-2}
\raggedright A\-R\-E\-A\-\_\-C\-I\-R\-C\-L\-E\- & \raggedright \textbf{Value:} 
{\tt 0}&\\
\cline{1-2}
\raggedright A\-R\-E\-A\-\_\-R\-E\-C\-T\-A\-N\-G\-L\-E\- & \raggedright \textbf{Value:} 
{\tt 1}&\\
\cline{1-2}
\raggedright C\-L\-I\-C\-K\-\_\-S\-E\-N\-S\-I\-T\-I\-V\-E\-\_\-A\-R\-E\-A\-S\- & \raggedright \textbf{Value:} 
{\tt \texttt{[}\texttt{]}}&\\
\cline{1-2}
\end{longtable}


%%%%%%%%%%%%%%%%%%%%%%%%%%%%%%%%%%%%%%%%%%%%%%%%%%%%%%%%%%%%%%%%%%%%%%%%%%%
%%                           Class Description                           %%
%%%%%%%%%%%%%%%%%%%%%%%%%%%%%%%%%%%%%%%%%%%%%%%%%%%%%%%%%%%%%%%%%%%%%%%%%%%

    \index{pygtk\_chart \textit{(package)}!pygtk\_chart.chart \textit{(module)}!pygtk\_chart.chart.Chart \textit{(class)}|(}
\subsection{Class Chart}

    \label{pygtk_chart:chart:Chart}
\begin{tabular}{cccccccccccccc}
% Line for object, linespec=[False, False, False, False, False]
\multicolumn{2}{r}{\settowidth{\BCL}{object}\multirow{2}{\BCL}{object}}
&&
&&
&&
&&
&&
  \\\cline{3-3}
  &&\multicolumn{1}{c|}{}
&&
&&
&&
&&
&&
  \\
% Line for ??.GObject, linespec=[False, False, False, False]
\multicolumn{4}{r}{\settowidth{\BCL}{??.GObject}\multirow{2}{\BCL}{??.GObject}}
&&
&&
&&
&&
  \\\cline{5-5}
  &&&&\multicolumn{1}{c|}{}
&&
&&
&&
&&
  \\
% Line for gtk.Object, linespec=[False, False, False]
\multicolumn{6}{r}{\settowidth{\BCL}{gtk.Object}\multirow{2}{\BCL}{gtk.Object}}
&&
&&
&&
  \\\cline{7-7}
  &&&&&&\multicolumn{1}{c|}{}
&&
&&
&&
  \\
% Line for object, linespec=[False, False, True, False, False]
\multicolumn{2}{r}{\settowidth{\BCL}{object}\multirow{2}{\BCL}{object}}
&&
&&
&&\multicolumn{1}{|c}{}
&&
&&
  \\\cline{3-3}
  &&\multicolumn{1}{c|}{}
&&
&&
&\multicolumn{1}{|c}{}&
&&
&&
  \\
% Line for gobject.GInterface, linespec=[False, True, False, False]
\multicolumn{4}{r}{\settowidth{\BCL}{gobject.GInterface}\multirow{2}{\BCL}{gobject.GInterface}}
&&
&&\multicolumn{1}{|c}{}
&&
&&
  \\\cline{5-5}
  &&&&\multicolumn{1}{c|}{}
&&
&\multicolumn{1}{|c}{}&
&&
&&
  \\
% Line for atk.ImplementorIface, linespec=[True, False, False]
\multicolumn{6}{r}{\settowidth{\BCL}{atk.ImplementorIface}\multirow{2}{\BCL}{atk.ImplementorIface}}
&&\multicolumn{1}{|c}{}
&&
&&
  \\\cline{7-7}
  &&&&&&\multicolumn{1}{c|}{}
&\multicolumn{1}{|c}{}&
&&
&&
  \\
% Line for object, linespec=[False, False, True, False, False]
\multicolumn{2}{r}{\settowidth{\BCL}{object}\multirow{2}{\BCL}{object}}
&&
&&
&&\multicolumn{1}{|c}{}
&&
&&
  \\\cline{3-3}
  &&\multicolumn{1}{c|}{}
&&
&&
&\multicolumn{1}{|c}{}&
&&
&&
  \\
% Line for gobject.GInterface, linespec=[False, True, False, False]
\multicolumn{4}{r}{\settowidth{\BCL}{gobject.GInterface}\multirow{2}{\BCL}{gobject.GInterface}}
&&
&&\multicolumn{1}{|c}{}
&&
&&
  \\\cline{5-5}
  &&&&\multicolumn{1}{c|}{}
&&
&\multicolumn{1}{|c}{}&
&&
&&
  \\
% Line for gtk.Buildable, linespec=[True, False, False]
\multicolumn{6}{r}{\settowidth{\BCL}{gtk.Buildable}\multirow{2}{\BCL}{gtk.Buildable}}
&&\multicolumn{1}{|c}{}
&&
&&
  \\\cline{7-7}
  &&&&&&\multicolumn{1}{c|}{}
&\multicolumn{1}{|c}{}&
&&
&&
  \\
% Line for gtk.Widget, linespec=[False, False]
\multicolumn{8}{r}{\settowidth{\BCL}{gtk.Widget}\multirow{2}{\BCL}{gtk.Widget}}
&&
&&
  \\\cline{9-9}
  &&&&&&&&\multicolumn{1}{c|}{}
&&
&&
  \\
% Line for gtk.DrawingArea, linespec=[False]
\multicolumn{10}{r}{\settowidth{\BCL}{gtk.DrawingArea}\multirow{2}{\BCL}{gtk.DrawingArea}}
&&
  \\\cline{11-11}
  &&&&&&&&&&\multicolumn{1}{c|}{}
&&
  \\
&&&&&&&&&&\multicolumn{2}{l}{\textbf{pygtk\_chart.chart.Chart}}
\end{tabular}

\textbf{Known Subclasses:}
pygtk\_chart.bar\_chart.BarChart,
    pygtk\_chart.line\_chart.LineChart,
    pygtk\_chart.pie\_chart.PieChart

This is the base class for all chart widgets.

(section) Properties

  The Chart class inherits properties from gtk.DrawingArea. Additional 
  properties:

  \begin{itemize}
  \setlength{\parskip}{0.6ex}
    \item padding (the amount of free white space between the chart's content 
      and its border in px, type: int in [0, 100].

  \end{itemize}

(section) Signals

  The Chart class inherits signals from gtk.DrawingArea.


%%%%%%%%%%%%%%%%%%%%%%%%%%%%%%%%%%%%%%%%%%%%%%%%%%%%%%%%%%%%%%%%%%%%%%%%%%%
%%                                Methods                                %%
%%%%%%%%%%%%%%%%%%%%%%%%%%%%%%%%%%%%%%%%%%%%%%%%%%%%%%%%%%%%%%%%%%%%%%%%%%%

  \subsubsection{Methods}

    \vspace{0.5ex}

\hspace{.8\funcindent}\begin{boxedminipage}{\funcwidth}

    \raggedright \textbf{\_\_init\_\_}(\textit{self})

\setlength{\parskip}{2ex}
    x.\_\_init\_\_(...) initializes x; see x.\_\_class\_\_.\_\_doc\_\_ for 
    signature

\setlength{\parskip}{1ex}
      Overrides: object.\_\_init\_\_ 	extit{(inherited documentation)}

    \end{boxedminipage}

    \label{pygtk_chart:chart:Chart:do_get_property}
    \index{pygtk\_chart \textit{(package)}!pygtk\_chart.chart \textit{(module)}!pygtk\_chart.chart.Chart \textit{(class)}!pygtk\_chart.chart.Chart.do\_get\_property \textit{(method)}}

    \vspace{0.5ex}

\hspace{.8\funcindent}\begin{boxedminipage}{\funcwidth}

    \raggedright \textbf{do\_get\_property}(\textit{self}, \textit{property})

\setlength{\parskip}{2ex}
\setlength{\parskip}{1ex}
    \end{boxedminipage}

    \label{pygtk_chart:chart:Chart:do_set_property}
    \index{pygtk\_chart \textit{(package)}!pygtk\_chart.chart \textit{(module)}!pygtk\_chart.chart.Chart \textit{(class)}!pygtk\_chart.chart.Chart.do\_set\_property \textit{(method)}}

    \vspace{0.5ex}

\hspace{.8\funcindent}\begin{boxedminipage}{\funcwidth}

    \raggedright \textbf{do\_set\_property}(\textit{self}, \textit{property}, \textit{value})

\setlength{\parskip}{2ex}
\setlength{\parskip}{1ex}
    \end{boxedminipage}

    \label{pygtk_chart:chart:Chart:draw_basics}
    \index{pygtk\_chart \textit{(package)}!pygtk\_chart.chart \textit{(module)}!pygtk\_chart.chart.Chart \textit{(class)}!pygtk\_chart.chart.Chart.draw\_basics \textit{(method)}}

    \vspace{0.5ex}

\hspace{.8\funcindent}\begin{boxedminipage}{\funcwidth}

    \raggedright \textbf{draw\_basics}(\textit{self}, \textit{context}, \textit{rect})

    \vspace{-1.5ex}

    \rule{\textwidth}{0.5\fboxrule}
\setlength{\parskip}{2ex}
    Draw basic things that every plot has (background, title, ...).

\setlength{\parskip}{1ex}
      \textbf{Parameters}
      \vspace{-1ex}

      \begin{quote}
        \begin{Ventry}{xxxxxxx}

          \item[context]

          The context to draw on.

            {\it (type=cairo.Context)}

          \item[rect]

          A rectangle representing the charts area.

            {\it (type=gtk.gdk.Rectangle)}

        \end{Ventry}

      \end{quote}

    \end{boxedminipage}

    \vspace{0.5ex}

\hspace{.8\funcindent}\begin{boxedminipage}{\funcwidth}

    \raggedright \textbf{draw}(\textit{self}, \textit{context})

    \vspace{-1.5ex}

    \rule{\textwidth}{0.5\fboxrule}
\setlength{\parskip}{2ex}
    Draw the widget. This method is called automatically. Don't call it 
    yourself. If you want to force a redrawing of the widget, call the 
    queue\_draw() method.

\setlength{\parskip}{1ex}
      \textbf{Parameters}
      \vspace{-1ex}

      \begin{quote}
        \begin{Ventry}{xxxxxxx}

          \item[context]

          The context to draw on.

            {\it (type=cairo.Context)}

        \end{Ventry}

      \end{quote}

      Overrides: gtk.Widget.draw

    \end{boxedminipage}

    \label{pygtk_chart:chart:Chart:export_svg}
    \index{pygtk\_chart \textit{(package)}!pygtk\_chart.chart \textit{(module)}!pygtk\_chart.chart.Chart \textit{(class)}!pygtk\_chart.chart.Chart.export\_svg \textit{(method)}}

    \vspace{0.5ex}

\hspace{.8\funcindent}\begin{boxedminipage}{\funcwidth}

    \raggedright \textbf{export\_svg}(\textit{self}, \textit{filename}, \textit{size}={\tt None})

    \vspace{-1.5ex}

    \rule{\textwidth}{0.5\fboxrule}
\setlength{\parskip}{2ex}
    Saves the contents of the widget to svg file. The size of the image 
    will be the size of the widget.

\setlength{\parskip}{1ex}
      \textbf{Parameters}
      \vspace{-1ex}

      \begin{quote}
        \begin{Ventry}{xxxxxxxx}

          \item[filename]

          The path to the file where you want the chart to be saved.

            {\it (type=string)}

          \item[size]

          Optional parameter to give the desired height and width of the 
          image.

            {\it (type=tuple)}

        \end{Ventry}

      \end{quote}

    \end{boxedminipage}

    \label{pygtk_chart:chart:Chart:export_png}
    \index{pygtk\_chart \textit{(package)}!pygtk\_chart.chart \textit{(module)}!pygtk\_chart.chart.Chart \textit{(class)}!pygtk\_chart.chart.Chart.export\_png \textit{(method)}}

    \vspace{0.5ex}

\hspace{.8\funcindent}\begin{boxedminipage}{\funcwidth}

    \raggedright \textbf{export\_png}(\textit{self}, \textit{filename}, \textit{size}={\tt None})

    \vspace{-1.5ex}

    \rule{\textwidth}{0.5\fboxrule}
\setlength{\parskip}{2ex}
    Saves the contents of the widget to png file. The size of the image 
    will be the size of the widget.

\setlength{\parskip}{1ex}
      \textbf{Parameters}
      \vspace{-1ex}

      \begin{quote}
        \begin{Ventry}{xxxxxxxx}

          \item[filename]

          The path to the file where you want the chart to be saved.

            {\it (type=string)}

          \item[size]

          Optional parameter to give the desired height and width of the 
          image.

            {\it (type=tuple)}

        \end{Ventry}

      \end{quote}

    \end{boxedminipage}

    \label{pygtk_chart:chart:Chart:set_padding}
    \index{pygtk\_chart \textit{(package)}!pygtk\_chart.chart \textit{(module)}!pygtk\_chart.chart.Chart \textit{(class)}!pygtk\_chart.chart.Chart.set\_padding \textit{(method)}}

    \vspace{0.5ex}

\hspace{.8\funcindent}\begin{boxedminipage}{\funcwidth}

    \raggedright \textbf{set\_padding}(\textit{self}, \textit{padding})

    \vspace{-1.5ex}

    \rule{\textwidth}{0.5\fboxrule}
\setlength{\parskip}{2ex}
    Set the chart's padding.

\setlength{\parskip}{1ex}
      \textbf{Parameters}
      \vspace{-1ex}

      \begin{quote}
        \begin{Ventry}{xxxxxxx}

          \item[padding]

          the padding in px

            {\it (type=int in [0, 100] (default: 16).)}

        \end{Ventry}

      \end{quote}

    \end{boxedminipage}

    \label{pygtk_chart:chart:Chart:get_padding}
    \index{pygtk\_chart \textit{(package)}!pygtk\_chart.chart \textit{(module)}!pygtk\_chart.chart.Chart \textit{(class)}!pygtk\_chart.chart.Chart.get\_padding \textit{(method)}}

    \vspace{0.5ex}

\hspace{.8\funcindent}\begin{boxedminipage}{\funcwidth}

    \raggedright \textbf{get\_padding}(\textit{self})

    \vspace{-1.5ex}

    \rule{\textwidth}{0.5\fboxrule}
\setlength{\parskip}{2ex}
    Returns the chart's padding.

\setlength{\parskip}{1ex}
      \textbf{Return Value}
    \vspace{-1ex}

      \begin{quote}
      int in [0, 100].

      \end{quote}

    \end{boxedminipage}


\large{\textbf{\textit{Inherited from gtk.DrawingArea}}}

\begin{quote}
size()
\end{quote}

\large{\textbf{\textit{Inherited from gtk.Widget}}}

\begin{quote}
activate(), add\_accelerator(), add\_events(), add\_mnemonic\_label(), can\_activate\_accel(), child\_focus(), child\_notify(), class\_path(), create\_pango\_context(), create\_pango\_layout(), destroy(), do\_button\_press\_event(), do\_button\_release\_event(), do\_can\_activate\_accel(), do\_client\_event(), do\_composited\_changed(), do\_configure\_event(), do\_delete\_event(), do\_destroy\_event(), do\_direction\_changed(), do\_drag\_begin(), do\_drag\_data\_delete(), do\_drag\_data\_get(), do\_drag\_data\_received(), do\_drag\_drop(), do\_drag\_end(), do\_drag\_leave(), do\_drag\_motion(), do\_enter\_notify\_event(), do\_event(), do\_expose\_event(), do\_focus(), do\_focus\_in\_event(), do\_focus\_out\_event(), do\_get\_accessible(), do\_grab\_broken\_event(), do\_grab\_focus(), do\_grab\_notify(), do\_hide(), do\_hide\_all(), do\_hierarchy\_changed(), do\_key\_press\_event(), do\_key\_release\_event(), do\_leave\_notify\_event(), do\_map(), do\_map\_event(), do\_mnemonic\_activate(), do\_motion\_notify\_event(), do\_no\_expose\_event(), do\_parent\_set(), do\_popup\_menu(), do\_property\_notify\_event(), do\_proximity\_in\_event(), do\_proximity\_out\_event(), do\_realize(), do\_screen\_changed(), do\_scroll\_event(), do\_selection\_clear\_event(), do\_selection\_get(), do\_selection\_notify\_event(), do\_selection\_received(), do\_selection\_request\_event(), do\_show(), do\_show\_all(), do\_show\_help(), do\_size\_allocate(), do\_size\_request(), do\_state\_changed(), do\_style\_set(), do\_unmap(), do\_unmap\_event(), do\_unrealize(), do\_visibility\_notify\_event(), do\_window\_state\_event(), drag\_begin(), drag\_check\_threshold(), drag\_dest\_add\_image\_targets(), drag\_dest\_add\_text\_targets(), drag\_dest\_add\_uri\_targets(), drag\_dest\_find\_target(), drag\_dest\_get\_target\_list(), drag\_dest\_get\_track\_motion(), drag\_dest\_set(), drag\_dest\_set\_proxy(), drag\_dest\_set\_target\_list(), drag\_dest\_set\_track\_motion(), drag\_dest\_unset(), drag\_get\_data(), drag\_highlight(), drag\_source\_add\_image\_targets(), drag\_source\_add\_text\_targets(), drag\_source\_add\_uri\_targets(), drag\_source\_get\_target\_list(), drag\_source\_set(), drag\_source\_set\_icon(), drag\_source\_set\_icon\_name(), drag\_source\_set\_icon\_pixbuf(), drag\_source\_set\_icon\_stock(), drag\_source\_set\_target\_list(), drag\_source\_unset(), drag\_unhighlight(), ensure\_style(), error\_bell(), event(), freeze\_child\_notify(), get\_accessible(), get\_action(), get\_activate\_signal(), get\_allocation(), get\_ancestor(), get\_child\_requisition(), get\_child\_visible(), get\_clipboard(), get\_colormap(), get\_composite\_name(), get\_direction(), get\_display(), get\_events(), get\_extension\_events(), get\_has\_tooltip(), get\_modifier\_style(), get\_name(), get\_no\_show\_all(), get\_pango\_context(), get\_parent(), get\_parent\_window(), get\_pointer(), get\_root\_window(), get\_screen(), get\_settings(), get\_size\_request(), get\_snapshot(), get\_style(), get\_tooltip\_markup(), get\_tooltip\_text(), get\_tooltip\_window(), get\_toplevel(), get\_visual(), get\_window(), grab\_add(), grab\_default(), grab\_focus(), grab\_remove(), has\_screen(), hide(), hide\_all(), hide\_on\_delete(), input\_shape\_combine\_mask(), intersect(), is\_ancestor(), is\_composited(), is\_focus(), keynav\_failed(), list\_mnemonic\_labels(), map(), menu\_get\_for\_attach\_widget(), mnemonic\_activate(), modify\_base(), modify\_bg(), modify\_cursor(), modify\_fg(), modify\_font(), modify\_style(), modify\_text(), path(), queue\_clear(), queue\_clear\_area(), queue\_draw(), queue\_draw\_area(), queue\_resize(), queue\_resize\_no\_redraw(), rc\_get\_style(), realize(), region\_intersect(), remove\_accelerator(), remove\_mnemonic\_label(), render\_icon(), reparent(), reset\_rc\_styles(), reset\_shapes(), selection\_add\_target(), selection\_add\_targets(), selection\_clear\_targets(), selection\_convert(), selection\_owner\_set(), selection\_remove\_all(), send\_expose(), set\_accel\_path(), set\_activate\_signal(), set\_app\_paintable(), set\_child\_visible(), set\_colormap(), set\_composite\_name(), set\_direction(), set\_double\_buffered(), set\_events(), set\_extension\_events(), set\_has\_tooltip(), set\_name(), set\_no\_show\_all(), set\_parent(), set\_parent\_window(), set\_redraw\_on\_allocate(), set\_scroll\_adjustments(), set\_sensitive(), set\_set\_scroll\_adjustments\_signal(), set\_size\_request(), set\_state(), set\_style(), set\_tooltip\_markup(), set\_tooltip\_text(), set\_tooltip\_window(), set\_uposition(), set\_usize(), shape\_combine\_mask(), show(), show\_all(), show\_now(), size\_allocate(), size\_request(), style\_get\_property(), thaw\_child\_notify(), translate\_coordinates(), trigger\_tooltip\_query(), unmap(), unparent(), unrealize()
\end{quote}

\large{\textbf{\textit{Inherited from gtk.Object}}}

\begin{quote}
do\_destroy(), flags(), remove\_data(), remove\_no\_notify(), set\_flags(), unset\_flags()
\end{quote}

\large{\textbf{\textit{Inherited from ??.GObject}}}

\begin{quote}
\_\_cmp\_\_(), \_\_copy\_\_(), \_\_deepcopy\_\_(), \_\_delattr\_\_(), \_\_gdoc\_\_(), \_\_gobject\_init\_\_(), \_\_hash\_\_(), \_\_new\_\_(), \_\_repr\_\_(), \_\_setattr\_\_(), chain(), connect(), connect\_after(), connect\_object(), connect\_object\_after(), disconnect(), disconnect\_by\_func(), emit(), emit\_stop\_by\_name(), freeze\_notify(), get\_data(), get\_properties(), get\_property(), handler\_block(), handler\_block\_by\_func(), handler\_disconnect(), handler\_is\_connected(), handler\_unblock(), handler\_unblock\_by\_func(), notify(), props(), set\_data(), set\_properties(), set\_property(), stop\_emission(), thaw\_notify(), weak\_ref()
\end{quote}

\large{\textbf{\textit{Inherited from atk.ImplementorIface}}}

\begin{quote}
ref\_accessible()
\end{quote}

\large{\textbf{\textit{Inherited from gtk.Buildable}}}

\begin{quote}
add\_child(), construct\_child(), do\_add\_child(), do\_construct\_child(), do\_get\_internal\_child(), do\_parser\_finished(), do\_set\_name(), get\_internal\_child(), parser\_finished()
\end{quote}

\large{\textbf{\textit{Inherited from object}}}

\begin{quote}
\_\_getattribute\_\_(), \_\_reduce\_\_(), \_\_reduce\_ex\_\_(), \_\_str\_\_()
\end{quote}

%%%%%%%%%%%%%%%%%%%%%%%%%%%%%%%%%%%%%%%%%%%%%%%%%%%%%%%%%%%%%%%%%%%%%%%%%%%
%%                              Properties                               %%
%%%%%%%%%%%%%%%%%%%%%%%%%%%%%%%%%%%%%%%%%%%%%%%%%%%%%%%%%%%%%%%%%%%%%%%%%%%

  \subsubsection{Properties}

    \vspace{-1cm}
\hspace{\varindent}\begin{longtable}{|p{\varnamewidth}|p{\vardescrwidth}|l}
\cline{1-2}
\cline{1-2} \centering \textbf{Name} & \centering \textbf{Description}& \\
\cline{1-2}
\endhead\cline{1-2}\multicolumn{3}{r}{\small\textit{continued on next page}}\\\endfoot\cline{1-2}
\endlastfoot\multicolumn{2}{|l|}{\textit{Inherited from gtk.Widget}}\\
\multicolumn{2}{|p{\varwidth}|}{\raggedright allocation, name, parent, requisition, saved\_state, state, style, window}\\
\cline{1-2}
\multicolumn{2}{|l|}{\textit{Inherited from ??.GObject}}\\
\multicolumn{2}{|p{\varwidth}|}{\raggedright \_\_grefcount\_\_}\\
\cline{1-2}
\multicolumn{2}{|l|}{\textit{Inherited from object}}\\
\multicolumn{2}{|p{\varwidth}|}{\raggedright \_\_class\_\_}\\
\cline{1-2}
\end{longtable}


%%%%%%%%%%%%%%%%%%%%%%%%%%%%%%%%%%%%%%%%%%%%%%%%%%%%%%%%%%%%%%%%%%%%%%%%%%%
%%                            Class Variables                            %%
%%%%%%%%%%%%%%%%%%%%%%%%%%%%%%%%%%%%%%%%%%%%%%%%%%%%%%%%%%%%%%%%%%%%%%%%%%%

  \subsubsection{Class Variables}

    \vspace{-1cm}
\hspace{\varindent}\begin{longtable}{|p{\varnamewidth}|p{\vardescrwidth}|l}
\cline{1-2}
\cline{1-2} \centering \textbf{Name} & \centering \textbf{Description}& \\
\cline{1-2}
\endhead\cline{1-2}\multicolumn{3}{r}{\small\textit{continued on next page}}\\\endfoot\cline{1-2}
\endlastfoot\raggedright \_\-\_\-g\-p\-r\-o\-p\-e\-r\-t\-i\-e\-s\-\_\-\_\- & \raggedright \textbf{Value:} 
{\tt \{"padding":(gobject.TYPE\_INT, "padding", "The chart's pad\texttt{...}}&\\
\cline{1-2}
\raggedright \_\-\_\-g\-t\-y\-p\-e\-\_\-\_\- & \raggedright \textbf{Value:} 
{\tt {\textless}GType pygtk\_chart+chart+Chart (168641056){\textgreater}}&\\
\cline{1-2}
\end{longtable}

    \index{pygtk\_chart \textit{(package)}!pygtk\_chart.chart \textit{(module)}!pygtk\_chart.chart.Chart \textit{(class)}|)}

%%%%%%%%%%%%%%%%%%%%%%%%%%%%%%%%%%%%%%%%%%%%%%%%%%%%%%%%%%%%%%%%%%%%%%%%%%%
%%                           Class Description                           %%
%%%%%%%%%%%%%%%%%%%%%%%%%%%%%%%%%%%%%%%%%%%%%%%%%%%%%%%%%%%%%%%%%%%%%%%%%%%

    \index{pygtk\_chart \textit{(package)}!pygtk\_chart.chart \textit{(module)}!pygtk\_chart.chart.Background \textit{(class)}|(}
\subsection{Class Background}

    \label{pygtk_chart:chart:Background}
\begin{tabular}{cccccccccc}
% Line for object, linespec=[False, False, False]
\multicolumn{2}{r}{\settowidth{\BCL}{object}\multirow{2}{\BCL}{object}}
&&
&&
&&
  \\\cline{3-3}
  &&\multicolumn{1}{c|}{}
&&
&&
&&
  \\
% Line for ??.GObject, linespec=[False, False]
\multicolumn{4}{r}{\settowidth{\BCL}{??.GObject}\multirow{2}{\BCL}{??.GObject}}
&&
&&
  \\\cline{5-5}
  &&&&\multicolumn{1}{c|}{}
&&
&&
  \\
% Line for pygtk\_chart.chart\_object.ChartObject, linespec=[False]
\multicolumn{6}{r}{\settowidth{\BCL}{pygtk\_chart.chart\_object.ChartObject}\multirow{2}{\BCL}{pygtk\_chart.chart\_object.ChartObject}}
&&
  \\\cline{7-7}
  &&&&&&\multicolumn{1}{c|}{}
&&
  \\
&&&&&&\multicolumn{2}{l}{\textbf{pygtk\_chart.chart.Background}}
\end{tabular}

The background of a chart.

(section) Properties

  This class inherits properties from chart\_object.ChartObject. Additional
  properties:

  \begin{itemize}
  \setlength{\parskip}{0.6ex}
    \item color (the background color, type: gtk.gdk.Color)

    \item gradient (the background gradient, type: a pair of gtk.gdk.Color)

    \item image (path to the background image file, type: string)

  \end{itemize}

(section) Signals

  The Background class inherits signals from chart\_object.ChartObject.


%%%%%%%%%%%%%%%%%%%%%%%%%%%%%%%%%%%%%%%%%%%%%%%%%%%%%%%%%%%%%%%%%%%%%%%%%%%
%%                                Methods                                %%
%%%%%%%%%%%%%%%%%%%%%%%%%%%%%%%%%%%%%%%%%%%%%%%%%%%%%%%%%%%%%%%%%%%%%%%%%%%

  \subsubsection{Methods}

    \vspace{0.5ex}

\hspace{.8\funcindent}\begin{boxedminipage}{\funcwidth}

    \raggedright \textbf{\_\_init\_\_}(\textit{self})

\setlength{\parskip}{2ex}
    x.\_\_init\_\_(...) initializes x; see x.\_\_class\_\_.\_\_doc\_\_ for 
    signature

\setlength{\parskip}{1ex}
      Overrides: object.\_\_init\_\_ 	extit{(inherited documentation)}

    \end{boxedminipage}

    \vspace{0.5ex}

\hspace{.8\funcindent}\begin{boxedminipage}{\funcwidth}

    \raggedright \textbf{do\_get\_property}(\textit{self}, \textit{property})

\setlength{\parskip}{2ex}
\setlength{\parskip}{1ex}
      Overrides: pygtk\_chart.chart\_object.ChartObject.do\_get\_property

    \end{boxedminipage}

    \vspace{0.5ex}

\hspace{.8\funcindent}\begin{boxedminipage}{\funcwidth}

    \raggedright \textbf{do\_set\_property}(\textit{self}, \textit{property}, \textit{value})

\setlength{\parskip}{2ex}
\setlength{\parskip}{1ex}
      Overrides: pygtk\_chart.chart\_object.ChartObject.do\_set\_property

    \end{boxedminipage}

    \label{pygtk_chart:chart:Background:set_color}
    \index{pygtk\_chart \textit{(package)}!pygtk\_chart.chart \textit{(module)}!pygtk\_chart.chart.Background \textit{(class)}!pygtk\_chart.chart.Background.set\_color \textit{(method)}}

    \vspace{0.5ex}

\hspace{.8\funcindent}\begin{boxedminipage}{\funcwidth}

    \raggedright \textbf{set\_color}(\textit{self}, \textit{color})

    \vspace{-1.5ex}

    \rule{\textwidth}{0.5\fboxrule}
\setlength{\parskip}{2ex}
    The set\_color() method can be used to change the color of the 
    background.

\setlength{\parskip}{1ex}
      \textbf{Parameters}
      \vspace{-1ex}

      \begin{quote}
        \begin{Ventry}{xxxxx}

          \item[color]

          Set the background to be filles with this color.

            {\it (type=gtk.gdk.Color)}

        \end{Ventry}

      \end{quote}

    \end{boxedminipage}

    \label{pygtk_chart:chart:Background:get_color}
    \index{pygtk\_chart \textit{(package)}!pygtk\_chart.chart \textit{(module)}!pygtk\_chart.chart.Background \textit{(class)}!pygtk\_chart.chart.Background.get\_color \textit{(method)}}

    \vspace{0.5ex}

\hspace{.8\funcindent}\begin{boxedminipage}{\funcwidth}

    \raggedright \textbf{get\_color}(\textit{self})

    \vspace{-1.5ex}

    \rule{\textwidth}{0.5\fboxrule}
\setlength{\parskip}{2ex}
    Returns the background's color.

\setlength{\parskip}{1ex}
      \textbf{Return Value}
    \vspace{-1ex}

      \begin{quote}
      gtk.gdk.Color.

      \end{quote}

    \end{boxedminipage}

    \label{pygtk_chart:chart:Background:set_gradient}
    \index{pygtk\_chart \textit{(package)}!pygtk\_chart.chart \textit{(module)}!pygtk\_chart.chart.Background \textit{(class)}!pygtk\_chart.chart.Background.set\_gradient \textit{(method)}}

    \vspace{0.5ex}

\hspace{.8\funcindent}\begin{boxedminipage}{\funcwidth}

    \raggedright \textbf{set\_gradient}(\textit{self}, \textit{color\_start}, \textit{color\_end})

    \vspace{-1.5ex}

    \rule{\textwidth}{0.5\fboxrule}
\setlength{\parskip}{2ex}
    Use set\_gradient() to define a vertical gradient as the background.

\setlength{\parskip}{1ex}
      \textbf{Parameters}
      \vspace{-1ex}

      \begin{quote}
        \begin{Ventry}{xxxxxxxxxxx}

          \item[color\_start]

          The starting (top) color of the gradient.

            {\it (type=gtk.gdk.Color)}

          \item[color\_end]

          The ending (bottom) color of the gradient.

            {\it (type=gtk.gdk.Color)}

        \end{Ventry}

      \end{quote}

    \end{boxedminipage}

    \label{pygtk_chart:chart:Background:get_gradient}
    \index{pygtk\_chart \textit{(package)}!pygtk\_chart.chart \textit{(module)}!pygtk\_chart.chart.Background \textit{(class)}!pygtk\_chart.chart.Background.get\_gradient \textit{(method)}}

    \vspace{0.5ex}

\hspace{.8\funcindent}\begin{boxedminipage}{\funcwidth}

    \raggedright \textbf{get\_gradient}(\textit{self})

    \vspace{-1.5ex}

    \rule{\textwidth}{0.5\fboxrule}
\setlength{\parskip}{2ex}
    Returns the gradient of the background or None.

\setlength{\parskip}{1ex}
      \textbf{Return Value}
    \vspace{-1ex}

      \begin{quote}
      A (gtk.gdk.Color, gtk.gdk.Color) tuple or None.

      \end{quote}

    \end{boxedminipage}

    \label{pygtk_chart:chart:Background:set_image}
    \index{pygtk\_chart \textit{(package)}!pygtk\_chart.chart \textit{(module)}!pygtk\_chart.chart.Background \textit{(class)}!pygtk\_chart.chart.Background.set\_image \textit{(method)}}

    \vspace{0.5ex}

\hspace{.8\funcindent}\begin{boxedminipage}{\funcwidth}

    \raggedright \textbf{set\_image}(\textit{self}, \textit{filename})

    \vspace{-1.5ex}

    \rule{\textwidth}{0.5\fboxrule}
\setlength{\parskip}{2ex}
    The set\_image() method sets the background to be filled with an image.

\setlength{\parskip}{1ex}
      \textbf{Parameters}
      \vspace{-1ex}

      \begin{quote}
        \begin{Ventry}{xxxxxxxx}

          \item[filename]

          Path to the file you want to use as background image. If the file
          does not exists, the background is set to white.

            {\it (type=string)}

        \end{Ventry}

      \end{quote}

    \end{boxedminipage}

    \label{pygtk_chart:chart:Background:get_image}
    \index{pygtk\_chart \textit{(package)}!pygtk\_chart.chart \textit{(module)}!pygtk\_chart.chart.Background \textit{(class)}!pygtk\_chart.chart.Background.get\_image \textit{(method)}}

    \vspace{0.5ex}

\hspace{.8\funcindent}\begin{boxedminipage}{\funcwidth}

    \raggedright \textbf{get\_image}(\textit{self})

\setlength{\parskip}{2ex}
\setlength{\parskip}{1ex}
    \end{boxedminipage}


\large{\textbf{\textit{Inherited from pygtk\_chart.chart\_object.ChartObject\textit{(Section \ref{pygtk_chart:chart_object:ChartObject})}}}}

\begin{quote}
draw(), get\_antialias(), get\_visible(), set\_antialias(), set\_visible()
\end{quote}

\large{\textbf{\textit{Inherited from ??.GObject}}}

\begin{quote}
\_\_cmp\_\_(), \_\_copy\_\_(), \_\_deepcopy\_\_(), \_\_delattr\_\_(), \_\_gdoc\_\_(), \_\_gobject\_init\_\_(), \_\_hash\_\_(), \_\_new\_\_(), \_\_repr\_\_(), \_\_setattr\_\_(), chain(), connect(), connect\_after(), connect\_object(), connect\_object\_after(), disconnect(), disconnect\_by\_func(), emit(), emit\_stop\_by\_name(), freeze\_notify(), get\_data(), get\_properties(), get\_property(), handler\_block(), handler\_block\_by\_func(), handler\_disconnect(), handler\_is\_connected(), handler\_unblock(), handler\_unblock\_by\_func(), notify(), props(), set\_data(), set\_properties(), set\_property(), stop\_emission(), thaw\_notify(), weak\_ref()
\end{quote}

\large{\textbf{\textit{Inherited from object}}}

\begin{quote}
\_\_getattribute\_\_(), \_\_reduce\_\_(), \_\_reduce\_ex\_\_(), \_\_str\_\_()
\end{quote}

%%%%%%%%%%%%%%%%%%%%%%%%%%%%%%%%%%%%%%%%%%%%%%%%%%%%%%%%%%%%%%%%%%%%%%%%%%%
%%                              Properties                               %%
%%%%%%%%%%%%%%%%%%%%%%%%%%%%%%%%%%%%%%%%%%%%%%%%%%%%%%%%%%%%%%%%%%%%%%%%%%%

  \subsubsection{Properties}

    \vspace{-1cm}
\hspace{\varindent}\begin{longtable}{|p{\varnamewidth}|p{\vardescrwidth}|l}
\cline{1-2}
\cline{1-2} \centering \textbf{Name} & \centering \textbf{Description}& \\
\cline{1-2}
\endhead\cline{1-2}\multicolumn{3}{r}{\small\textit{continued on next page}}\\\endfoot\cline{1-2}
\endlastfoot\multicolumn{2}{|l|}{\textit{Inherited from ??.GObject}}\\
\multicolumn{2}{|p{\varwidth}|}{\raggedright \_\_grefcount\_\_}\\
\cline{1-2}
\multicolumn{2}{|l|}{\textit{Inherited from object}}\\
\multicolumn{2}{|p{\varwidth}|}{\raggedright \_\_class\_\_}\\
\cline{1-2}
\end{longtable}


%%%%%%%%%%%%%%%%%%%%%%%%%%%%%%%%%%%%%%%%%%%%%%%%%%%%%%%%%%%%%%%%%%%%%%%%%%%
%%                            Class Variables                            %%
%%%%%%%%%%%%%%%%%%%%%%%%%%%%%%%%%%%%%%%%%%%%%%%%%%%%%%%%%%%%%%%%%%%%%%%%%%%

  \subsubsection{Class Variables}

    \vspace{-1cm}
\hspace{\varindent}\begin{longtable}{|p{\varnamewidth}|p{\vardescrwidth}|l}
\cline{1-2}
\cline{1-2} \centering \textbf{Name} & \centering \textbf{Description}& \\
\cline{1-2}
\endhead\cline{1-2}\multicolumn{3}{r}{\small\textit{continued on next page}}\\\endfoot\cline{1-2}
\endlastfoot\raggedright \_\-\_\-g\-p\-r\-o\-p\-e\-r\-t\-i\-e\-s\-\_\-\_\- & \raggedright \textbf{Value:} 
{\tt \{"color":(gobject.TYPE\_PYOBJECT, "background color", "The\texttt{...}}&\\
\cline{1-2}
\raggedright \_\-\_\-g\-t\-y\-p\-e\-\_\-\_\- & \raggedright \textbf{Value:} 
{\tt {\textless}GType pygtk\_chart+chart+Background (168686296){\textgreater}}&\\
\cline{1-2}
\multicolumn{2}{|l|}{\textit{Inherited from pygtk\_chart.chart\_object.ChartObject \textit{(Section \ref{pygtk_chart:chart_object:ChartObject})}}}\\
\multicolumn{2}{|p{\varwidth}|}{\raggedright \_\_gsignals\_\_}\\
\cline{1-2}
\end{longtable}

    \index{pygtk\_chart \textit{(package)}!pygtk\_chart.chart \textit{(module)}!pygtk\_chart.chart.Background \textit{(class)}|)}

%%%%%%%%%%%%%%%%%%%%%%%%%%%%%%%%%%%%%%%%%%%%%%%%%%%%%%%%%%%%%%%%%%%%%%%%%%%
%%                           Class Description                           %%
%%%%%%%%%%%%%%%%%%%%%%%%%%%%%%%%%%%%%%%%%%%%%%%%%%%%%%%%%%%%%%%%%%%%%%%%%%%

    \index{pygtk\_chart \textit{(package)}!pygtk\_chart.chart \textit{(module)}!pygtk\_chart.chart.Title \textit{(class)}|(}
\subsection{Class Title}

    \label{pygtk_chart:chart:Title}
\begin{tabular}{cccccccccccc}
% Line for object, linespec=[False, False, False, False]
\multicolumn{2}{r}{\settowidth{\BCL}{object}\multirow{2}{\BCL}{object}}
&&
&&
&&
&&
  \\\cline{3-3}
  &&\multicolumn{1}{c|}{}
&&
&&
&&
&&
  \\
% Line for ??.GObject, linespec=[False, False, False]
\multicolumn{4}{r}{\settowidth{\BCL}{??.GObject}\multirow{2}{\BCL}{??.GObject}}
&&
&&
&&
  \\\cline{5-5}
  &&&&\multicolumn{1}{c|}{}
&&
&&
&&
  \\
% Line for pygtk\_chart.chart\_object.ChartObject, linespec=[False, False]
\multicolumn{6}{r}{\settowidth{\BCL}{pygtk\_chart.chart\_object.ChartObject}\multirow{2}{\BCL}{pygtk\_chart.chart\_object.ChartObject}}
&&
&&
  \\\cline{7-7}
  &&&&&&\multicolumn{1}{c|}{}
&&
&&
  \\
% Line for pygtk\_chart.label.Label, linespec=[False]
\multicolumn{8}{r}{\settowidth{\BCL}{pygtk\_chart.label.Label}\multirow{2}{\BCL}{pygtk\_chart.label.Label}}
&&
  \\\cline{9-9}
  &&&&&&&&\multicolumn{1}{c|}{}
&&
  \\
&&&&&&&&\multicolumn{2}{l}{\textbf{pygtk\_chart.chart.Title}}
\end{tabular}

The title of a chart. The title will be drawn centered at the top of the 
chart.

(section) Properties

  The Title class inherits properties from label.Label.

(section) Signals

  The Title class inherits signals from label.Label.


%%%%%%%%%%%%%%%%%%%%%%%%%%%%%%%%%%%%%%%%%%%%%%%%%%%%%%%%%%%%%%%%%%%%%%%%%%%
%%                                Methods                                %%
%%%%%%%%%%%%%%%%%%%%%%%%%%%%%%%%%%%%%%%%%%%%%%%%%%%%%%%%%%%%%%%%%%%%%%%%%%%

  \subsubsection{Methods}

    \vspace{0.5ex}

\hspace{.8\funcindent}\begin{boxedminipage}{\funcwidth}

    \raggedright \textbf{\_\_init\_\_}(\textit{self}, \textit{text}={\tt \texttt{'}\texttt{}\texttt{'}})

\setlength{\parskip}{2ex}
    x.\_\_init\_\_(...) initializes x; see x.\_\_class\_\_.\_\_doc\_\_ for 
    signature

\setlength{\parskip}{1ex}
      Overrides: object.\_\_init\_\_ 	extit{(inherited documentation)}

    \end{boxedminipage}


\large{\textbf{\textit{Inherited from pygtk\_chart.label.Label\textit{(Section \ref{pygtk_chart:label:Label})}}}}

\begin{quote}
do\_get\_property(), do\_set\_property(), get\_allocation(), get\_anchor(), get\_calculated\_dimensions(), get\_color(), get\_fixed(), get\_line\_count(), get\_max\_width(), get\_position(), get\_real\_dimensions(), get\_real\_position(), get\_rotation(), get\_size(), get\_slant(), get\_text(), get\_underline(), get\_weight(), get\_wrap(), set\_anchor(), set\_color(), set\_fixed(), set\_max\_width(), set\_position(), set\_rotation(), set\_size(), set\_slant(), set\_text(), set\_underline(), set\_weight(), set\_wrap()
\end{quote}

\large{\textbf{\textit{Inherited from pygtk\_chart.chart\_object.ChartObject\textit{(Section \ref{pygtk_chart:chart_object:ChartObject})}}}}

\begin{quote}
draw(), get\_antialias(), get\_visible(), set\_antialias(), set\_visible()
\end{quote}

\large{\textbf{\textit{Inherited from ??.GObject}}}

\begin{quote}
\_\_cmp\_\_(), \_\_copy\_\_(), \_\_deepcopy\_\_(), \_\_delattr\_\_(), \_\_gdoc\_\_(), \_\_gobject\_init\_\_(), \_\_hash\_\_(), \_\_new\_\_(), \_\_repr\_\_(), \_\_setattr\_\_(), chain(), connect(), connect\_after(), connect\_object(), connect\_object\_after(), disconnect(), disconnect\_by\_func(), emit(), emit\_stop\_by\_name(), freeze\_notify(), get\_data(), get\_properties(), get\_property(), handler\_block(), handler\_block\_by\_func(), handler\_disconnect(), handler\_is\_connected(), handler\_unblock(), handler\_unblock\_by\_func(), notify(), props(), set\_data(), set\_properties(), set\_property(), stop\_emission(), thaw\_notify(), weak\_ref()
\end{quote}

\large{\textbf{\textit{Inherited from object}}}

\begin{quote}
\_\_getattribute\_\_(), \_\_reduce\_\_(), \_\_reduce\_ex\_\_(), \_\_str\_\_()
\end{quote}

%%%%%%%%%%%%%%%%%%%%%%%%%%%%%%%%%%%%%%%%%%%%%%%%%%%%%%%%%%%%%%%%%%%%%%%%%%%
%%                              Properties                               %%
%%%%%%%%%%%%%%%%%%%%%%%%%%%%%%%%%%%%%%%%%%%%%%%%%%%%%%%%%%%%%%%%%%%%%%%%%%%

  \subsubsection{Properties}

    \vspace{-1cm}
\hspace{\varindent}\begin{longtable}{|p{\varnamewidth}|p{\vardescrwidth}|l}
\cline{1-2}
\cline{1-2} \centering \textbf{Name} & \centering \textbf{Description}& \\
\cline{1-2}
\endhead\cline{1-2}\multicolumn{3}{r}{\small\textit{continued on next page}}\\\endfoot\cline{1-2}
\endlastfoot\multicolumn{2}{|l|}{\textit{Inherited from ??.GObject}}\\
\multicolumn{2}{|p{\varwidth}|}{\raggedright \_\_grefcount\_\_}\\
\cline{1-2}
\multicolumn{2}{|l|}{\textit{Inherited from object}}\\
\multicolumn{2}{|p{\varwidth}|}{\raggedright \_\_class\_\_}\\
\cline{1-2}
\end{longtable}


%%%%%%%%%%%%%%%%%%%%%%%%%%%%%%%%%%%%%%%%%%%%%%%%%%%%%%%%%%%%%%%%%%%%%%%%%%%
%%                            Class Variables                            %%
%%%%%%%%%%%%%%%%%%%%%%%%%%%%%%%%%%%%%%%%%%%%%%%%%%%%%%%%%%%%%%%%%%%%%%%%%%%

  \subsubsection{Class Variables}

    \vspace{-1cm}
\hspace{\varindent}\begin{longtable}{|p{\varnamewidth}|p{\vardescrwidth}|l}
\cline{1-2}
\cline{1-2} \centering \textbf{Name} & \centering \textbf{Description}& \\
\cline{1-2}
\endhead\cline{1-2}\multicolumn{3}{r}{\small\textit{continued on next page}}\\\endfoot\cline{1-2}
\endlastfoot\multicolumn{2}{|l|}{\textit{Inherited from pygtk\_chart.label.Label \textit{(Section \ref{pygtk_chart:label:Label})}}}\\
\multicolumn{2}{|p{\varwidth}|}{\raggedright \_\_gproperties\_\_, \_\_gtype\_\_}\\
\cline{1-2}
\multicolumn{2}{|l|}{\textit{Inherited from pygtk\_chart.chart\_object.ChartObject \textit{(Section \ref{pygtk_chart:chart_object:ChartObject})}}}\\
\multicolumn{2}{|p{\varwidth}|}{\raggedright \_\_gsignals\_\_}\\
\cline{1-2}
\end{longtable}

    \index{pygtk\_chart \textit{(package)}!pygtk\_chart.chart \textit{(module)}!pygtk\_chart.chart.Title \textit{(class)}|)}

%%%%%%%%%%%%%%%%%%%%%%%%%%%%%%%%%%%%%%%%%%%%%%%%%%%%%%%%%%%%%%%%%%%%%%%%%%%
%%                           Class Description                           %%
%%%%%%%%%%%%%%%%%%%%%%%%%%%%%%%%%%%%%%%%%%%%%%%%%%%%%%%%%%%%%%%%%%%%%%%%%%%

    \index{pygtk\_chart \textit{(package)}!pygtk\_chart.chart \textit{(module)}!pygtk\_chart.chart.Area \textit{(class)}|(}
\subsection{Class Area}

    \label{pygtk_chart:chart:Area}
\begin{tabular}{cccccccccc}
% Line for object, linespec=[False, False, False]
\multicolumn{2}{r}{\settowidth{\BCL}{object}\multirow{2}{\BCL}{object}}
&&
&&
&&
  \\\cline{3-3}
  &&\multicolumn{1}{c|}{}
&&
&&
&&
  \\
% Line for ??.GObject, linespec=[False, False]
\multicolumn{4}{r}{\settowidth{\BCL}{??.GObject}\multirow{2}{\BCL}{??.GObject}}
&&
&&
  \\\cline{5-5}
  &&&&\multicolumn{1}{c|}{}
&&
&&
  \\
% Line for pygtk\_chart.chart\_object.ChartObject, linespec=[False]
\multicolumn{6}{r}{\settowidth{\BCL}{pygtk\_chart.chart\_object.ChartObject}\multirow{2}{\BCL}{pygtk\_chart.chart\_object.ChartObject}}
&&
  \\\cline{7-7}
  &&&&&&\multicolumn{1}{c|}{}
&&
  \\
&&&&&&\multicolumn{2}{l}{\textbf{pygtk\_chart.chart.Area}}
\end{tabular}

\textbf{Known Subclasses:}
pygtk\_chart.bar\_chart.Bar,
    pygtk\_chart.pie\_chart.PieArea

This is a base class for classes that represent areas, e.g. the 
pie\_chart.PieArea class and the bar\_chart.Bar class.

(section) Properties

  The Area class inherits properties from chart\_object.ChartObject. 
  Additional properties:

  \begin{itemize}
  \setlength{\parskip}{0.6ex}
    \item name (a unique name for the area, type: string, read only)

    \item value (the value of the area, type: float)

    \item color (the area's color, type: gtk.gdk.Color)

    \item label (a label for the area, type: string)

    \item highlighted (set whether the area should be highlighted, type: 
      boolean).

  \end{itemize}

(section) Signals

  The Area class inherits signals from chart\_object.ChartObject.


%%%%%%%%%%%%%%%%%%%%%%%%%%%%%%%%%%%%%%%%%%%%%%%%%%%%%%%%%%%%%%%%%%%%%%%%%%%
%%                                Methods                                %%
%%%%%%%%%%%%%%%%%%%%%%%%%%%%%%%%%%%%%%%%%%%%%%%%%%%%%%%%%%%%%%%%%%%%%%%%%%%

  \subsubsection{Methods}

    \vspace{0.5ex}

\hspace{.8\funcindent}\begin{boxedminipage}{\funcwidth}

    \raggedright \textbf{\_\_init\_\_}(\textit{self}, \textit{name}, \textit{value}, \textit{title}={\tt \texttt{'}\texttt{}\texttt{'}})

\setlength{\parskip}{2ex}
    x.\_\_init\_\_(...) initializes x; see x.\_\_class\_\_.\_\_doc\_\_ for 
    signature

\setlength{\parskip}{1ex}
      Overrides: object.\_\_init\_\_ 	extit{(inherited documentation)}

    \end{boxedminipage}

    \vspace{0.5ex}

\hspace{.8\funcindent}\begin{boxedminipage}{\funcwidth}

    \raggedright \textbf{do\_get\_property}(\textit{self}, \textit{property})

\setlength{\parskip}{2ex}
\setlength{\parskip}{1ex}
      Overrides: pygtk\_chart.chart\_object.ChartObject.do\_get\_property

    \end{boxedminipage}

    \vspace{0.5ex}

\hspace{.8\funcindent}\begin{boxedminipage}{\funcwidth}

    \raggedright \textbf{do\_set\_property}(\textit{self}, \textit{property}, \textit{value})

\setlength{\parskip}{2ex}
\setlength{\parskip}{1ex}
      Overrides: pygtk\_chart.chart\_object.ChartObject.do\_set\_property

    \end{boxedminipage}

    \label{pygtk_chart:chart:Area:set_value}
    \index{pygtk\_chart \textit{(package)}!pygtk\_chart.chart \textit{(module)}!pygtk\_chart.chart.Area \textit{(class)}!pygtk\_chart.chart.Area.set\_value \textit{(method)}}

    \vspace{0.5ex}

\hspace{.8\funcindent}\begin{boxedminipage}{\funcwidth}

    \raggedright \textbf{set\_value}(\textit{self}, \textit{value})

    \vspace{-1.5ex}

    \rule{\textwidth}{0.5\fboxrule}
\setlength{\parskip}{2ex}
    Set the value of the area.

\setlength{\parskip}{1ex}
      \textbf{Parameters}
      \vspace{-1ex}

      \begin{quote}
        \begin{Ventry}{xxxxx}

          \item[value]

            {\it (type=float.)}

        \end{Ventry}

      \end{quote}

    \end{boxedminipage}

    \label{pygtk_chart:chart:Area:get_value}
    \index{pygtk\_chart \textit{(package)}!pygtk\_chart.chart \textit{(module)}!pygtk\_chart.chart.Area \textit{(class)}!pygtk\_chart.chart.Area.get\_value \textit{(method)}}

    \vspace{0.5ex}

\hspace{.8\funcindent}\begin{boxedminipage}{\funcwidth}

    \raggedright \textbf{get\_value}(\textit{self})

    \vspace{-1.5ex}

    \rule{\textwidth}{0.5\fboxrule}
\setlength{\parskip}{2ex}
    Returns the current value of the area.

\setlength{\parskip}{1ex}
      \textbf{Return Value}
    \vspace{-1ex}

      \begin{quote}
      float.

      \end{quote}

    \end{boxedminipage}

    \label{pygtk_chart:chart:Area:set_color}
    \index{pygtk\_chart \textit{(package)}!pygtk\_chart.chart \textit{(module)}!pygtk\_chart.chart.Area \textit{(class)}!pygtk\_chart.chart.Area.set\_color \textit{(method)}}

    \vspace{0.5ex}

\hspace{.8\funcindent}\begin{boxedminipage}{\funcwidth}

    \raggedright \textbf{set\_color}(\textit{self}, \textit{color})

    \vspace{-1.5ex}

    \rule{\textwidth}{0.5\fboxrule}
\setlength{\parskip}{2ex}
    Set the color of the area.

\setlength{\parskip}{1ex}
      \textbf{Parameters}
      \vspace{-1ex}

      \begin{quote}
        \begin{Ventry}{xxxxx}

          \item[color]

            {\it (type=gtk.gdk.Color.)}

        \end{Ventry}

      \end{quote}

    \end{boxedminipage}

    \label{pygtk_chart:chart:Area:get_color}
    \index{pygtk\_chart \textit{(package)}!pygtk\_chart.chart \textit{(module)}!pygtk\_chart.chart.Area \textit{(class)}!pygtk\_chart.chart.Area.get\_color \textit{(method)}}

    \vspace{0.5ex}

\hspace{.8\funcindent}\begin{boxedminipage}{\funcwidth}

    \raggedright \textbf{get\_color}(\textit{self})

    \vspace{-1.5ex}

    \rule{\textwidth}{0.5\fboxrule}
\setlength{\parskip}{2ex}
    Returns the current color of the area or COLOR\_AUTO.

\setlength{\parskip}{1ex}
      \textbf{Return Value}
    \vspace{-1ex}

      \begin{quote}
      gtk.gdk.Color or COLOR\_AUTO.

      \end{quote}

    \end{boxedminipage}

    \label{pygtk_chart:chart:Area:set_label}
    \index{pygtk\_chart \textit{(package)}!pygtk\_chart.chart \textit{(module)}!pygtk\_chart.chart.Area \textit{(class)}!pygtk\_chart.chart.Area.set\_label \textit{(method)}}

    \vspace{0.5ex}

\hspace{.8\funcindent}\begin{boxedminipage}{\funcwidth}

    \raggedright \textbf{set\_label}(\textit{self}, \textit{label})

    \vspace{-1.5ex}

    \rule{\textwidth}{0.5\fboxrule}
\setlength{\parskip}{2ex}
    Set the label for the area.

\setlength{\parskip}{1ex}
      \textbf{Parameters}
      \vspace{-1ex}

      \begin{quote}
        \begin{Ventry}{xxxxx}

          \item[label]

          the new label

            {\it (type=string.)}

        \end{Ventry}

      \end{quote}

    \end{boxedminipage}

    \label{pygtk_chart:chart:Area:get_label}
    \index{pygtk\_chart \textit{(package)}!pygtk\_chart.chart \textit{(module)}!pygtk\_chart.chart.Area \textit{(class)}!pygtk\_chart.chart.Area.get\_label \textit{(method)}}

    \vspace{0.5ex}

\hspace{.8\funcindent}\begin{boxedminipage}{\funcwidth}

    \raggedright \textbf{get\_label}(\textit{self})

    \vspace{-1.5ex}

    \rule{\textwidth}{0.5\fboxrule}
\setlength{\parskip}{2ex}
    Returns the current label of the area.

\setlength{\parskip}{1ex}
      \textbf{Return Value}
    \vspace{-1ex}

      \begin{quote}
      string.

      \end{quote}

    \end{boxedminipage}

    \label{pygtk_chart:chart:Area:set_highlighted}
    \index{pygtk\_chart \textit{(package)}!pygtk\_chart.chart \textit{(module)}!pygtk\_chart.chart.Area \textit{(class)}!pygtk\_chart.chart.Area.set\_highlighted \textit{(method)}}

    \vspace{0.5ex}

\hspace{.8\funcindent}\begin{boxedminipage}{\funcwidth}

    \raggedright \textbf{set\_highlighted}(\textit{self}, \textit{highlighted})

    \vspace{-1.5ex}

    \rule{\textwidth}{0.5\fboxrule}
\setlength{\parskip}{2ex}
    Set whether the area should be highlighted.

\setlength{\parskip}{1ex}
      \textbf{Parameters}
      \vspace{-1ex}

      \begin{quote}
        \begin{Ventry}{xxxxxxxxxxx}

          \item[highlighted]

            {\it (type=boolean.)}

        \end{Ventry}

      \end{quote}

    \end{boxedminipage}

    \label{pygtk_chart:chart:Area:get_highlighted}
    \index{pygtk\_chart \textit{(package)}!pygtk\_chart.chart \textit{(module)}!pygtk\_chart.chart.Area \textit{(class)}!pygtk\_chart.chart.Area.get\_highlighted \textit{(method)}}

    \vspace{0.5ex}

\hspace{.8\funcindent}\begin{boxedminipage}{\funcwidth}

    \raggedright \textbf{get\_highlighted}(\textit{self})

    \vspace{-1.5ex}

    \rule{\textwidth}{0.5\fboxrule}
\setlength{\parskip}{2ex}
    Returns True if the area is currently highlighted.

\setlength{\parskip}{1ex}
      \textbf{Return Value}
    \vspace{-1ex}

      \begin{quote}
      boolean.

      \end{quote}

    \end{boxedminipage}


\large{\textbf{\textit{Inherited from pygtk\_chart.chart\_object.ChartObject\textit{(Section \ref{pygtk_chart:chart_object:ChartObject})}}}}

\begin{quote}
draw(), get\_antialias(), get\_visible(), set\_antialias(), set\_visible()
\end{quote}

\large{\textbf{\textit{Inherited from ??.GObject}}}

\begin{quote}
\_\_cmp\_\_(), \_\_copy\_\_(), \_\_deepcopy\_\_(), \_\_delattr\_\_(), \_\_gdoc\_\_(), \_\_gobject\_init\_\_(), \_\_hash\_\_(), \_\_new\_\_(), \_\_repr\_\_(), \_\_setattr\_\_(), chain(), connect(), connect\_after(), connect\_object(), connect\_object\_after(), disconnect(), disconnect\_by\_func(), emit(), emit\_stop\_by\_name(), freeze\_notify(), get\_data(), get\_properties(), get\_property(), handler\_block(), handler\_block\_by\_func(), handler\_disconnect(), handler\_is\_connected(), handler\_unblock(), handler\_unblock\_by\_func(), notify(), props(), set\_data(), set\_properties(), set\_property(), stop\_emission(), thaw\_notify(), weak\_ref()
\end{quote}

\large{\textbf{\textit{Inherited from object}}}

\begin{quote}
\_\_getattribute\_\_(), \_\_reduce\_\_(), \_\_reduce\_ex\_\_(), \_\_str\_\_()
\end{quote}

%%%%%%%%%%%%%%%%%%%%%%%%%%%%%%%%%%%%%%%%%%%%%%%%%%%%%%%%%%%%%%%%%%%%%%%%%%%
%%                              Properties                               %%
%%%%%%%%%%%%%%%%%%%%%%%%%%%%%%%%%%%%%%%%%%%%%%%%%%%%%%%%%%%%%%%%%%%%%%%%%%%

  \subsubsection{Properties}

    \vspace{-1cm}
\hspace{\varindent}\begin{longtable}{|p{\varnamewidth}|p{\vardescrwidth}|l}
\cline{1-2}
\cline{1-2} \centering \textbf{Name} & \centering \textbf{Description}& \\
\cline{1-2}
\endhead\cline{1-2}\multicolumn{3}{r}{\small\textit{continued on next page}}\\\endfoot\cline{1-2}
\endlastfoot\multicolumn{2}{|l|}{\textit{Inherited from ??.GObject}}\\
\multicolumn{2}{|p{\varwidth}|}{\raggedright \_\_grefcount\_\_}\\
\cline{1-2}
\multicolumn{2}{|l|}{\textit{Inherited from object}}\\
\multicolumn{2}{|p{\varwidth}|}{\raggedright \_\_class\_\_}\\
\cline{1-2}
\end{longtable}


%%%%%%%%%%%%%%%%%%%%%%%%%%%%%%%%%%%%%%%%%%%%%%%%%%%%%%%%%%%%%%%%%%%%%%%%%%%
%%                            Class Variables                            %%
%%%%%%%%%%%%%%%%%%%%%%%%%%%%%%%%%%%%%%%%%%%%%%%%%%%%%%%%%%%%%%%%%%%%%%%%%%%

  \subsubsection{Class Variables}

    \vspace{-1cm}
\hspace{\varindent}\begin{longtable}{|p{\varnamewidth}|p{\vardescrwidth}|l}
\cline{1-2}
\cline{1-2} \centering \textbf{Name} & \centering \textbf{Description}& \\
\cline{1-2}
\endhead\cline{1-2}\multicolumn{3}{r}{\small\textit{continued on next page}}\\\endfoot\cline{1-2}
\endlastfoot\raggedright \_\-\_\-g\-p\-r\-o\-p\-e\-r\-t\-i\-e\-s\-\_\-\_\- & \raggedright \textbf{Value:} 
{\tt \{"name":(gobject.TYPE\_STRING, "area name", "A unique name\texttt{...}}&\\
\cline{1-2}
\raggedright \_\-\_\-g\-t\-y\-p\-e\-\_\-\_\- & \raggedright \textbf{Value:} 
{\tt {\textless}GType pygtk\_chart+chart+Area (168686448){\textgreater}}&\\
\cline{1-2}
\multicolumn{2}{|l|}{\textit{Inherited from pygtk\_chart.chart\_object.ChartObject \textit{(Section \ref{pygtk_chart:chart_object:ChartObject})}}}\\
\multicolumn{2}{|p{\varwidth}|}{\raggedright \_\_gsignals\_\_}\\
\cline{1-2}
\end{longtable}

    \index{pygtk\_chart \textit{(package)}!pygtk\_chart.chart \textit{(module)}!pygtk\_chart.chart.Area \textit{(class)}|)}
    \index{pygtk\_chart \textit{(package)}!pygtk\_chart.chart \textit{(module)}|)}
