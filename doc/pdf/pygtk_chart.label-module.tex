%
% API Documentation for pygtkChart
% Module pygtk_chart.label
%
% Generated by epydoc 3.0.1
% [Mon Sep 28 11:06:52 2009]
%

%%%%%%%%%%%%%%%%%%%%%%%%%%%%%%%%%%%%%%%%%%%%%%%%%%%%%%%%%%%%%%%%%%%%%%%%%%%
%%                          Module Description                           %%
%%%%%%%%%%%%%%%%%%%%%%%%%%%%%%%%%%%%%%%%%%%%%%%%%%%%%%%%%%%%%%%%%%%%%%%%%%%

    \index{pygtk\_chart \textit{(package)}!pygtk\_chart.label \textit{(module)}|(}
\section{Module pygtk\_chart.label}

    \label{pygtk_chart:label}
Contains the Label class.

Author: Sven Festersen (sven@sven-festersen.de)


%%%%%%%%%%%%%%%%%%%%%%%%%%%%%%%%%%%%%%%%%%%%%%%%%%%%%%%%%%%%%%%%%%%%%%%%%%%
%%                               Functions                               %%
%%%%%%%%%%%%%%%%%%%%%%%%%%%%%%%%%%%%%%%%%%%%%%%%%%%%%%%%%%%%%%%%%%%%%%%%%%%

  \subsection{Functions}

    \label{pygtk_chart:label:begin_drawing}
    \index{pygtk\_chart \textit{(package)}!pygtk\_chart.label \textit{(module)}!pygtk\_chart.label.begin\_drawing \textit{(function)}}

    \vspace{0.5ex}

\hspace{.8\funcindent}\begin{boxedminipage}{\funcwidth}

    \raggedright \textbf{begin\_drawing}()

\setlength{\parskip}{2ex}
\setlength{\parskip}{1ex}
    \end{boxedminipage}

    \label{pygtk_chart:label:finish_drawing}
    \index{pygtk\_chart \textit{(package)}!pygtk\_chart.label \textit{(module)}!pygtk\_chart.label.finish\_drawing \textit{(function)}}

    \vspace{0.5ex}

\hspace{.8\funcindent}\begin{boxedminipage}{\funcwidth}

    \raggedright \textbf{finish\_drawing}()

\setlength{\parskip}{2ex}
\setlength{\parskip}{1ex}
    \end{boxedminipage}

    \label{pygtk_chart:label:register_label}
    \index{pygtk\_chart \textit{(package)}!pygtk\_chart.label \textit{(module)}!pygtk\_chart.label.register\_label \textit{(function)}}

    \vspace{0.5ex}

\hspace{.8\funcindent}\begin{boxedminipage}{\funcwidth}

    \raggedright \textbf{register\_label}(\textit{label})

\setlength{\parskip}{2ex}
\setlength{\parskip}{1ex}
    \end{boxedminipage}

    \label{pygtk_chart:label:get_registered_labels}
    \index{pygtk\_chart \textit{(package)}!pygtk\_chart.label \textit{(module)}!pygtk\_chart.label.get\_registered\_labels \textit{(function)}}

    \vspace{0.5ex}

\hspace{.8\funcindent}\begin{boxedminipage}{\funcwidth}

    \raggedright \textbf{get\_registered\_labels}()

\setlength{\parskip}{2ex}
\setlength{\parskip}{1ex}
    \end{boxedminipage}

    \label{pygtk_chart:label:get_text_pos}
    \index{pygtk\_chart \textit{(package)}!pygtk\_chart.label \textit{(module)}!pygtk\_chart.label.get\_text\_pos \textit{(function)}}

    \vspace{0.5ex}

\hspace{.8\funcindent}\begin{boxedminipage}{\funcwidth}

    \raggedright \textbf{get\_text\_pos}(\textit{layout}, \textit{pos}, \textit{anchor}, \textit{angle})

    \vspace{-1.5ex}

    \rule{\textwidth}{0.5\fboxrule}
\setlength{\parskip}{2ex}
    This function calculates the position of bottom left point of the 
    layout respecting the given anchor point.

\setlength{\parskip}{1ex}
      \textbf{Return Value}
    \vspace{-1ex}

      \begin{quote}
      (x, y) pair

      \end{quote}

    \end{boxedminipage}


%%%%%%%%%%%%%%%%%%%%%%%%%%%%%%%%%%%%%%%%%%%%%%%%%%%%%%%%%%%%%%%%%%%%%%%%%%%
%%                               Variables                               %%
%%%%%%%%%%%%%%%%%%%%%%%%%%%%%%%%%%%%%%%%%%%%%%%%%%%%%%%%%%%%%%%%%%%%%%%%%%%

  \subsection{Variables}

    \vspace{-1cm}
\hspace{\varindent}\begin{longtable}{|p{\varnamewidth}|p{\vardescrwidth}|l}
\cline{1-2}
\cline{1-2} \centering \textbf{Name} & \centering \textbf{Description}& \\
\cline{1-2}
\endhead\cline{1-2}\multicolumn{3}{r}{\small\textit{continued on next page}}\\\endfoot\cline{1-2}
\endlastfoot\raggedright A\-N\-C\-H\-O\-R\-\_\-B\-O\-T\-T\-O\-M\-\_\-L\-E\-F\-T\- & \raggedright \textbf{Value:} 
{\tt 0}&\\
\cline{1-2}
\raggedright A\-N\-C\-H\-O\-R\-\_\-T\-O\-P\-\_\-L\-E\-F\-T\- & \raggedright \textbf{Value:} 
{\tt 1}&\\
\cline{1-2}
\raggedright A\-N\-C\-H\-O\-R\-\_\-T\-O\-P\-\_\-R\-I\-G\-H\-T\- & \raggedright \textbf{Value:} 
{\tt 2}&\\
\cline{1-2}
\raggedright A\-N\-C\-H\-O\-R\-\_\-B\-O\-T\-T\-O\-M\-\_\-R\-I\-G\-H\-T\- & \raggedright \textbf{Value:} 
{\tt 4}&\\
\cline{1-2}
\raggedright A\-N\-C\-H\-O\-R\-\_\-C\-E\-N\-T\-E\-R\- & \raggedright \textbf{Value:} 
{\tt 5}&\\
\cline{1-2}
\raggedright A\-N\-C\-H\-O\-R\-\_\-T\-O\-P\-\_\-C\-E\-N\-T\-E\-R\- & \raggedright \textbf{Value:} 
{\tt 6}&\\
\cline{1-2}
\raggedright A\-N\-C\-H\-O\-R\-\_\-B\-O\-T\-T\-O\-M\-\_\-C\-E\-N\-T\-E\-R\- & \raggedright \textbf{Value:} 
{\tt 7}&\\
\cline{1-2}
\raggedright A\-N\-C\-H\-O\-R\-\_\-L\-E\-F\-T\-\_\-C\-E\-N\-T\-E\-R\- & \raggedright \textbf{Value:} 
{\tt 8}&\\
\cline{1-2}
\raggedright A\-N\-C\-H\-O\-R\-\_\-R\-I\-G\-H\-T\-\_\-C\-E\-N\-T\-E\-R\- & \raggedright \textbf{Value:} 
{\tt 9}&\\
\cline{1-2}
\raggedright U\-N\-D\-E\-R\-L\-I\-N\-E\-\_\-N\-O\-N\-E\- & \raggedright \textbf{Value:} 
{\tt {\textless}enum PANGO\_UNDERLINE\_NONE of type PangoUnderline{\textgreater}}&\\
\cline{1-2}
\raggedright U\-N\-D\-E\-R\-L\-I\-N\-E\-\_\-S\-I\-N\-G\-L\-E\- & \raggedright \textbf{Value:} 
{\tt {\textless}enum PANGO\_UNDERLINE\_SINGLE of type PangoUnderline{\textgreater}}&\\
\cline{1-2}
\raggedright U\-N\-D\-E\-R\-L\-I\-N\-E\-\_\-D\-O\-U\-B\-L\-E\- & \raggedright \textbf{Value:} 
{\tt {\textless}enum PANGO\_UNDERLINE\_DOUBLE of type PangoUnderline{\textgreater}}&\\
\cline{1-2}
\raggedright U\-N\-D\-E\-R\-L\-I\-N\-E\-\_\-L\-O\-W\- & \raggedright \textbf{Value:} 
{\tt {\textless}enum PANGO\_UNDERLINE\_LOW of type PangoUnderline{\textgreater}}&\\
\cline{1-2}
\raggedright S\-T\-Y\-L\-E\-\_\-N\-O\-R\-M\-A\-L\- & \raggedright \textbf{Value:} 
{\tt {\textless}enum PANGO\_STYLE\_NORMAL of type PangoStyle{\textgreater}}&\\
\cline{1-2}
\raggedright S\-T\-Y\-L\-E\-\_\-O\-B\-L\-I\-Q\-U\-E\- & \raggedright \textbf{Value:} 
{\tt {\textless}enum PANGO\_STYLE\_OBLIQUE of type PangoStyle{\textgreater}}&\\
\cline{1-2}
\raggedright S\-T\-Y\-L\-E\-\_\-I\-T\-A\-L\-I\-C\- & \raggedright \textbf{Value:} 
{\tt {\textless}enum PANGO\_STYLE\_ITALIC of type PangoStyle{\textgreater}}&\\
\cline{1-2}
\raggedright W\-E\-I\-G\-H\-T\-\_\-U\-L\-T\-R\-A\-L\-I\-G\-H\-T\- & \raggedright \textbf{Value:} 
{\tt {\textless}enum PANGO\_WEIGHT\_ULTRALIGHT of type PangoWeight{\textgreater}}&\\
\cline{1-2}
\raggedright W\-E\-I\-G\-H\-T\-\_\-L\-I\-G\-H\-T\- & \raggedright \textbf{Value:} 
{\tt {\textless}enum PANGO\_WEIGHT\_LIGHT of type PangoWeight{\textgreater}}&\\
\cline{1-2}
\raggedright W\-E\-I\-G\-H\-T\-\_\-N\-O\-R\-M\-A\-L\- & \raggedright \textbf{Value:} 
{\tt {\textless}enum PANGO\_WEIGHT\_NORMAL of type PangoWeight{\textgreater}}&\\
\cline{1-2}
\raggedright W\-E\-I\-G\-H\-T\-\_\-B\-O\-L\-D\- & \raggedright \textbf{Value:} 
{\tt {\textless}enum PANGO\_WEIGHT\_BOLD of type PangoWeight{\textgreater}}&\\
\cline{1-2}
\raggedright W\-E\-I\-G\-H\-T\-\_\-U\-L\-T\-R\-A\-B\-O\-L\-D\- & \raggedright \textbf{Value:} 
{\tt {\textless}enum PANGO\_WEIGHT\_ULTRABOLD of type PangoWeight{\textgreater}}&\\
\cline{1-2}
\raggedright W\-E\-I\-G\-H\-T\-\_\-H\-E\-A\-V\-Y\- & \raggedright \textbf{Value:} 
{\tt {\textless}enum PANGO\_WEIGHT\_HEAVY of type PangoWeight{\textgreater}}&\\
\cline{1-2}
\raggedright D\-R\-A\-W\-I\-N\-G\-\_\-I\-N\-I\-T\-I\-A\-L\-I\-Z\-E\-D\- & \raggedright \textbf{Value:} 
{\tt False}&\\
\cline{1-2}
\raggedright R\-E\-G\-I\-S\-T\-E\-R\-E\-D\-\_\-L\-A\-B\-E\-L\-S\- & \raggedright \textbf{Value:} 
{\tt \texttt{[}\texttt{]}}&\\
\cline{1-2}
\end{longtable}


%%%%%%%%%%%%%%%%%%%%%%%%%%%%%%%%%%%%%%%%%%%%%%%%%%%%%%%%%%%%%%%%%%%%%%%%%%%
%%                           Class Description                           %%
%%%%%%%%%%%%%%%%%%%%%%%%%%%%%%%%%%%%%%%%%%%%%%%%%%%%%%%%%%%%%%%%%%%%%%%%%%%

    \index{pygtk\_chart \textit{(package)}!pygtk\_chart.label \textit{(module)}!pygtk\_chart.label.Label \textit{(class)}|(}
\subsection{Class Label}

    \label{pygtk_chart:label:Label}
\begin{tabular}{cccccccccc}
% Line for object, linespec=[False, False, False]
\multicolumn{2}{r}{\settowidth{\BCL}{object}\multirow{2}{\BCL}{object}}
&&
&&
&&
  \\\cline{3-3}
  &&\multicolumn{1}{c|}{}
&&
&&
&&
  \\
% Line for ??.GObject, linespec=[False, False]
\multicolumn{4}{r}{\settowidth{\BCL}{??.GObject}\multirow{2}{\BCL}{??.GObject}}
&&
&&
  \\\cline{5-5}
  &&&&\multicolumn{1}{c|}{}
&&
&&
  \\
% Line for pygtk\_chart.chart\_object.ChartObject, linespec=[False]
\multicolumn{6}{r}{\settowidth{\BCL}{pygtk\_chart.chart\_object.ChartObject}\multirow{2}{\BCL}{pygtk\_chart.chart\_object.ChartObject}}
&&
  \\\cline{7-7}
  &&&&&&\multicolumn{1}{c|}{}
&&
  \\
&&&&&&\multicolumn{2}{l}{\textbf{pygtk\_chart.label.Label}}
\end{tabular}

\textbf{Known Subclasses:} pygtk\_chart.chart.Title

This class is used for drawing all the text on the chart widgets. It uses 
the pango layout engine.

(section) Properties

  The Label class inherits properties from chart\_object.ChartObject. 
  Additional properties:

  \begin{itemize}
  \setlength{\parskip}{0.6ex}
    \item color (the label's color, type: gtk.gdk.Color)

    \item text (text to display, type: string)

    \item position (the label's position, type: pair of float)

    \item anchor (the anchor that should be used to position the label, type: 
      an anchor constant)

    \item underline (sets the type of underline, type; an underline constant)

    \item max-width (the maximum width of the label in px, type: int)

    \item rotation (angle of rotation in degrees, type: int)

    \item size (the size of the label's text in px, type: int)

    \item slant (the font slant, type: a slant style constant)

    \item weight (the font weight, type: a font weight constant)

    \item fixed (sets whether the position of the label may be changed 
      dynamicly or not, type: boolean)

    \item wrap (sets whether the label's text should be wrapped if it's longer 
      than max-width, type: boolean).

  \end{itemize}

(section) Signals

  The Label class inherits signals from chart\_object.ChartObject.


%%%%%%%%%%%%%%%%%%%%%%%%%%%%%%%%%%%%%%%%%%%%%%%%%%%%%%%%%%%%%%%%%%%%%%%%%%%
%%                                Methods                                %%
%%%%%%%%%%%%%%%%%%%%%%%%%%%%%%%%%%%%%%%%%%%%%%%%%%%%%%%%%%%%%%%%%%%%%%%%%%%

  \subsubsection{Methods}

    \vspace{0.5ex}

\hspace{.8\funcindent}\begin{boxedminipage}{\funcwidth}

    \raggedright \textbf{\_\_init\_\_}(\textit{self}, \textit{position}, \textit{text}, \textit{size}={\tt None}, \textit{slant}={\tt {\textless}enum PANGO\_STYLE\_NORMAL of type PangoStyle{\textgreater}}, \textit{weight}={\tt {\textless}enum PANGO\_WEIGHT\_NORMAL of type PangoWeight{\textgreater}}, \textit{underline}={\tt {\textless}enum PANGO\_UNDERLINE\_NONE of type PangoUnderline{\textgreater}}, \textit{anchor}={\tt 0}, \textit{max\_width}={\tt 99999}, \textit{fixed}={\tt False})

\setlength{\parskip}{2ex}
    x.\_\_init\_\_(...) initializes x; see x.\_\_class\_\_.\_\_doc\_\_ for 
    signature

\setlength{\parskip}{1ex}
      Overrides: object.\_\_init\_\_ 	extit{(inherited documentation)}

    \end{boxedminipage}

    \vspace{0.5ex}

\hspace{.8\funcindent}\begin{boxedminipage}{\funcwidth}

    \raggedright \textbf{do\_get\_property}(\textit{self}, \textit{property})

\setlength{\parskip}{2ex}
\setlength{\parskip}{1ex}
      Overrides: pygtk\_chart.chart\_object.ChartObject.do\_get\_property

    \end{boxedminipage}

    \vspace{0.5ex}

\hspace{.8\funcindent}\begin{boxedminipage}{\funcwidth}

    \raggedright \textbf{do\_set\_property}(\textit{self}, \textit{property}, \textit{value})

\setlength{\parskip}{2ex}
\setlength{\parskip}{1ex}
      Overrides: pygtk\_chart.chart\_object.ChartObject.do\_set\_property

    \end{boxedminipage}

    \label{pygtk_chart:label:Label:get_calculated_dimensions}
    \index{pygtk\_chart \textit{(package)}!pygtk\_chart.label \textit{(module)}!pygtk\_chart.label.Label \textit{(class)}!pygtk\_chart.label.Label.get\_calculated\_dimensions \textit{(method)}}

    \vspace{0.5ex}

\hspace{.8\funcindent}\begin{boxedminipage}{\funcwidth}

    \raggedright \textbf{get\_calculated\_dimensions}(\textit{self}, \textit{context}, \textit{rect})

\setlength{\parskip}{2ex}
\setlength{\parskip}{1ex}
    \end{boxedminipage}

    \label{pygtk_chart:label:Label:set_text}
    \index{pygtk\_chart \textit{(package)}!pygtk\_chart.label \textit{(module)}!pygtk\_chart.label.Label \textit{(class)}!pygtk\_chart.label.Label.set\_text \textit{(method)}}

    \vspace{0.5ex}

\hspace{.8\funcindent}\begin{boxedminipage}{\funcwidth}

    \raggedright \textbf{set\_text}(\textit{self}, \textit{text})

    \vspace{-1.5ex}

    \rule{\textwidth}{0.5\fboxrule}
\setlength{\parskip}{2ex}
    Use this method to set the text that should be displayed by the label.

\setlength{\parskip}{1ex}
      \textbf{Parameters}
      \vspace{-1ex}

      \begin{quote}
        \begin{Ventry}{xxxx}

          \item[text]

          the text to display.

            {\it (type=string)}

        \end{Ventry}

      \end{quote}

    \end{boxedminipage}

    \label{pygtk_chart:label:Label:get_text}
    \index{pygtk\_chart \textit{(package)}!pygtk\_chart.label \textit{(module)}!pygtk\_chart.label.Label \textit{(class)}!pygtk\_chart.label.Label.get\_text \textit{(method)}}

    \vspace{0.5ex}

\hspace{.8\funcindent}\begin{boxedminipage}{\funcwidth}

    \raggedright \textbf{get\_text}(\textit{self})

    \vspace{-1.5ex}

    \rule{\textwidth}{0.5\fboxrule}
\setlength{\parskip}{2ex}
    Returns the text currently displayed.

\setlength{\parskip}{1ex}
      \textbf{Return Value}
    \vspace{-1ex}

      \begin{quote}
      string.

      \end{quote}

    \end{boxedminipage}

    \label{pygtk_chart:label:Label:set_color}
    \index{pygtk\_chart \textit{(package)}!pygtk\_chart.label \textit{(module)}!pygtk\_chart.label.Label \textit{(class)}!pygtk\_chart.label.Label.set\_color \textit{(method)}}

    \vspace{0.5ex}

\hspace{.8\funcindent}\begin{boxedminipage}{\funcwidth}

    \raggedright \textbf{set\_color}(\textit{self}, \textit{color})

    \vspace{-1.5ex}

    \rule{\textwidth}{0.5\fboxrule}
\setlength{\parskip}{2ex}
    Set the color of the label. color has to be a gtk.gdk.Color.

\setlength{\parskip}{1ex}
      \textbf{Parameters}
      \vspace{-1ex}

      \begin{quote}
        \begin{Ventry}{xxxxx}

          \item[color]

          the color of the label

            {\it (type=gtk.gdk.Color.)}

        \end{Ventry}

      \end{quote}

    \end{boxedminipage}

    \label{pygtk_chart:label:Label:get_color}
    \index{pygtk\_chart \textit{(package)}!pygtk\_chart.label \textit{(module)}!pygtk\_chart.label.Label \textit{(class)}!pygtk\_chart.label.Label.get\_color \textit{(method)}}

    \vspace{0.5ex}

\hspace{.8\funcindent}\begin{boxedminipage}{\funcwidth}

    \raggedright \textbf{get\_color}(\textit{self})

    \vspace{-1.5ex}

    \rule{\textwidth}{0.5\fboxrule}
\setlength{\parskip}{2ex}
    Returns the current color of the label.

\setlength{\parskip}{1ex}
      \textbf{Return Value}
    \vspace{-1ex}

      \begin{quote}
      gtk.gdk.Color.

      \end{quote}

    \end{boxedminipage}

    \label{pygtk_chart:label:Label:set_position}
    \index{pygtk\_chart \textit{(package)}!pygtk\_chart.label \textit{(module)}!pygtk\_chart.label.Label \textit{(class)}!pygtk\_chart.label.Label.set\_position \textit{(method)}}

    \vspace{0.5ex}

\hspace{.8\funcindent}\begin{boxedminipage}{\funcwidth}

    \raggedright \textbf{set\_position}(\textit{self}, \textit{pos})

    \vspace{-1.5ex}

    \rule{\textwidth}{0.5\fboxrule}
\setlength{\parskip}{2ex}
    Set the position of the label. pos has to be a x,y pair of absolute 
    pixel coordinates on the widget. The position is not the actual 
    position but the position of the Label's anchor point (see 
    \texttt{set\_anchor} for details).

\setlength{\parskip}{1ex}
      \textbf{Parameters}
      \vspace{-1ex}

      \begin{quote}
        \begin{Ventry}{xxx}

          \item[pos]

          new position of the label

            {\it (type=pair of (x, y).)}

        \end{Ventry}

      \end{quote}

    \end{boxedminipage}

    \label{pygtk_chart:label:Label:get_position}
    \index{pygtk\_chart \textit{(package)}!pygtk\_chart.label \textit{(module)}!pygtk\_chart.label.Label \textit{(class)}!pygtk\_chart.label.Label.get\_position \textit{(method)}}

    \vspace{0.5ex}

\hspace{.8\funcindent}\begin{boxedminipage}{\funcwidth}

    \raggedright \textbf{get\_position}(\textit{self})

    \vspace{-1.5ex}

    \rule{\textwidth}{0.5\fboxrule}
\setlength{\parskip}{2ex}
    Returns the current position of the label.

\setlength{\parskip}{1ex}
      \textbf{Return Value}
    \vspace{-1ex}

      \begin{quote}
      pair of (x, y).

      \end{quote}

    \end{boxedminipage}

    \label{pygtk_chart:label:Label:set_anchor}
    \index{pygtk\_chart \textit{(package)}!pygtk\_chart.label \textit{(module)}!pygtk\_chart.label.Label \textit{(class)}!pygtk\_chart.label.Label.set\_anchor \textit{(method)}}

    \vspace{0.5ex}

\hspace{.8\funcindent}\begin{boxedminipage}{\funcwidth}

    \raggedright \textbf{set\_anchor}(\textit{self}, \textit{anchor})

    \vspace{-1.5ex}

    \rule{\textwidth}{0.5\fboxrule}
\setlength{\parskip}{2ex}
    Set the anchor point of the label. The anchor point is the a point on 
    the label's edge that has the position you set with set\_position(). 
    anchor has to be one of the following constants:

    \begin{itemize}
    \setlength{\parskip}{0.6ex}
      \item label.ANCHOR\_BOTTOM\_LEFT

      \item label.ANCHOR\_TOP\_LEFT

      \item label.ANCHOR\_TOP\_RIGHT

      \item label.ANCHOR\_BOTTOM\_RIGHT

      \item label.ANCHOR\_CENTER

      \item label.ANCHOR\_TOP\_CENTER

      \item label.ANCHOR\_BOTTOM\_CENTER

      \item label.ANCHOR\_LEFT\_CENTER

      \item label.ANCHOR\_RIGHT\_CENTER

    \end{itemize}

    The meaning of the constants is illustrated below::

\begin{alltt}
    ANCHOR\_TOP\_LEFT     ANCHOR\_TOP\_CENTER   ANCHOR\_TOP\_RIGHT
                   *           *           *
                     \#\#\#\#\#\#\#\#\#\#\#\#\#\#\#\#\#\#\#\#\#
ANCHOR\_LEFT\_CENTER * \#         *         \# * ANCHOR\_RIGHT\_CENTER
                     \#\#\#\#\#\#\#\#\#\#\#\#\#\#\#\#\#\#\#\#\#
                   *           *           *
 ANCHOR\_BOTTOM\_LEFT   ANCHOR\_BOTTOM\_CENTER  ANCHOR\_BOTTOM\_RIGHT\end{alltt}

    The point in the center is of course referred to by constant 
    label.ANCHOR\_CENTER.

\setlength{\parskip}{1ex}
      \textbf{Parameters}
      \vspace{-1ex}

      \begin{quote}
        \begin{Ventry}{xxxxxx}

          \item[anchor]

          the anchor point of the label

            {\it (type=one of the constants described above.)}

        \end{Ventry}

      \end{quote}

    \end{boxedminipage}

    \label{pygtk_chart:label:Label:get_anchor}
    \index{pygtk\_chart \textit{(package)}!pygtk\_chart.label \textit{(module)}!pygtk\_chart.label.Label \textit{(class)}!pygtk\_chart.label.Label.get\_anchor \textit{(method)}}

    \vspace{0.5ex}

\hspace{.8\funcindent}\begin{boxedminipage}{\funcwidth}

    \raggedright \textbf{get\_anchor}(\textit{self})

    \vspace{-1.5ex}

    \rule{\textwidth}{0.5\fboxrule}
\setlength{\parskip}{2ex}
    Returns the current anchor point that's used to position the label. See
    \texttt{set\_anchor} for details.

\setlength{\parskip}{1ex}
      \textbf{Return Value}
    \vspace{-1ex}

      \begin{quote}
      one of the anchor constants described in \texttt{set\_anchor}.

      \end{quote}

    \end{boxedminipage}

    \label{pygtk_chart:label:Label:set_underline}
    \index{pygtk\_chart \textit{(package)}!pygtk\_chart.label \textit{(module)}!pygtk\_chart.label.Label \textit{(class)}!pygtk\_chart.label.Label.set\_underline \textit{(method)}}

    \vspace{0.5ex}

\hspace{.8\funcindent}\begin{boxedminipage}{\funcwidth}

    \raggedright \textbf{set\_underline}(\textit{self}, \textit{underline})

    \vspace{-1.5ex}

    \rule{\textwidth}{0.5\fboxrule}
\setlength{\parskip}{2ex}
    Set the underline style of the label. underline has to be one of the 
    following constants:

    \begin{itemize}
    \setlength{\parskip}{0.6ex}
      \item label.UNDERLINE\_NONE: do not underline the text

      \item label.UNDERLINE\_SINGLE: draw a single underline (the normal 
        underline method)

      \item label.UNDERLINE\_DOUBLE: draw a double underline

      \item label.UNDERLINE\_LOW; draw a single low underline.

    \end{itemize}

\setlength{\parskip}{1ex}
      \textbf{Parameters}
      \vspace{-1ex}

      \begin{quote}
        \begin{Ventry}{xxxxxxxxx}

          \item[underline]

          the underline style

            {\it (type=one of the constants above.)}

        \end{Ventry}

      \end{quote}

    \end{boxedminipage}

    \label{pygtk_chart:label:Label:get_underline}
    \index{pygtk\_chart \textit{(package)}!pygtk\_chart.label \textit{(module)}!pygtk\_chart.label.Label \textit{(class)}!pygtk\_chart.label.Label.get\_underline \textit{(method)}}

    \vspace{0.5ex}

\hspace{.8\funcindent}\begin{boxedminipage}{\funcwidth}

    \raggedright \textbf{get\_underline}(\textit{self})

    \vspace{-1.5ex}

    \rule{\textwidth}{0.5\fboxrule}
\setlength{\parskip}{2ex}
    Returns the current underline style. See \texttt{set\_underline} for 
    details.

\setlength{\parskip}{1ex}
      \textbf{Return Value}
    \vspace{-1ex}

      \begin{quote}
      an underline constant (see \texttt{set\_underline}).

      \end{quote}

    \end{boxedminipage}

    \label{pygtk_chart:label:Label:set_max_width}
    \index{pygtk\_chart \textit{(package)}!pygtk\_chart.label \textit{(module)}!pygtk\_chart.label.Label \textit{(class)}!pygtk\_chart.label.Label.set\_max\_width \textit{(method)}}

    \vspace{0.5ex}

\hspace{.8\funcindent}\begin{boxedminipage}{\funcwidth}

    \raggedright \textbf{set\_max\_width}(\textit{self}, \textit{width})

    \vspace{-1.5ex}

    \rule{\textwidth}{0.5\fboxrule}
\setlength{\parskip}{2ex}
    Set the maximum width of the label in pixels.

\setlength{\parskip}{1ex}
      \textbf{Parameters}
      \vspace{-1ex}

      \begin{quote}
        \begin{Ventry}{xxxxx}

          \item[width]

          the maximum width

            {\it (type=integer.)}

        \end{Ventry}

      \end{quote}

    \end{boxedminipage}

    \label{pygtk_chart:label:Label:get_max_width}
    \index{pygtk\_chart \textit{(package)}!pygtk\_chart.label \textit{(module)}!pygtk\_chart.label.Label \textit{(class)}!pygtk\_chart.label.Label.get\_max\_width \textit{(method)}}

    \vspace{0.5ex}

\hspace{.8\funcindent}\begin{boxedminipage}{\funcwidth}

    \raggedright \textbf{get\_max\_width}(\textit{self})

    \vspace{-1.5ex}

    \rule{\textwidth}{0.5\fboxrule}
\setlength{\parskip}{2ex}
    Returns the maximum width of the label.

\setlength{\parskip}{1ex}
      \textbf{Return Value}
    \vspace{-1ex}

      \begin{quote}
      integer.

      \end{quote}

    \end{boxedminipage}

    \label{pygtk_chart:label:Label:set_rotation}
    \index{pygtk\_chart \textit{(package)}!pygtk\_chart.label \textit{(module)}!pygtk\_chart.label.Label \textit{(class)}!pygtk\_chart.label.Label.set\_rotation \textit{(method)}}

    \vspace{0.5ex}

\hspace{.8\funcindent}\begin{boxedminipage}{\funcwidth}

    \raggedright \textbf{set\_rotation}(\textit{self}, \textit{angle})

    \vspace{-1.5ex}

    \rule{\textwidth}{0.5\fboxrule}
\setlength{\parskip}{2ex}
    Use this method to set the rotation of the label in degrees.

\setlength{\parskip}{1ex}
      \textbf{Parameters}
      \vspace{-1ex}

      \begin{quote}
        \begin{Ventry}{xxxxx}

          \item[angle]

          the rotation angle

            {\it (type=integer in [0, 360].)}

        \end{Ventry}

      \end{quote}

    \end{boxedminipage}

    \label{pygtk_chart:label:Label:get_rotation}
    \index{pygtk\_chart \textit{(package)}!pygtk\_chart.label \textit{(module)}!pygtk\_chart.label.Label \textit{(class)}!pygtk\_chart.label.Label.get\_rotation \textit{(method)}}

    \vspace{0.5ex}

\hspace{.8\funcindent}\begin{boxedminipage}{\funcwidth}

    \raggedright \textbf{get\_rotation}(\textit{self})

    \vspace{-1.5ex}

    \rule{\textwidth}{0.5\fboxrule}
\setlength{\parskip}{2ex}
    Returns the current rotation angle.

\setlength{\parskip}{1ex}
      \textbf{Return Value}
    \vspace{-1ex}

      \begin{quote}
      integer in [0, 360].

      \end{quote}

    \end{boxedminipage}

    \label{pygtk_chart:label:Label:set_size}
    \index{pygtk\_chart \textit{(package)}!pygtk\_chart.label \textit{(module)}!pygtk\_chart.label.Label \textit{(class)}!pygtk\_chart.label.Label.set\_size \textit{(method)}}

    \vspace{0.5ex}

\hspace{.8\funcindent}\begin{boxedminipage}{\funcwidth}

    \raggedright \textbf{set\_size}(\textit{self}, \textit{size})

    \vspace{-1.5ex}

    \rule{\textwidth}{0.5\fboxrule}
\setlength{\parskip}{2ex}
    Set the size of the text in pixels.

\setlength{\parskip}{1ex}
      \textbf{Parameters}
      \vspace{-1ex}

      \begin{quote}
        \begin{Ventry}{xxxx}

          \item[size]

          size of the text

            {\it (type=integer.)}

        \end{Ventry}

      \end{quote}

    \end{boxedminipage}

    \label{pygtk_chart:label:Label:get_size}
    \index{pygtk\_chart \textit{(package)}!pygtk\_chart.label \textit{(module)}!pygtk\_chart.label.Label \textit{(class)}!pygtk\_chart.label.Label.get\_size \textit{(method)}}

    \vspace{0.5ex}

\hspace{.8\funcindent}\begin{boxedminipage}{\funcwidth}

    \raggedright \textbf{get\_size}(\textit{self})

    \vspace{-1.5ex}

    \rule{\textwidth}{0.5\fboxrule}
\setlength{\parskip}{2ex}
    Returns the current size of the text in pixels.

\setlength{\parskip}{1ex}
      \textbf{Return Value}
    \vspace{-1ex}

      \begin{quote}
      integer.

      \end{quote}

    \end{boxedminipage}

    \label{pygtk_chart:label:Label:set_slant}
    \index{pygtk\_chart \textit{(package)}!pygtk\_chart.label \textit{(module)}!pygtk\_chart.label.Label \textit{(class)}!pygtk\_chart.label.Label.set\_slant \textit{(method)}}

    \vspace{0.5ex}

\hspace{.8\funcindent}\begin{boxedminipage}{\funcwidth}

    \raggedright \textbf{set\_slant}(\textit{self}, \textit{slant})

    \vspace{-1.5ex}

    \rule{\textwidth}{0.5\fboxrule}
\setlength{\parskip}{2ex}
    Set the font slant. slat has to be one of the following:

    \begin{itemize}
    \setlength{\parskip}{0.6ex}
      \item label.STYLE\_NORMAL

      \item label.STYLE\_OBLIQUE

      \item label.STYLE\_ITALIC

    \end{itemize}

\setlength{\parskip}{1ex}
      \textbf{Parameters}
      \vspace{-1ex}

      \begin{quote}
        \begin{Ventry}{xxxxx}

          \item[slant]

          the font slant style

            {\it (type=one of the constants above.)}

        \end{Ventry}

      \end{quote}

    \end{boxedminipage}

    \label{pygtk_chart:label:Label:get_slant}
    \index{pygtk\_chart \textit{(package)}!pygtk\_chart.label \textit{(module)}!pygtk\_chart.label.Label \textit{(class)}!pygtk\_chart.label.Label.get\_slant \textit{(method)}}

    \vspace{0.5ex}

\hspace{.8\funcindent}\begin{boxedminipage}{\funcwidth}

    \raggedright \textbf{get\_slant}(\textit{self})

    \vspace{-1.5ex}

    \rule{\textwidth}{0.5\fboxrule}
\setlength{\parskip}{2ex}
    Returns the current font slant style. See \texttt{set\_slant} for 
    details.

\setlength{\parskip}{1ex}
      \textbf{Return Value}
    \vspace{-1ex}

      \begin{quote}
      a slant style constant.

      \end{quote}

    \end{boxedminipage}

    \label{pygtk_chart:label:Label:set_weight}
    \index{pygtk\_chart \textit{(package)}!pygtk\_chart.label \textit{(module)}!pygtk\_chart.label.Label \textit{(class)}!pygtk\_chart.label.Label.set\_weight \textit{(method)}}

    \vspace{0.5ex}

\hspace{.8\funcindent}\begin{boxedminipage}{\funcwidth}

    \raggedright \textbf{set\_weight}(\textit{self}, \textit{weight})

    \vspace{-1.5ex}

    \rule{\textwidth}{0.5\fboxrule}
\setlength{\parskip}{2ex}
    Set the font weight. weight has to be one of the following:

    \begin{itemize}
    \setlength{\parskip}{0.6ex}
      \item label.WEIGHT\_ULTRALIGHT

      \item label.WEIGHT\_LIGHT

      \item label.WEIGHT\_NORMAL

      \item label.WEIGHT\_BOLD

      \item label.WEIGHT\_ULTRABOLD

      \item label.WEIGHT\_HEAVY

    \end{itemize}

\setlength{\parskip}{1ex}
      \textbf{Parameters}
      \vspace{-1ex}

      \begin{quote}
        \begin{Ventry}{xxxxxx}

          \item[weight]

          the font weight

            {\it (type=one of the constants above.)}

        \end{Ventry}

      \end{quote}

    \end{boxedminipage}

    \label{pygtk_chart:label:Label:get_weight}
    \index{pygtk\_chart \textit{(package)}!pygtk\_chart.label \textit{(module)}!pygtk\_chart.label.Label \textit{(class)}!pygtk\_chart.label.Label.get\_weight \textit{(method)}}

    \vspace{0.5ex}

\hspace{.8\funcindent}\begin{boxedminipage}{\funcwidth}

    \raggedright \textbf{get\_weight}(\textit{self})

    \vspace{-1.5ex}

    \rule{\textwidth}{0.5\fboxrule}
\setlength{\parskip}{2ex}
    Returns the current font weight. See \texttt{set\_weight} for details.

\setlength{\parskip}{1ex}
      \textbf{Return Value}
    \vspace{-1ex}

      \begin{quote}
      a font weight constant.

      \end{quote}

    \end{boxedminipage}

    \label{pygtk_chart:label:Label:set_fixed}
    \index{pygtk\_chart \textit{(package)}!pygtk\_chart.label \textit{(module)}!pygtk\_chart.label.Label \textit{(class)}!pygtk\_chart.label.Label.set\_fixed \textit{(method)}}

    \vspace{0.5ex}

\hspace{.8\funcindent}\begin{boxedminipage}{\funcwidth}

    \raggedright \textbf{set\_fixed}(\textit{self}, \textit{fixed})

    \vspace{-1.5ex}

    \rule{\textwidth}{0.5\fboxrule}
\setlength{\parskip}{2ex}
    Set whether the position of the label should be forced (fixed=True) or 
    if it should be positioned avoiding intersection with other labels.

\setlength{\parskip}{1ex}
      \textbf{Parameters}
      \vspace{-1ex}

      \begin{quote}
        \begin{Ventry}{xxxxx}

          \item[fixed]

            {\it (type=boolean.)}

        \end{Ventry}

      \end{quote}

    \end{boxedminipage}

    \label{pygtk_chart:label:Label:get_fixed}
    \index{pygtk\_chart \textit{(package)}!pygtk\_chart.label \textit{(module)}!pygtk\_chart.label.Label \textit{(class)}!pygtk\_chart.label.Label.get\_fixed \textit{(method)}}

    \vspace{0.5ex}

\hspace{.8\funcindent}\begin{boxedminipage}{\funcwidth}

    \raggedright \textbf{get\_fixed}(\textit{self})

    \vspace{-1.5ex}

    \rule{\textwidth}{0.5\fboxrule}
\setlength{\parskip}{2ex}
    Returns True if the label's position is forced.

\setlength{\parskip}{1ex}
      \textbf{Return Value}
    \vspace{-1ex}

      \begin{quote}
      boolean

      \end{quote}

    \end{boxedminipage}

    \label{pygtk_chart:label:Label:set_wrap}
    \index{pygtk\_chart \textit{(package)}!pygtk\_chart.label \textit{(module)}!pygtk\_chart.label.Label \textit{(class)}!pygtk\_chart.label.Label.set\_wrap \textit{(method)}}

    \vspace{0.5ex}

\hspace{.8\funcindent}\begin{boxedminipage}{\funcwidth}

    \raggedright \textbf{set\_wrap}(\textit{self}, \textit{wrap})

    \vspace{-1.5ex}

    \rule{\textwidth}{0.5\fboxrule}
\setlength{\parskip}{2ex}
    Set whether too long text should be wrapped.

\setlength{\parskip}{1ex}
      \textbf{Parameters}
      \vspace{-1ex}

      \begin{quote}
        \begin{Ventry}{xxxx}

          \item[wrap]

            {\it (type=boolean.)}

        \end{Ventry}

      \end{quote}

    \end{boxedminipage}

    \label{pygtk_chart:label:Label:get_wrap}
    \index{pygtk\_chart \textit{(package)}!pygtk\_chart.label \textit{(module)}!pygtk\_chart.label.Label \textit{(class)}!pygtk\_chart.label.Label.get\_wrap \textit{(method)}}

    \vspace{0.5ex}

\hspace{.8\funcindent}\begin{boxedminipage}{\funcwidth}

    \raggedright \textbf{get\_wrap}(\textit{self})

    \vspace{-1.5ex}

    \rule{\textwidth}{0.5\fboxrule}
\setlength{\parskip}{2ex}
    Returns True if too long text should be wrapped.

\setlength{\parskip}{1ex}
      \textbf{Return Value}
    \vspace{-1ex}

      \begin{quote}
      boolean.

      \end{quote}

    \end{boxedminipage}

    \label{pygtk_chart:label:Label:get_real_dimensions}
    \index{pygtk\_chart \textit{(package)}!pygtk\_chart.label \textit{(module)}!pygtk\_chart.label.Label \textit{(class)}!pygtk\_chart.label.Label.get\_real\_dimensions \textit{(method)}}

    \vspace{0.5ex}

\hspace{.8\funcindent}\begin{boxedminipage}{\funcwidth}

    \raggedright \textbf{get\_real\_dimensions}(\textit{self})

    \vspace{-1.5ex}

    \rule{\textwidth}{0.5\fboxrule}
\setlength{\parskip}{2ex}
    This method returns a pair (width, height) with the dimensions the 
    label was drawn with. Call this method \textit{after} drawing the 
    label.

\setlength{\parskip}{1ex}
      \textbf{Return Value}
    \vspace{-1ex}

      \begin{quote}
      a (width, height) pair.

      \end{quote}

    \end{boxedminipage}

    \label{pygtk_chart:label:Label:get_real_position}
    \index{pygtk\_chart \textit{(package)}!pygtk\_chart.label \textit{(module)}!pygtk\_chart.label.Label \textit{(class)}!pygtk\_chart.label.Label.get\_real\_position \textit{(method)}}

    \vspace{0.5ex}

\hspace{.8\funcindent}\begin{boxedminipage}{\funcwidth}

    \raggedright \textbf{get\_real\_position}(\textit{self})

    \vspace{-1.5ex}

    \rule{\textwidth}{0.5\fboxrule}
\setlength{\parskip}{2ex}
    Returns the position of the label where it was really drawn.

\setlength{\parskip}{1ex}
      \textbf{Return Value}
    \vspace{-1ex}

      \begin{quote}
      a (x, y) pair.

      \end{quote}

    \end{boxedminipage}

    \label{pygtk_chart:label:Label:get_allocation}
    \index{pygtk\_chart \textit{(package)}!pygtk\_chart.label \textit{(module)}!pygtk\_chart.label.Label \textit{(class)}!pygtk\_chart.label.Label.get\_allocation \textit{(method)}}

    \vspace{0.5ex}

\hspace{.8\funcindent}\begin{boxedminipage}{\funcwidth}

    \raggedright \textbf{get\_allocation}(\textit{self})

    \vspace{-1.5ex}

    \rule{\textwidth}{0.5\fboxrule}
\setlength{\parskip}{2ex}
    Returns an allocation rectangle.

\setlength{\parskip}{1ex}
      \textbf{Return Value}
    \vspace{-1ex}

      \begin{quote}
      gtk.gdk.Rectangle.

      \end{quote}

    \end{boxedminipage}

    \label{pygtk_chart:label:Label:get_line_count}
    \index{pygtk\_chart \textit{(package)}!pygtk\_chart.label \textit{(module)}!pygtk\_chart.label.Label \textit{(class)}!pygtk\_chart.label.Label.get\_line\_count \textit{(method)}}

    \vspace{0.5ex}

\hspace{.8\funcindent}\begin{boxedminipage}{\funcwidth}

    \raggedright \textbf{get\_line\_count}(\textit{self})

    \vspace{-1.5ex}

    \rule{\textwidth}{0.5\fboxrule}
\setlength{\parskip}{2ex}
    Returns the number of lines.

\setlength{\parskip}{1ex}
      \textbf{Return Value}
    \vspace{-1ex}

      \begin{quote}
      int.

      \end{quote}

    \end{boxedminipage}


\large{\textbf{\textit{Inherited from pygtk\_chart.chart\_object.ChartObject\textit{(Section \ref{pygtk_chart:chart_object:ChartObject})}}}}

\begin{quote}
draw(), get\_antialias(), get\_visible(), set\_antialias(), set\_visible()
\end{quote}

\large{\textbf{\textit{Inherited from ??.GObject}}}

\begin{quote}
\_\_cmp\_\_(), \_\_copy\_\_(), \_\_deepcopy\_\_(), \_\_delattr\_\_(), \_\_gdoc\_\_(), \_\_gobject\_init\_\_(), \_\_hash\_\_(), \_\_new\_\_(), \_\_repr\_\_(), \_\_setattr\_\_(), chain(), connect(), connect\_after(), connect\_object(), connect\_object\_after(), disconnect(), disconnect\_by\_func(), emit(), emit\_stop\_by\_name(), freeze\_notify(), get\_data(), get\_properties(), get\_property(), handler\_block(), handler\_block\_by\_func(), handler\_disconnect(), handler\_is\_connected(), handler\_unblock(), handler\_unblock\_by\_func(), notify(), props(), set\_data(), set\_properties(), set\_property(), stop\_emission(), thaw\_notify(), weak\_ref()
\end{quote}

\large{\textbf{\textit{Inherited from object}}}

\begin{quote}
\_\_getattribute\_\_(), \_\_reduce\_\_(), \_\_reduce\_ex\_\_(), \_\_str\_\_()
\end{quote}

%%%%%%%%%%%%%%%%%%%%%%%%%%%%%%%%%%%%%%%%%%%%%%%%%%%%%%%%%%%%%%%%%%%%%%%%%%%
%%                              Properties                               %%
%%%%%%%%%%%%%%%%%%%%%%%%%%%%%%%%%%%%%%%%%%%%%%%%%%%%%%%%%%%%%%%%%%%%%%%%%%%

  \subsubsection{Properties}

    \vspace{-1cm}
\hspace{\varindent}\begin{longtable}{|p{\varnamewidth}|p{\vardescrwidth}|l}
\cline{1-2}
\cline{1-2} \centering \textbf{Name} & \centering \textbf{Description}& \\
\cline{1-2}
\endhead\cline{1-2}\multicolumn{3}{r}{\small\textit{continued on next page}}\\\endfoot\cline{1-2}
\endlastfoot\multicolumn{2}{|l|}{\textit{Inherited from ??.GObject}}\\
\multicolumn{2}{|p{\varwidth}|}{\raggedright \_\_grefcount\_\_}\\
\cline{1-2}
\multicolumn{2}{|l|}{\textit{Inherited from object}}\\
\multicolumn{2}{|p{\varwidth}|}{\raggedright \_\_class\_\_}\\
\cline{1-2}
\end{longtable}


%%%%%%%%%%%%%%%%%%%%%%%%%%%%%%%%%%%%%%%%%%%%%%%%%%%%%%%%%%%%%%%%%%%%%%%%%%%
%%                            Class Variables                            %%
%%%%%%%%%%%%%%%%%%%%%%%%%%%%%%%%%%%%%%%%%%%%%%%%%%%%%%%%%%%%%%%%%%%%%%%%%%%

  \subsubsection{Class Variables}

    \vspace{-1cm}
\hspace{\varindent}\begin{longtable}{|p{\varnamewidth}|p{\vardescrwidth}|l}
\cline{1-2}
\cline{1-2} \centering \textbf{Name} & \centering \textbf{Description}& \\
\cline{1-2}
\endhead\cline{1-2}\multicolumn{3}{r}{\small\textit{continued on next page}}\\\endfoot\cline{1-2}
\endlastfoot\raggedright \_\-\_\-g\-p\-r\-o\-p\-e\-r\-t\-i\-e\-s\-\_\-\_\- & \raggedright \textbf{Value:} 
{\tt \{"color":(gobject.TYPE\_PYOBJECT, "label color", "The colo\texttt{...}}&\\
\cline{1-2}
\raggedright \_\-\_\-g\-t\-y\-p\-e\-\_\-\_\- & \raggedright \textbf{Value:} 
{\tt {\textless}GType pygtk\_chart+label+Label (168663000){\textgreater}}&\\
\cline{1-2}
\multicolumn{2}{|l|}{\textit{Inherited from pygtk\_chart.chart\_object.ChartObject \textit{(Section \ref{pygtk_chart:chart_object:ChartObject})}}}\\
\multicolumn{2}{|p{\varwidth}|}{\raggedright \_\_gsignals\_\_}\\
\cline{1-2}
\end{longtable}

    \index{pygtk\_chart \textit{(package)}!pygtk\_chart.label \textit{(module)}!pygtk\_chart.label.Label \textit{(class)}|)}
    \index{pygtk\_chart \textit{(package)}!pygtk\_chart.label \textit{(module)}|)}
