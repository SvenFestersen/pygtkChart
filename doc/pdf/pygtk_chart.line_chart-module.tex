%
% API Documentation for pygtkChart
% Module pygtk_chart.line_chart
%
% Generated by epydoc 3.0.1
% [Mon Sep 28 11:06:52 2009]
%

%%%%%%%%%%%%%%%%%%%%%%%%%%%%%%%%%%%%%%%%%%%%%%%%%%%%%%%%%%%%%%%%%%%%%%%%%%%
%%                          Module Description                           %%
%%%%%%%%%%%%%%%%%%%%%%%%%%%%%%%%%%%%%%%%%%%%%%%%%%%%%%%%%%%%%%%%%%%%%%%%%%%

    \index{pygtk\_chart \textit{(package)}!pygtk\_chart.line\_chart \textit{(module)}|(}
\section{Module pygtk\_chart.line\_chart}

    \label{pygtk_chart:line_chart}
Contains the LineChart widget.

Author: Sven Festersen (sven@sven-festersen.de)


%%%%%%%%%%%%%%%%%%%%%%%%%%%%%%%%%%%%%%%%%%%%%%%%%%%%%%%%%%%%%%%%%%%%%%%%%%%
%%                               Functions                               %%
%%%%%%%%%%%%%%%%%%%%%%%%%%%%%%%%%%%%%%%%%%%%%%%%%%%%%%%%%%%%%%%%%%%%%%%%%%%

  \subsection{Functions}

    \label{pygtk_chart:line_chart:draw_point}
    \index{pygtk\_chart \textit{(package)}!pygtk\_chart.line\_chart \textit{(module)}!pygtk\_chart.line\_chart.draw\_point \textit{(function)}}

    \vspace{0.5ex}

\hspace{.8\funcindent}\begin{boxedminipage}{\funcwidth}

    \raggedright \textbf{draw\_point}(\textit{context}, \textit{x}, \textit{y}, \textit{radius}, \textit{style})

\setlength{\parskip}{2ex}
\setlength{\parskip}{1ex}
    \end{boxedminipage}

    \label{pygtk_chart:line_chart:draw_point_pixbuf}
    \index{pygtk\_chart \textit{(package)}!pygtk\_chart.line\_chart \textit{(module)}!pygtk\_chart.line\_chart.draw\_point\_pixbuf \textit{(function)}}

    \vspace{0.5ex}

\hspace{.8\funcindent}\begin{boxedminipage}{\funcwidth}

    \raggedright \textbf{draw\_point\_pixbuf}(\textit{context}, \textit{x}, \textit{y}, \textit{pixbuf})

\setlength{\parskip}{2ex}
\setlength{\parskip}{1ex}
    \end{boxedminipage}

    \label{pygtk_chart:line_chart:draw_errors}
    \index{pygtk\_chart \textit{(package)}!pygtk\_chart.line\_chart \textit{(module)}!pygtk\_chart.line\_chart.draw\_errors \textit{(function)}}

    \vspace{0.5ex}

\hspace{.8\funcindent}\begin{boxedminipage}{\funcwidth}

    \raggedright \textbf{draw\_errors}(\textit{context}, \textit{rect}, \textit{range\_calc}, \textit{x}, \textit{y}, \textit{errors}, \textit{draw\_x}, \textit{draw\_y}, \textit{xaxis}, \textit{yaxis}, \textit{size})

\setlength{\parskip}{2ex}
\setlength{\parskip}{1ex}
    \end{boxedminipage}

    \label{pygtk_chart:line_chart:separate_data_and_errors}
    \index{pygtk\_chart \textit{(package)}!pygtk\_chart.line\_chart \textit{(module)}!pygtk\_chart.line\_chart.separate\_data\_and\_errors \textit{(function)}}

    \vspace{0.5ex}

\hspace{.8\funcindent}\begin{boxedminipage}{\funcwidth}

    \raggedright \textbf{separate\_data\_and\_errors}(\textit{old\_data})

\setlength{\parskip}{2ex}
\setlength{\parskip}{1ex}
    \end{boxedminipage}

    \label{pygtk_chart:line_chart:graph_new_from_function}
    \index{pygtk\_chart \textit{(package)}!pygtk\_chart.line\_chart \textit{(module)}!pygtk\_chart.line\_chart.graph\_new\_from\_function \textit{(function)}}

    \vspace{0.5ex}

\hspace{.8\funcindent}\begin{boxedminipage}{\funcwidth}

    \raggedright \textbf{graph\_new\_from\_function}(\textit{func}, \textit{xmin}, \textit{xmax}, \textit{graph\_name}, \textit{samples}={\tt 100}, \textit{do\_optimize\_sampling}={\tt True})

    \vspace{-1.5ex}

    \rule{\textwidth}{0.5\fboxrule}
\setlength{\parskip}{2ex}
    Returns a line\_chart.Graph with data created from the function y = 
    func(x) with x in [xmin, xmax]. The id of the new graph is graph\_name.
    The parameter samples gives the number of points that should be 
    evaluated in [xmin, xmax] (default: 100). If do\_optimize\_sampling is 
    True (default) additional points will be evaluated to smoothen the 
    curve.

\setlength{\parskip}{1ex}
      \textbf{Parameters}
      \vspace{-1ex}

      \begin{quote}
        \begin{Ventry}{xxxxxxxxxxxxxxxxxxxx}

          \item[func]

          the function to evaluate

            {\it (type=a function)}

          \item[xmin]

          the minimum x value to evaluate

            {\it (type=float)}

          \item[xmax]

          the maximum x value to evaluate

            {\it (type=float)}

          \item[graph\_name]

          a unique name for the new graph

            {\it (type=string)}

          \item[samples]

          number of samples

            {\it (type=int)}

          \item[do\_optimize\_sampling]

          set whether to add additional points

            {\it (type=boolean)}

        \end{Ventry}

      \end{quote}

      \textbf{Return Value}
    \vspace{-1ex}

      \begin{quote}
      line\_chart.Graph

      \end{quote}

    \end{boxedminipage}

    \label{pygtk_chart:line_chart:optimize_sampling}
    \index{pygtk\_chart \textit{(package)}!pygtk\_chart.line\_chart \textit{(module)}!pygtk\_chart.line\_chart.optimize\_sampling \textit{(function)}}

    \vspace{0.5ex}

\hspace{.8\funcindent}\begin{boxedminipage}{\funcwidth}

    \raggedright \textbf{optimize\_sampling}(\textit{func}, \textit{data})

\setlength{\parskip}{2ex}
\setlength{\parskip}{1ex}
    \end{boxedminipage}

    \label{pygtk_chart:line_chart:graph_new_from_file}
    \index{pygtk\_chart \textit{(package)}!pygtk\_chart.line\_chart \textit{(module)}!pygtk\_chart.line\_chart.graph\_new\_from\_file \textit{(function)}}

    \vspace{0.5ex}

\hspace{.8\funcindent}\begin{boxedminipage}{\funcwidth}

    \raggedright \textbf{graph\_new\_from\_file}(\textit{filename}, \textit{graph\_name}, \textit{x\_col}={\tt 0}, \textit{y\_col}={\tt 1}, \textit{xerror\_col}={\tt -1}, \textit{yerror\_col}={\tt -1})

    \vspace{-1.5ex}

    \rule{\textwidth}{0.5\fboxrule}
\setlength{\parskip}{2ex}
    Returns a line\_chart.Graph with point taken from data file filename. 
    The id of the new graph is graph\_name.

    Data file format: The columns in the file have to be separated by tabs 
    or one or more spaces. Everything after '\#' is ignored (comment).

    Use the parameters x\_col and y\_col to control which columns to use 
    for plotting. By default, the first column (x\_col=0) is used for x 
    values, the second (y\_col=1) is used for y values.

    The parameters xerror\_col and yerror\_col should point to the column 
    in which the x/y error values are. If you do not want to provide x or y
    error data, omit the paramter or set it to -1 (default).

\setlength{\parskip}{1ex}
      \textbf{Parameters}
      \vspace{-1ex}

      \begin{quote}
        \begin{Ventry}{xxxxxxxxxx}

          \item[filename]

          path to the data file

            {\it (type=string)}

          \item[graph\_name]

          a unique name for the graph

            {\it (type=string)}

          \item[x\_col]

          the number of the column to use for x values

            {\it (type=int)}

          \item[y\_col]

          the number of the column to use for y values

            {\it (type=int)}

          \item[xerror\_col]

          index of the column for x error values

            {\it (type=int)}

          \item[yerror\_col]

          index of the column for y error values

            {\it (type=int)}

        \end{Ventry}

      \end{quote}

      \textbf{Return Value}
    \vspace{-1ex}

      \begin{quote}
      line\_chart.Graph

      \end{quote}

    \end{boxedminipage}


%%%%%%%%%%%%%%%%%%%%%%%%%%%%%%%%%%%%%%%%%%%%%%%%%%%%%%%%%%%%%%%%%%%%%%%%%%%
%%                               Variables                               %%
%%%%%%%%%%%%%%%%%%%%%%%%%%%%%%%%%%%%%%%%%%%%%%%%%%%%%%%%%%%%%%%%%%%%%%%%%%%

  \subsection{Variables}

    \vspace{-1cm}
\hspace{\varindent}\begin{longtable}{|p{\varnamewidth}|p{\vardescrwidth}|l}
\cline{1-2}
\cline{1-2} \centering \textbf{Name} & \centering \textbf{Description}& \\
\cline{1-2}
\endhead\cline{1-2}\multicolumn{3}{r}{\small\textit{continued on next page}}\\\endfoot\cline{1-2}
\endlastfoot\raggedright R\-A\-N\-G\-E\-\_\-A\-U\-T\-O\- & \raggedright \textbf{Value:} 
{\tt 0}&\\
\cline{1-2}
\raggedright G\-R\-A\-P\-H\-\_\-P\-A\-D\-D\-I\-N\-G\- & \raggedright \textbf{Value:} 
{\tt 0.0666666666667}&\\
\cline{1-2}
\raggedright G\-R\-A\-P\-H\-\_\-P\-O\-I\-N\-T\-S\- & \raggedright \textbf{Value:} 
{\tt 1}&\\
\cline{1-2}
\raggedright G\-R\-A\-P\-H\-\_\-L\-I\-N\-E\-S\- & \raggedright \textbf{Value:} 
{\tt 2}&\\
\cline{1-2}
\raggedright G\-R\-A\-P\-H\-\_\-B\-O\-T\-H\- & \raggedright \textbf{Value:} 
{\tt 3}&\\
\cline{1-2}
\raggedright C\-O\-L\-O\-R\-\_\-A\-U\-T\-O\- & \raggedright \textbf{Value:} 
{\tt 4}&\\
\cline{1-2}
\raggedright P\-O\-S\-I\-T\-I\-O\-N\-\_\-A\-U\-T\-O\- & \raggedright \textbf{Value:} 
{\tt 5}&\\
\cline{1-2}
\raggedright P\-O\-S\-I\-T\-I\-O\-N\-\_\-L\-E\-F\-T\- & \raggedright \textbf{Value:} 
{\tt 6}&\\
\cline{1-2}
\raggedright P\-O\-S\-I\-T\-I\-O\-N\-\_\-R\-I\-G\-H\-T\- & \raggedright \textbf{Value:} 
{\tt 7}&\\
\cline{1-2}
\raggedright P\-O\-S\-I\-T\-I\-O\-N\-\_\-B\-O\-T\-T\-O\-M\- & \raggedright \textbf{Value:} 
{\tt 6}&\\
\cline{1-2}
\raggedright P\-O\-S\-I\-T\-I\-O\-N\-\_\-T\-O\-P\- & \raggedright \textbf{Value:} 
{\tt 7}&\\
\cline{1-2}
\raggedright P\-O\-S\-I\-T\-I\-O\-N\-\_\-T\-O\-P\-\_\-R\-I\-G\-H\-T\- & \raggedright \textbf{Value:} 
{\tt 8}&\\
\cline{1-2}
\raggedright P\-O\-S\-I\-T\-I\-O\-N\-\_\-B\-O\-T\-T\-O\-M\-\_\-R\-I\-G\-H\-T\- & \raggedright \textbf{Value:} 
{\tt 9}&\\
\cline{1-2}
\raggedright P\-O\-S\-I\-T\-I\-O\-N\-\_\-B\-O\-T\-T\-O\-M\-\_\-L\-E\-F\-T\- & \raggedright \textbf{Value:} 
{\tt 10}&\\
\cline{1-2}
\raggedright P\-O\-S\-I\-T\-I\-O\-N\-\_\-T\-O\-P\-\_\-L\-E\-F\-T\- & \raggedright \textbf{Value:} 
{\tt 11}&\\
\cline{1-2}
\end{longtable}


%%%%%%%%%%%%%%%%%%%%%%%%%%%%%%%%%%%%%%%%%%%%%%%%%%%%%%%%%%%%%%%%%%%%%%%%%%%
%%                           Class Description                           %%
%%%%%%%%%%%%%%%%%%%%%%%%%%%%%%%%%%%%%%%%%%%%%%%%%%%%%%%%%%%%%%%%%%%%%%%%%%%

    \index{pygtk\_chart \textit{(package)}!pygtk\_chart.line\_chart \textit{(module)}!pygtk\_chart.line\_chart.RangeCalculator \textit{(class)}|(}
\subsection{Class RangeCalculator}

    \label{pygtk_chart:line_chart:RangeCalculator}
This helper class calculates ranges. It is used by the LineChart widget 
internally, there is no need to create an instance yourself.


%%%%%%%%%%%%%%%%%%%%%%%%%%%%%%%%%%%%%%%%%%%%%%%%%%%%%%%%%%%%%%%%%%%%%%%%%%%
%%                                Methods                                %%
%%%%%%%%%%%%%%%%%%%%%%%%%%%%%%%%%%%%%%%%%%%%%%%%%%%%%%%%%%%%%%%%%%%%%%%%%%%

  \subsubsection{Methods}

    \label{pygtk_chart:line_chart:RangeCalculator:__init__}
    \index{pygtk\_chart \textit{(package)}!pygtk\_chart.line\_chart \textit{(module)}!pygtk\_chart.line\_chart.RangeCalculator \textit{(class)}!pygtk\_chart.line\_chart.RangeCalculator.\_\_init\_\_ \textit{(method)}}

    \vspace{0.5ex}

\hspace{.8\funcindent}\begin{boxedminipage}{\funcwidth}

    \raggedright \textbf{\_\_init\_\_}(\textit{self})

\setlength{\parskip}{2ex}
\setlength{\parskip}{1ex}
    \end{boxedminipage}

    \label{pygtk_chart:line_chart:RangeCalculator:add_graph}
    \index{pygtk\_chart \textit{(package)}!pygtk\_chart.line\_chart \textit{(module)}!pygtk\_chart.line\_chart.RangeCalculator \textit{(class)}!pygtk\_chart.line\_chart.RangeCalculator.add\_graph \textit{(method)}}

    \vspace{0.5ex}

\hspace{.8\funcindent}\begin{boxedminipage}{\funcwidth}

    \raggedright \textbf{add\_graph}(\textit{self}, \textit{graph})

\setlength{\parskip}{2ex}
\setlength{\parskip}{1ex}
    \end{boxedminipage}

    \label{pygtk_chart:line_chart:RangeCalculator:get_ranges}
    \index{pygtk\_chart \textit{(package)}!pygtk\_chart.line\_chart \textit{(module)}!pygtk\_chart.line\_chart.RangeCalculator \textit{(class)}!pygtk\_chart.line\_chart.RangeCalculator.get\_ranges \textit{(method)}}

    \vspace{0.5ex}

\hspace{.8\funcindent}\begin{boxedminipage}{\funcwidth}

    \raggedright \textbf{get\_ranges}(\textit{self}, \textit{xaxis}, \textit{yaxis})

\setlength{\parskip}{2ex}
\setlength{\parskip}{1ex}
    \end{boxedminipage}

    \label{pygtk_chart:line_chart:RangeCalculator:set_xrange}
    \index{pygtk\_chart \textit{(package)}!pygtk\_chart.line\_chart \textit{(module)}!pygtk\_chart.line\_chart.RangeCalculator \textit{(class)}!pygtk\_chart.line\_chart.RangeCalculator.set\_xrange \textit{(method)}}

    \vspace{0.5ex}

\hspace{.8\funcindent}\begin{boxedminipage}{\funcwidth}

    \raggedright \textbf{set\_xrange}(\textit{self}, \textit{xrange})

\setlength{\parskip}{2ex}
\setlength{\parskip}{1ex}
    \end{boxedminipage}

    \label{pygtk_chart:line_chart:RangeCalculator:set_yrange}
    \index{pygtk\_chart \textit{(package)}!pygtk\_chart.line\_chart \textit{(module)}!pygtk\_chart.line\_chart.RangeCalculator \textit{(class)}!pygtk\_chart.line\_chart.RangeCalculator.set\_yrange \textit{(method)}}

    \vspace{0.5ex}

\hspace{.8\funcindent}\begin{boxedminipage}{\funcwidth}

    \raggedright \textbf{set\_yrange}(\textit{self}, \textit{yrange})

\setlength{\parskip}{2ex}
\setlength{\parskip}{1ex}
    \end{boxedminipage}

    \label{pygtk_chart:line_chart:RangeCalculator:get_absolute_zero}
    \index{pygtk\_chart \textit{(package)}!pygtk\_chart.line\_chart \textit{(module)}!pygtk\_chart.line\_chart.RangeCalculator \textit{(class)}!pygtk\_chart.line\_chart.RangeCalculator.get\_absolute\_zero \textit{(method)}}

    \vspace{0.5ex}

\hspace{.8\funcindent}\begin{boxedminipage}{\funcwidth}

    \raggedright \textbf{get\_absolute\_zero}(\textit{self}, \textit{rect}, \textit{xaxis}, \textit{yaxis})

\setlength{\parskip}{2ex}
\setlength{\parskip}{1ex}
    \end{boxedminipage}

    \label{pygtk_chart:line_chart:RangeCalculator:get_absolute_point}
    \index{pygtk\_chart \textit{(package)}!pygtk\_chart.line\_chart \textit{(module)}!pygtk\_chart.line\_chart.RangeCalculator \textit{(class)}!pygtk\_chart.line\_chart.RangeCalculator.get\_absolute\_point \textit{(method)}}

    \vspace{0.5ex}

\hspace{.8\funcindent}\begin{boxedminipage}{\funcwidth}

    \raggedright \textbf{get\_absolute\_point}(\textit{self}, \textit{rect}, \textit{x}, \textit{y}, \textit{xaxis}, \textit{yaxis})

\setlength{\parskip}{2ex}
\setlength{\parskip}{1ex}
    \end{boxedminipage}

    \label{pygtk_chart:line_chart:RangeCalculator:prepare_tics}
    \index{pygtk\_chart \textit{(package)}!pygtk\_chart.line\_chart \textit{(module)}!pygtk\_chart.line\_chart.RangeCalculator \textit{(class)}!pygtk\_chart.line\_chart.RangeCalculator.prepare\_tics \textit{(method)}}

    \vspace{0.5ex}

\hspace{.8\funcindent}\begin{boxedminipage}{\funcwidth}

    \raggedright \textbf{prepare\_tics}(\textit{self}, \textit{rect}, \textit{xaxis}, \textit{yaxis})

\setlength{\parskip}{2ex}
\setlength{\parskip}{1ex}
    \end{boxedminipage}

    \label{pygtk_chart:line_chart:RangeCalculator:get_xtics}
    \index{pygtk\_chart \textit{(package)}!pygtk\_chart.line\_chart \textit{(module)}!pygtk\_chart.line\_chart.RangeCalculator \textit{(class)}!pygtk\_chart.line\_chart.RangeCalculator.get\_xtics \textit{(method)}}

    \vspace{0.5ex}

\hspace{.8\funcindent}\begin{boxedminipage}{\funcwidth}

    \raggedright \textbf{get\_xtics}(\textit{self}, \textit{rect})

\setlength{\parskip}{2ex}
\setlength{\parskip}{1ex}
    \end{boxedminipage}

    \label{pygtk_chart:line_chart:RangeCalculator:get_ytics}
    \index{pygtk\_chart \textit{(package)}!pygtk\_chart.line\_chart \textit{(module)}!pygtk\_chart.line\_chart.RangeCalculator \textit{(class)}!pygtk\_chart.line\_chart.RangeCalculator.get\_ytics \textit{(method)}}

    \vspace{0.5ex}

\hspace{.8\funcindent}\begin{boxedminipage}{\funcwidth}

    \raggedright \textbf{get\_ytics}(\textit{self}, \textit{rect})

\setlength{\parskip}{2ex}
\setlength{\parskip}{1ex}
    \end{boxedminipage}

    \index{pygtk\_chart \textit{(package)}!pygtk\_chart.line\_chart \textit{(module)}!pygtk\_chart.line\_chart.RangeCalculator \textit{(class)}|)}

%%%%%%%%%%%%%%%%%%%%%%%%%%%%%%%%%%%%%%%%%%%%%%%%%%%%%%%%%%%%%%%%%%%%%%%%%%%
%%                           Class Description                           %%
%%%%%%%%%%%%%%%%%%%%%%%%%%%%%%%%%%%%%%%%%%%%%%%%%%%%%%%%%%%%%%%%%%%%%%%%%%%

    \index{pygtk\_chart \textit{(package)}!pygtk\_chart.line\_chart \textit{(module)}!pygtk\_chart.line\_chart.LineChart \textit{(class)}|(}
\subsection{Class LineChart}

    \label{pygtk_chart:line_chart:LineChart}
\begin{tabular}{cccccccccccccccc}
% Line for object, linespec=[False, False, False, False, False, False]
\multicolumn{2}{r}{\settowidth{\BCL}{object}\multirow{2}{\BCL}{object}}
&&
&&
&&
&&
&&
&&
  \\\cline{3-3}
  &&\multicolumn{1}{c|}{}
&&
&&
&&
&&
&&
&&
  \\
% Line for ??.GObject, linespec=[False, False, False, False, False]
\multicolumn{4}{r}{\settowidth{\BCL}{??.GObject}\multirow{2}{\BCL}{??.GObject}}
&&
&&
&&
&&
&&
  \\\cline{5-5}
  &&&&\multicolumn{1}{c|}{}
&&
&&
&&
&&
&&
  \\
% Line for gtk.Object, linespec=[False, False, False, False]
\multicolumn{6}{r}{\settowidth{\BCL}{gtk.Object}\multirow{2}{\BCL}{gtk.Object}}
&&
&&
&&
&&
  \\\cline{7-7}
  &&&&&&\multicolumn{1}{c|}{}
&&
&&
&&
&&
  \\
% Line for object, linespec=[False, False, True, False, False, False]
\multicolumn{2}{r}{\settowidth{\BCL}{object}\multirow{2}{\BCL}{object}}
&&
&&
&&\multicolumn{1}{|c}{}
&&
&&
&&
  \\\cline{3-3}
  &&\multicolumn{1}{c|}{}
&&
&&
&\multicolumn{1}{|c}{}&
&&
&&
&&
  \\
% Line for gobject.GInterface, linespec=[False, True, False, False, False]
\multicolumn{4}{r}{\settowidth{\BCL}{gobject.GInterface}\multirow{2}{\BCL}{gobject.GInterface}}
&&
&&\multicolumn{1}{|c}{}
&&
&&
&&
  \\\cline{5-5}
  &&&&\multicolumn{1}{c|}{}
&&
&\multicolumn{1}{|c}{}&
&&
&&
&&
  \\
% Line for atk.ImplementorIface, linespec=[True, False, False, False]
\multicolumn{6}{r}{\settowidth{\BCL}{atk.ImplementorIface}\multirow{2}{\BCL}{atk.ImplementorIface}}
&&\multicolumn{1}{|c}{}
&&
&&
&&
  \\\cline{7-7}
  &&&&&&\multicolumn{1}{c|}{}
&\multicolumn{1}{|c}{}&
&&
&&
&&
  \\
% Line for object, linespec=[False, False, True, False, False, False]
\multicolumn{2}{r}{\settowidth{\BCL}{object}\multirow{2}{\BCL}{object}}
&&
&&
&&\multicolumn{1}{|c}{}
&&
&&
&&
  \\\cline{3-3}
  &&\multicolumn{1}{c|}{}
&&
&&
&\multicolumn{1}{|c}{}&
&&
&&
&&
  \\
% Line for gobject.GInterface, linespec=[False, True, False, False, False]
\multicolumn{4}{r}{\settowidth{\BCL}{gobject.GInterface}\multirow{2}{\BCL}{gobject.GInterface}}
&&
&&\multicolumn{1}{|c}{}
&&
&&
&&
  \\\cline{5-5}
  &&&&\multicolumn{1}{c|}{}
&&
&\multicolumn{1}{|c}{}&
&&
&&
&&
  \\
% Line for gtk.Buildable, linespec=[True, False, False, False]
\multicolumn{6}{r}{\settowidth{\BCL}{gtk.Buildable}\multirow{2}{\BCL}{gtk.Buildable}}
&&\multicolumn{1}{|c}{}
&&
&&
&&
  \\\cline{7-7}
  &&&&&&\multicolumn{1}{c|}{}
&\multicolumn{1}{|c}{}&
&&
&&
&&
  \\
% Line for gtk.Widget, linespec=[False, False, False]
\multicolumn{8}{r}{\settowidth{\BCL}{gtk.Widget}\multirow{2}{\BCL}{gtk.Widget}}
&&
&&
&&
  \\\cline{9-9}
  &&&&&&&&\multicolumn{1}{c|}{}
&&
&&
&&
  \\
% Line for gtk.DrawingArea, linespec=[False, False]
\multicolumn{10}{r}{\settowidth{\BCL}{gtk.DrawingArea}\multirow{2}{\BCL}{gtk.DrawingArea}}
&&
&&
  \\\cline{11-11}
  &&&&&&&&&&\multicolumn{1}{c|}{}
&&
&&
  \\
% Line for pygtk\_chart.chart.Chart, linespec=[False]
\multicolumn{12}{r}{\settowidth{\BCL}{pygtk\_chart.chart.Chart}\multirow{2}{\BCL}{pygtk\_chart.chart.Chart}}
&&
  \\\cline{13-13}
  &&&&&&&&&&&&\multicolumn{1}{c|}{}
&&
  \\
&&&&&&&&&&&&\multicolumn{2}{l}{\textbf{pygtk\_chart.line\_chart.LineChart}}
\end{tabular}

A widget that shows a line chart. The following attributes can be accessed:

\begin{itemize}
\setlength{\parskip}{0.6ex}
  \item LineChart.background (inherited from chart.Chart)

  \item LineChart.title (inherited from chart.Chart)

  \item LineChart.graphs (a dict that holds the graphs identified by their 
    name)

  \item LineChart.grid

  \item LineChart.xaxis

  \item LineChart.yaxis

\end{itemize}

(section) Properties

  LineChart inherits properties from chart.Chart.

(section) Signals

  The LineChart class inherits signals from chart.Chart. Additional chart:

  \begin{itemize}
  \setlength{\parskip}{0.6ex}
    \item datapoint-clicked (emitted if a datapoint is clicked)

    \item datapoint-hovered (emitted if a datapoint is hovered with the mouse 
      pointer)

  \end{itemize}

  Callback signature for both signals: def callback(linechart, graph, (x, 
  y))


%%%%%%%%%%%%%%%%%%%%%%%%%%%%%%%%%%%%%%%%%%%%%%%%%%%%%%%%%%%%%%%%%%%%%%%%%%%
%%                                Methods                                %%
%%%%%%%%%%%%%%%%%%%%%%%%%%%%%%%%%%%%%%%%%%%%%%%%%%%%%%%%%%%%%%%%%%%%%%%%%%%

  \subsubsection{Methods}

    \vspace{0.5ex}

\hspace{.8\funcindent}\begin{boxedminipage}{\funcwidth}

    \raggedright \textbf{\_\_init\_\_}(\textit{self})

\setlength{\parskip}{2ex}
    x.\_\_init\_\_(...) initializes x; see x.\_\_class\_\_.\_\_doc\_\_ for 
    signature

\setlength{\parskip}{1ex}
      Overrides: object.\_\_init\_\_ 	extit{(inherited documentation)}

    \end{boxedminipage}

    \label{pygtk_chart:line_chart:LineChart:__iter__}
    \index{pygtk\_chart \textit{(package)}!pygtk\_chart.line\_chart \textit{(module)}!pygtk\_chart.line\_chart.LineChart \textit{(class)}!pygtk\_chart.line\_chart.LineChart.\_\_iter\_\_ \textit{(method)}}

    \vspace{0.5ex}

\hspace{.8\funcindent}\begin{boxedminipage}{\funcwidth}

    \raggedright \textbf{\_\_iter\_\_}(\textit{self})

\setlength{\parskip}{2ex}
\setlength{\parskip}{1ex}
    \end{boxedminipage}

    \vspace{0.5ex}

\hspace{.8\funcindent}\begin{boxedminipage}{\funcwidth}

    \raggedright \textbf{draw}(\textit{self}, \textit{context})

    \vspace{-1.5ex}

    \rule{\textwidth}{0.5\fboxrule}
\setlength{\parskip}{2ex}
    Draw the widget. This method is called automatically. Don't call it 
    yourself. If you want to force a redrawing of the widget, call the 
    queue\_draw() method.

\setlength{\parskip}{1ex}
      \textbf{Parameters}
      \vspace{-1ex}

      \begin{quote}
        \begin{Ventry}{xxxxxxx}

          \item[context]

          The context to draw on.

            {\it (type=cairo.Context)}

        \end{Ventry}

      \end{quote}

      Overrides: gtk.Widget.draw

    \end{boxedminipage}

    \label{pygtk_chart:line_chart:LineChart:add_graph}
    \index{pygtk\_chart \textit{(package)}!pygtk\_chart.line\_chart \textit{(module)}!pygtk\_chart.line\_chart.LineChart \textit{(class)}!pygtk\_chart.line\_chart.LineChart.add\_graph \textit{(method)}}

    \vspace{0.5ex}

\hspace{.8\funcindent}\begin{boxedminipage}{\funcwidth}

    \raggedright \textbf{add\_graph}(\textit{self}, \textit{graph})

    \vspace{-1.5ex}

    \rule{\textwidth}{0.5\fboxrule}
\setlength{\parskip}{2ex}
    Add a graph object to the plot.

\setlength{\parskip}{1ex}
      \textbf{Parameters}
      \vspace{-1ex}

      \begin{quote}
        \begin{Ventry}{xxxxx}

          \item[graph]

          The graph to add.

            {\it (type=line\_chart.Graph)}

        \end{Ventry}

      \end{quote}

    \end{boxedminipage}

    \label{pygtk_chart:line_chart:LineChart:remove_graph}
    \index{pygtk\_chart \textit{(package)}!pygtk\_chart.line\_chart \textit{(module)}!pygtk\_chart.line\_chart.LineChart \textit{(class)}!pygtk\_chart.line\_chart.LineChart.remove\_graph \textit{(method)}}

    \vspace{0.5ex}

\hspace{.8\funcindent}\begin{boxedminipage}{\funcwidth}

    \raggedright \textbf{remove\_graph}(\textit{self}, \textit{name})

    \vspace{-1.5ex}

    \rule{\textwidth}{0.5\fboxrule}
\setlength{\parskip}{2ex}
    Remove a graph from the plot.

\setlength{\parskip}{1ex}
      \textbf{Parameters}
      \vspace{-1ex}

      \begin{quote}
        \begin{Ventry}{xxxx}

          \item[name]

          The name of the graph to remove.

            {\it (type=string)}

        \end{Ventry}

      \end{quote}

    \end{boxedminipage}

    \label{pygtk_chart:line_chart:LineChart:set_xrange}
    \index{pygtk\_chart \textit{(package)}!pygtk\_chart.line\_chart \textit{(module)}!pygtk\_chart.line\_chart.LineChart \textit{(class)}!pygtk\_chart.line\_chart.LineChart.set\_xrange \textit{(method)}}

    \vspace{0.5ex}

\hspace{.8\funcindent}\begin{boxedminipage}{\funcwidth}

    \raggedright \textbf{set\_xrange}(\textit{self}, \textit{xrange})

    \vspace{-1.5ex}

    \rule{\textwidth}{0.5\fboxrule}
\setlength{\parskip}{2ex}
    Set the visible xrange. xrange has to be a pair: (xmin, xmax) or 
    RANGE\_AUTO. If you set it to RANGE\_AUTO, the visible range will be 
    calculated.

\setlength{\parskip}{1ex}
      \textbf{Parameters}
      \vspace{-1ex}

      \begin{quote}
        \begin{Ventry}{xxxxxx}

          \item[xrange]

          The new xrange.

            {\it (type=pair of numbers)}

        \end{Ventry}

      \end{quote}

    \end{boxedminipage}

    \label{pygtk_chart:line_chart:LineChart:get_xrange}
    \index{pygtk\_chart \textit{(package)}!pygtk\_chart.line\_chart \textit{(module)}!pygtk\_chart.line\_chart.LineChart \textit{(class)}!pygtk\_chart.line\_chart.LineChart.get\_xrange \textit{(method)}}

    \vspace{0.5ex}

\hspace{.8\funcindent}\begin{boxedminipage}{\funcwidth}

    \raggedright \textbf{get\_xrange}(\textit{self})

\setlength{\parskip}{2ex}
\setlength{\parskip}{1ex}
    \end{boxedminipage}

    \label{pygtk_chart:line_chart:LineChart:set_yrange}
    \index{pygtk\_chart \textit{(package)}!pygtk\_chart.line\_chart \textit{(module)}!pygtk\_chart.line\_chart.LineChart \textit{(class)}!pygtk\_chart.line\_chart.LineChart.set\_yrange \textit{(method)}}

    \vspace{0.5ex}

\hspace{.8\funcindent}\begin{boxedminipage}{\funcwidth}

    \raggedright \textbf{set\_yrange}(\textit{self}, \textit{yrange})

    \vspace{-1.5ex}

    \rule{\textwidth}{0.5\fboxrule}
\setlength{\parskip}{2ex}
    Set the visible yrange. yrange has to be a pair: (ymin, ymax) or 
    RANGE\_AUTO. If you set it to RANGE\_AUTO, the visible range will be 
    calculated.

\setlength{\parskip}{1ex}
      \textbf{Parameters}
      \vspace{-1ex}

      \begin{quote}
        \begin{Ventry}{xxxxxx}

          \item[yrange]

          The new yrange.

            {\it (type=pair of numbers)}

        \end{Ventry}

      \end{quote}

    \end{boxedminipage}

    \label{pygtk_chart:line_chart:LineChart:get_yrange}
    \index{pygtk\_chart \textit{(package)}!pygtk\_chart.line\_chart \textit{(module)}!pygtk\_chart.line\_chart.LineChart \textit{(class)}!pygtk\_chart.line\_chart.LineChart.get\_yrange \textit{(method)}}

    \vspace{0.5ex}

\hspace{.8\funcindent}\begin{boxedminipage}{\funcwidth}

    \raggedright \textbf{get\_yrange}(\textit{self})

\setlength{\parskip}{2ex}
\setlength{\parskip}{1ex}
    \end{boxedminipage}


\large{\textbf{\textit{Inherited from pygtk\_chart.chart.Chart\textit{(Section \ref{pygtk_chart:chart:Chart})}}}}

\begin{quote}
do\_get\_property(), do\_set\_property(), draw\_basics(), export\_png(), export\_svg(), get\_padding(), set\_padding()
\end{quote}

\large{\textbf{\textit{Inherited from gtk.DrawingArea}}}

\begin{quote}
size()
\end{quote}

\large{\textbf{\textit{Inherited from gtk.Widget}}}

\begin{quote}
activate(), add\_accelerator(), add\_events(), add\_mnemonic\_label(), can\_activate\_accel(), child\_focus(), child\_notify(), class\_path(), create\_pango\_context(), create\_pango\_layout(), destroy(), do\_button\_press\_event(), do\_button\_release\_event(), do\_can\_activate\_accel(), do\_client\_event(), do\_composited\_changed(), do\_configure\_event(), do\_delete\_event(), do\_destroy\_event(), do\_direction\_changed(), do\_drag\_begin(), do\_drag\_data\_delete(), do\_drag\_data\_get(), do\_drag\_data\_received(), do\_drag\_drop(), do\_drag\_end(), do\_drag\_leave(), do\_drag\_motion(), do\_enter\_notify\_event(), do\_event(), do\_expose\_event(), do\_focus(), do\_focus\_in\_event(), do\_focus\_out\_event(), do\_get\_accessible(), do\_grab\_broken\_event(), do\_grab\_focus(), do\_grab\_notify(), do\_hide(), do\_hide\_all(), do\_hierarchy\_changed(), do\_key\_press\_event(), do\_key\_release\_event(), do\_leave\_notify\_event(), do\_map(), do\_map\_event(), do\_mnemonic\_activate(), do\_motion\_notify\_event(), do\_no\_expose\_event(), do\_parent\_set(), do\_popup\_menu(), do\_property\_notify\_event(), do\_proximity\_in\_event(), do\_proximity\_out\_event(), do\_realize(), do\_screen\_changed(), do\_scroll\_event(), do\_selection\_clear\_event(), do\_selection\_get(), do\_selection\_notify\_event(), do\_selection\_received(), do\_selection\_request\_event(), do\_show(), do\_show\_all(), do\_show\_help(), do\_size\_allocate(), do\_size\_request(), do\_state\_changed(), do\_style\_set(), do\_unmap(), do\_unmap\_event(), do\_unrealize(), do\_visibility\_notify\_event(), do\_window\_state\_event(), drag\_begin(), drag\_check\_threshold(), drag\_dest\_add\_image\_targets(), drag\_dest\_add\_text\_targets(), drag\_dest\_add\_uri\_targets(), drag\_dest\_find\_target(), drag\_dest\_get\_target\_list(), drag\_dest\_get\_track\_motion(), drag\_dest\_set(), drag\_dest\_set\_proxy(), drag\_dest\_set\_target\_list(), drag\_dest\_set\_track\_motion(), drag\_dest\_unset(), drag\_get\_data(), drag\_highlight(), drag\_source\_add\_image\_targets(), drag\_source\_add\_text\_targets(), drag\_source\_add\_uri\_targets(), drag\_source\_get\_target\_list(), drag\_source\_set(), drag\_source\_set\_icon(), drag\_source\_set\_icon\_name(), drag\_source\_set\_icon\_pixbuf(), drag\_source\_set\_icon\_stock(), drag\_source\_set\_target\_list(), drag\_source\_unset(), drag\_unhighlight(), ensure\_style(), error\_bell(), event(), freeze\_child\_notify(), get\_accessible(), get\_action(), get\_activate\_signal(), get\_allocation(), get\_ancestor(), get\_child\_requisition(), get\_child\_visible(), get\_clipboard(), get\_colormap(), get\_composite\_name(), get\_direction(), get\_display(), get\_events(), get\_extension\_events(), get\_has\_tooltip(), get\_modifier\_style(), get\_name(), get\_no\_show\_all(), get\_pango\_context(), get\_parent(), get\_parent\_window(), get\_pointer(), get\_root\_window(), get\_screen(), get\_settings(), get\_size\_request(), get\_snapshot(), get\_style(), get\_tooltip\_markup(), get\_tooltip\_text(), get\_tooltip\_window(), get\_toplevel(), get\_visual(), get\_window(), grab\_add(), grab\_default(), grab\_focus(), grab\_remove(), has\_screen(), hide(), hide\_all(), hide\_on\_delete(), input\_shape\_combine\_mask(), intersect(), is\_ancestor(), is\_composited(), is\_focus(), keynav\_failed(), list\_mnemonic\_labels(), map(), menu\_get\_for\_attach\_widget(), mnemonic\_activate(), modify\_base(), modify\_bg(), modify\_cursor(), modify\_fg(), modify\_font(), modify\_style(), modify\_text(), path(), queue\_clear(), queue\_clear\_area(), queue\_draw(), queue\_draw\_area(), queue\_resize(), queue\_resize\_no\_redraw(), rc\_get\_style(), realize(), region\_intersect(), remove\_accelerator(), remove\_mnemonic\_label(), render\_icon(), reparent(), reset\_rc\_styles(), reset\_shapes(), selection\_add\_target(), selection\_add\_targets(), selection\_clear\_targets(), selection\_convert(), selection\_owner\_set(), selection\_remove\_all(), send\_expose(), set\_accel\_path(), set\_activate\_signal(), set\_app\_paintable(), set\_child\_visible(), set\_colormap(), set\_composite\_name(), set\_direction(), set\_double\_buffered(), set\_events(), set\_extension\_events(), set\_has\_tooltip(), set\_name(), set\_no\_show\_all(), set\_parent(), set\_parent\_window(), set\_redraw\_on\_allocate(), set\_scroll\_adjustments(), set\_sensitive(), set\_set\_scroll\_adjustments\_signal(), set\_size\_request(), set\_state(), set\_style(), set\_tooltip\_markup(), set\_tooltip\_text(), set\_tooltip\_window(), set\_uposition(), set\_usize(), shape\_combine\_mask(), show(), show\_all(), show\_now(), size\_allocate(), size\_request(), style\_get\_property(), thaw\_child\_notify(), translate\_coordinates(), trigger\_tooltip\_query(), unmap(), unparent(), unrealize()
\end{quote}

\large{\textbf{\textit{Inherited from gtk.Object}}}

\begin{quote}
do\_destroy(), flags(), remove\_data(), remove\_no\_notify(), set\_flags(), unset\_flags()
\end{quote}

\large{\textbf{\textit{Inherited from ??.GObject}}}

\begin{quote}
\_\_cmp\_\_(), \_\_copy\_\_(), \_\_deepcopy\_\_(), \_\_delattr\_\_(), \_\_gdoc\_\_(), \_\_gobject\_init\_\_(), \_\_hash\_\_(), \_\_new\_\_(), \_\_repr\_\_(), \_\_setattr\_\_(), chain(), connect(), connect\_after(), connect\_object(), connect\_object\_after(), disconnect(), disconnect\_by\_func(), emit(), emit\_stop\_by\_name(), freeze\_notify(), get\_data(), get\_properties(), get\_property(), handler\_block(), handler\_block\_by\_func(), handler\_disconnect(), handler\_is\_connected(), handler\_unblock(), handler\_unblock\_by\_func(), notify(), props(), set\_data(), set\_properties(), set\_property(), stop\_emission(), thaw\_notify(), weak\_ref()
\end{quote}

\large{\textbf{\textit{Inherited from atk.ImplementorIface}}}

\begin{quote}
ref\_accessible()
\end{quote}

\large{\textbf{\textit{Inherited from gtk.Buildable}}}

\begin{quote}
add\_child(), construct\_child(), do\_add\_child(), do\_construct\_child(), do\_get\_internal\_child(), do\_parser\_finished(), do\_set\_name(), get\_internal\_child(), parser\_finished()
\end{quote}

\large{\textbf{\textit{Inherited from object}}}

\begin{quote}
\_\_getattribute\_\_(), \_\_reduce\_\_(), \_\_reduce\_ex\_\_(), \_\_str\_\_()
\end{quote}

%%%%%%%%%%%%%%%%%%%%%%%%%%%%%%%%%%%%%%%%%%%%%%%%%%%%%%%%%%%%%%%%%%%%%%%%%%%
%%                              Properties                               %%
%%%%%%%%%%%%%%%%%%%%%%%%%%%%%%%%%%%%%%%%%%%%%%%%%%%%%%%%%%%%%%%%%%%%%%%%%%%

  \subsubsection{Properties}

    \vspace{-1cm}
\hspace{\varindent}\begin{longtable}{|p{\varnamewidth}|p{\vardescrwidth}|l}
\cline{1-2}
\cline{1-2} \centering \textbf{Name} & \centering \textbf{Description}& \\
\cline{1-2}
\endhead\cline{1-2}\multicolumn{3}{r}{\small\textit{continued on next page}}\\\endfoot\cline{1-2}
\endlastfoot\multicolumn{2}{|l|}{\textit{Inherited from gtk.Widget}}\\
\multicolumn{2}{|p{\varwidth}|}{\raggedright allocation, name, parent, requisition, saved\_state, state, style, window}\\
\cline{1-2}
\multicolumn{2}{|l|}{\textit{Inherited from ??.GObject}}\\
\multicolumn{2}{|p{\varwidth}|}{\raggedright \_\_grefcount\_\_}\\
\cline{1-2}
\multicolumn{2}{|l|}{\textit{Inherited from object}}\\
\multicolumn{2}{|p{\varwidth}|}{\raggedright \_\_class\_\_}\\
\cline{1-2}
\end{longtable}


%%%%%%%%%%%%%%%%%%%%%%%%%%%%%%%%%%%%%%%%%%%%%%%%%%%%%%%%%%%%%%%%%%%%%%%%%%%
%%                            Class Variables                            %%
%%%%%%%%%%%%%%%%%%%%%%%%%%%%%%%%%%%%%%%%%%%%%%%%%%%%%%%%%%%%%%%%%%%%%%%%%%%

  \subsubsection{Class Variables}

    \vspace{-1cm}
\hspace{\varindent}\begin{longtable}{|p{\varnamewidth}|p{\vardescrwidth}|l}
\cline{1-2}
\cline{1-2} \centering \textbf{Name} & \centering \textbf{Description}& \\
\cline{1-2}
\endhead\cline{1-2}\multicolumn{3}{r}{\small\textit{continued on next page}}\\\endfoot\cline{1-2}
\endlastfoot\raggedright \_\-\_\-g\-s\-i\-g\-n\-a\-l\-s\-\_\-\_\- & \raggedright \textbf{Value:} 
{\tt \{"datapoint-clicked":(gobject.SIGNAL\_RUN\_LAST, gobject.TY\texttt{...}}&\\
\cline{1-2}
\raggedright \_\-\_\-g\-t\-y\-p\-e\-\_\-\_\- & \raggedright \textbf{Value:} 
{\tt {\textless}GType pygtk\_chart+line\_chart+LineChart (169249432){\textgreater}}&\\
\cline{1-2}
\multicolumn{2}{|l|}{\textit{Inherited from pygtk\_chart.chart.Chart \textit{(Section \ref{pygtk_chart:chart:Chart})}}}\\
\multicolumn{2}{|p{\varwidth}|}{\raggedright \_\_gproperties\_\_}\\
\cline{1-2}
\end{longtable}

    \index{pygtk\_chart \textit{(package)}!pygtk\_chart.line\_chart \textit{(module)}!pygtk\_chart.line\_chart.LineChart \textit{(class)}|)}

%%%%%%%%%%%%%%%%%%%%%%%%%%%%%%%%%%%%%%%%%%%%%%%%%%%%%%%%%%%%%%%%%%%%%%%%%%%
%%                           Class Description                           %%
%%%%%%%%%%%%%%%%%%%%%%%%%%%%%%%%%%%%%%%%%%%%%%%%%%%%%%%%%%%%%%%%%%%%%%%%%%%

    \index{pygtk\_chart \textit{(package)}!pygtk\_chart.line\_chart \textit{(module)}!pygtk\_chart.line\_chart.Axis \textit{(class)}|(}
\subsection{Class Axis}

    \label{pygtk_chart:line_chart:Axis}
\begin{tabular}{cccccccccc}
% Line for object, linespec=[False, False, False]
\multicolumn{2}{r}{\settowidth{\BCL}{object}\multirow{2}{\BCL}{object}}
&&
&&
&&
  \\\cline{3-3}
  &&\multicolumn{1}{c|}{}
&&
&&
&&
  \\
% Line for ??.GObject, linespec=[False, False]
\multicolumn{4}{r}{\settowidth{\BCL}{??.GObject}\multirow{2}{\BCL}{??.GObject}}
&&
&&
  \\\cline{5-5}
  &&&&\multicolumn{1}{c|}{}
&&
&&
  \\
% Line for pygtk\_chart.chart\_object.ChartObject, linespec=[False]
\multicolumn{6}{r}{\settowidth{\BCL}{pygtk\_chart.chart\_object.ChartObject}\multirow{2}{\BCL}{pygtk\_chart.chart\_object.ChartObject}}
&&
  \\\cline{7-7}
  &&&&&&\multicolumn{1}{c|}{}
&&
  \\
&&&&&&\multicolumn{2}{l}{\textbf{pygtk\_chart.line\_chart.Axis}}
\end{tabular}

\textbf{Known Subclasses:}
pygtk\_chart.line\_chart.XAxis,
    pygtk\_chart.line\_chart.YAxis

This class represents an axis on the line chart.

(section) Properties

  The Axis class inherits properties from chart\_object.ChartObject. 
  Additional properties:

  \begin{itemize}
  \setlength{\parskip}{0.6ex}
    \item label (a label for the axis, type: string)

    \item show-label (sets whether the axis' label should be shown, type: 
      boolean)

    \item position (position of the axis, type: an axis position constant)

    \item show-tics (sets whether tics should be shown at the axis, type: 
      boolean)

    \item show-tic-lables (sets whether labels should be shown at the tics, 
      type: boolean)

    \item tic-format-function (a function that is used to format the tic 
      labels, default: str)

    \item logarithmic (sets whether the axis should use a logarithmic scale, 
      type: boolean).

  \end{itemize}

(section) Signals

  The Axis class inherits signals from chart\_object.ChartObject.


%%%%%%%%%%%%%%%%%%%%%%%%%%%%%%%%%%%%%%%%%%%%%%%%%%%%%%%%%%%%%%%%%%%%%%%%%%%
%%                                Methods                                %%
%%%%%%%%%%%%%%%%%%%%%%%%%%%%%%%%%%%%%%%%%%%%%%%%%%%%%%%%%%%%%%%%%%%%%%%%%%%

  \subsubsection{Methods}

    \vspace{0.5ex}

\hspace{.8\funcindent}\begin{boxedminipage}{\funcwidth}

    \raggedright \textbf{\_\_init\_\_}(\textit{self}, \textit{range\_calc}, \textit{label})

\setlength{\parskip}{2ex}
    x.\_\_init\_\_(...) initializes x; see x.\_\_class\_\_.\_\_doc\_\_ for 
    signature

\setlength{\parskip}{1ex}
      Overrides: object.\_\_init\_\_ 	extit{(inherited documentation)}

    \end{boxedminipage}

    \vspace{0.5ex}

\hspace{.8\funcindent}\begin{boxedminipage}{\funcwidth}

    \raggedright \textbf{do\_get\_property}(\textit{self}, \textit{property})

\setlength{\parskip}{2ex}
\setlength{\parskip}{1ex}
      Overrides: pygtk\_chart.chart\_object.ChartObject.do\_get\_property

    \end{boxedminipage}

    \vspace{0.5ex}

\hspace{.8\funcindent}\begin{boxedminipage}{\funcwidth}

    \raggedright \textbf{do\_set\_property}(\textit{self}, \textit{property}, \textit{value})

\setlength{\parskip}{2ex}
\setlength{\parskip}{1ex}
      Overrides: pygtk\_chart.chart\_object.ChartObject.do\_set\_property

    \end{boxedminipage}

    \label{pygtk_chart:line_chart:Axis:set_label}
    \index{pygtk\_chart \textit{(package)}!pygtk\_chart.line\_chart \textit{(module)}!pygtk\_chart.line\_chart.Axis \textit{(class)}!pygtk\_chart.line\_chart.Axis.set\_label \textit{(method)}}

    \vspace{0.5ex}

\hspace{.8\funcindent}\begin{boxedminipage}{\funcwidth}

    \raggedright \textbf{set\_label}(\textit{self}, \textit{label})

    \vspace{-1.5ex}

    \rule{\textwidth}{0.5\fboxrule}
\setlength{\parskip}{2ex}
    Set the label of the axis.

\setlength{\parskip}{1ex}
      \textbf{Parameters}
      \vspace{-1ex}

      \begin{quote}
        \begin{Ventry}{xxxxx}

          \item[label]

          new label

            {\it (type=string.)}

        \end{Ventry}

      \end{quote}

    \end{boxedminipage}

    \label{pygtk_chart:line_chart:Axis:get_label}
    \index{pygtk\_chart \textit{(package)}!pygtk\_chart.line\_chart \textit{(module)}!pygtk\_chart.line\_chart.Axis \textit{(class)}!pygtk\_chart.line\_chart.Axis.get\_label \textit{(method)}}

    \vspace{0.5ex}

\hspace{.8\funcindent}\begin{boxedminipage}{\funcwidth}

    \raggedright \textbf{get\_label}(\textit{self})

    \vspace{-1.5ex}

    \rule{\textwidth}{0.5\fboxrule}
\setlength{\parskip}{2ex}
    Returns the current label of the axis.

\setlength{\parskip}{1ex}
      \textbf{Return Value}
    \vspace{-1ex}

      \begin{quote}
      string.

      \end{quote}

    \end{boxedminipage}

    \label{pygtk_chart:line_chart:Axis:set_show_label}
    \index{pygtk\_chart \textit{(package)}!pygtk\_chart.line\_chart \textit{(module)}!pygtk\_chart.line\_chart.Axis \textit{(class)}!pygtk\_chart.line\_chart.Axis.set\_show\_label \textit{(method)}}

    \vspace{0.5ex}

\hspace{.8\funcindent}\begin{boxedminipage}{\funcwidth}

    \raggedright \textbf{set\_show\_label}(\textit{self}, \textit{show})

    \vspace{-1.5ex}

    \rule{\textwidth}{0.5\fboxrule}
\setlength{\parskip}{2ex}
    Set whether to show the axis' label.

\setlength{\parskip}{1ex}
      \textbf{Parameters}
      \vspace{-1ex}

      \begin{quote}
        \begin{Ventry}{xxxx}

          \item[show]

            {\it (type=boolean.)}

        \end{Ventry}

      \end{quote}

    \end{boxedminipage}

    \label{pygtk_chart:line_chart:Axis:get_show_label}
    \index{pygtk\_chart \textit{(package)}!pygtk\_chart.line\_chart \textit{(module)}!pygtk\_chart.line\_chart.Axis \textit{(class)}!pygtk\_chart.line\_chart.Axis.get\_show\_label \textit{(method)}}

    \vspace{0.5ex}

\hspace{.8\funcindent}\begin{boxedminipage}{\funcwidth}

    \raggedright \textbf{get\_show\_label}(\textit{self})

    \vspace{-1.5ex}

    \rule{\textwidth}{0.5\fboxrule}
\setlength{\parskip}{2ex}
    Returns True if the axis' label is shown.

\setlength{\parskip}{1ex}
      \textbf{Return Value}
    \vspace{-1ex}

      \begin{quote}
      boolean.

      \end{quote}

    \end{boxedminipage}

    \label{pygtk_chart:line_chart:Axis:set_position}
    \index{pygtk\_chart \textit{(package)}!pygtk\_chart.line\_chart \textit{(module)}!pygtk\_chart.line\_chart.Axis \textit{(class)}!pygtk\_chart.line\_chart.Axis.set\_position \textit{(method)}}

    \vspace{0.5ex}

\hspace{.8\funcindent}\begin{boxedminipage}{\funcwidth}

    \raggedright \textbf{set\_position}(\textit{self}, \textit{pos})

    \vspace{-1.5ex}

    \rule{\textwidth}{0.5\fboxrule}
\setlength{\parskip}{2ex}
    Set the position of the axis. pos hast to be one these constants: 
    POSITION\_AUTO, POSITION\_BOTTOM, POSITION\_LEFT, POSITION\_RIGHT, 
    POSITION\_TOP.

\setlength{\parskip}{1ex}
    \end{boxedminipage}

    \label{pygtk_chart:line_chart:Axis:get_position}
    \index{pygtk\_chart \textit{(package)}!pygtk\_chart.line\_chart \textit{(module)}!pygtk\_chart.line\_chart.Axis \textit{(class)}!pygtk\_chart.line\_chart.Axis.get\_position \textit{(method)}}

    \vspace{0.5ex}

\hspace{.8\funcindent}\begin{boxedminipage}{\funcwidth}

    \raggedright \textbf{get\_position}(\textit{self})

    \vspace{-1.5ex}

    \rule{\textwidth}{0.5\fboxrule}
\setlength{\parskip}{2ex}
    Returns the position of the axis. (see set\_position for details).

\setlength{\parskip}{1ex}
    \end{boxedminipage}

    \label{pygtk_chart:line_chart:Axis:set_show_tics}
    \index{pygtk\_chart \textit{(package)}!pygtk\_chart.line\_chart \textit{(module)}!pygtk\_chart.line\_chart.Axis \textit{(class)}!pygtk\_chart.line\_chart.Axis.set\_show\_tics \textit{(method)}}

    \vspace{0.5ex}

\hspace{.8\funcindent}\begin{boxedminipage}{\funcwidth}

    \raggedright \textbf{set\_show\_tics}(\textit{self}, \textit{show})

    \vspace{-1.5ex}

    \rule{\textwidth}{0.5\fboxrule}
\setlength{\parskip}{2ex}
    Set whether to draw tics at the axis.

\setlength{\parskip}{1ex}
      \textbf{Parameters}
      \vspace{-1ex}

      \begin{quote}
        \begin{Ventry}{xxxx}

          \item[show]

            {\it (type=boolean.)}

        \end{Ventry}

      \end{quote}

    \end{boxedminipage}

    \label{pygtk_chart:line_chart:Axis:get_show_tics}
    \index{pygtk\_chart \textit{(package)}!pygtk\_chart.line\_chart \textit{(module)}!pygtk\_chart.line\_chart.Axis \textit{(class)}!pygtk\_chart.line\_chart.Axis.get\_show\_tics \textit{(method)}}

    \vspace{0.5ex}

\hspace{.8\funcindent}\begin{boxedminipage}{\funcwidth}

    \raggedright \textbf{get\_show\_tics}(\textit{self})

    \vspace{-1.5ex}

    \rule{\textwidth}{0.5\fboxrule}
\setlength{\parskip}{2ex}
    Returns True if tics are drawn.

\setlength{\parskip}{1ex}
      \textbf{Return Value}
    \vspace{-1ex}

      \begin{quote}
      boolean.

      \end{quote}

    \end{boxedminipage}

    \label{pygtk_chart:line_chart:Axis:set_show_tic_labels}
    \index{pygtk\_chart \textit{(package)}!pygtk\_chart.line\_chart \textit{(module)}!pygtk\_chart.line\_chart.Axis \textit{(class)}!pygtk\_chart.line\_chart.Axis.set\_show\_tic\_labels \textit{(method)}}

    \vspace{0.5ex}

\hspace{.8\funcindent}\begin{boxedminipage}{\funcwidth}

    \raggedright \textbf{set\_show\_tic\_labels}(\textit{self}, \textit{show})

    \vspace{-1.5ex}

    \rule{\textwidth}{0.5\fboxrule}
\setlength{\parskip}{2ex}
    Set whether to draw tic labels. Labels are only drawn if tics are 
    drawn.

\setlength{\parskip}{1ex}
      \textbf{Parameters}
      \vspace{-1ex}

      \begin{quote}
        \begin{Ventry}{xxxx}

          \item[show]

            {\it (type=boolean.)}

        \end{Ventry}

      \end{quote}

    \end{boxedminipage}

    \label{pygtk_chart:line_chart:Axis:get_show_tic_labels}
    \index{pygtk\_chart \textit{(package)}!pygtk\_chart.line\_chart \textit{(module)}!pygtk\_chart.line\_chart.Axis \textit{(class)}!pygtk\_chart.line\_chart.Axis.get\_show\_tic\_labels \textit{(method)}}

    \vspace{0.5ex}

\hspace{.8\funcindent}\begin{boxedminipage}{\funcwidth}

    \raggedright \textbf{get\_show\_tic\_labels}(\textit{self})

    \vspace{-1.5ex}

    \rule{\textwidth}{0.5\fboxrule}
\setlength{\parskip}{2ex}
    Returns True if tic labels are shown.

\setlength{\parskip}{1ex}
      \textbf{Return Value}
    \vspace{-1ex}

      \begin{quote}
      boolean.

      \end{quote}

    \end{boxedminipage}

    \label{pygtk_chart:line_chart:Axis:set_tic_format_function}
    \index{pygtk\_chart \textit{(package)}!pygtk\_chart.line\_chart \textit{(module)}!pygtk\_chart.line\_chart.Axis \textit{(class)}!pygtk\_chart.line\_chart.Axis.set\_tic\_format\_function \textit{(method)}}

    \vspace{0.5ex}

\hspace{.8\funcindent}\begin{boxedminipage}{\funcwidth}

    \raggedright \textbf{set\_tic\_format\_function}(\textit{self}, \textit{func})

    \vspace{-1.5ex}

    \rule{\textwidth}{0.5\fboxrule}
\setlength{\parskip}{2ex}
    Use this to set the function that should be used to label the tics. The
    function should take a number as the only argument and return a string.
    Default: str

\setlength{\parskip}{1ex}
      \textbf{Parameters}
      \vspace{-1ex}

      \begin{quote}
        \begin{Ventry}{xxxx}

          \item[func]

            {\it (type=function.)}

        \end{Ventry}

      \end{quote}

    \end{boxedminipage}

    \label{pygtk_chart:line_chart:Axis:get_tic_format_function}
    \index{pygtk\_chart \textit{(package)}!pygtk\_chart.line\_chart \textit{(module)}!pygtk\_chart.line\_chart.Axis \textit{(class)}!pygtk\_chart.line\_chart.Axis.get\_tic\_format\_function \textit{(method)}}

    \vspace{0.5ex}

\hspace{.8\funcindent}\begin{boxedminipage}{\funcwidth}

    \raggedright \textbf{get\_tic\_format\_function}(\textit{self})

    \vspace{-1.5ex}

    \rule{\textwidth}{0.5\fboxrule}
\setlength{\parskip}{2ex}
    Returns the function currently used for labeling the tics.

\setlength{\parskip}{1ex}
    \end{boxedminipage}

    \label{pygtk_chart:line_chart:Axis:set_logarithmic}
    \index{pygtk\_chart \textit{(package)}!pygtk\_chart.line\_chart \textit{(module)}!pygtk\_chart.line\_chart.Axis \textit{(class)}!pygtk\_chart.line\_chart.Axis.set\_logarithmic \textit{(method)}}

    \vspace{0.5ex}

\hspace{.8\funcindent}\begin{boxedminipage}{\funcwidth}

    \raggedright \textbf{set\_logarithmic}(\textit{self}, \textit{log})

    \vspace{-1.5ex}

    \rule{\textwidth}{0.5\fboxrule}
\setlength{\parskip}{2ex}
    Set whether the axis should use logarithmic (base 10) scale.

\setlength{\parskip}{1ex}
      \textbf{Parameters}
      \vspace{-1ex}

      \begin{quote}
        \begin{Ventry}{xxx}

          \item[log]

            {\it (type=boolean.)}

        \end{Ventry}

      \end{quote}

    \end{boxedminipage}

    \label{pygtk_chart:line_chart:Axis:get_logarithmic}
    \index{pygtk\_chart \textit{(package)}!pygtk\_chart.line\_chart \textit{(module)}!pygtk\_chart.line\_chart.Axis \textit{(class)}!pygtk\_chart.line\_chart.Axis.get\_logarithmic \textit{(method)}}

    \vspace{0.5ex}

\hspace{.8\funcindent}\begin{boxedminipage}{\funcwidth}

    \raggedright \textbf{get\_logarithmic}(\textit{self})

    \vspace{-1.5ex}

    \rule{\textwidth}{0.5\fboxrule}
\setlength{\parskip}{2ex}
    Returns True if the axis uses logarithmic scale.

\setlength{\parskip}{1ex}
      \textbf{Return Value}
    \vspace{-1ex}

      \begin{quote}
      boolean.

      \end{quote}

    \end{boxedminipage}


\large{\textbf{\textit{Inherited from pygtk\_chart.chart\_object.ChartObject\textit{(Section \ref{pygtk_chart:chart_object:ChartObject})}}}}

\begin{quote}
draw(), get\_antialias(), get\_visible(), set\_antialias(), set\_visible()
\end{quote}

\large{\textbf{\textit{Inherited from ??.GObject}}}

\begin{quote}
\_\_cmp\_\_(), \_\_copy\_\_(), \_\_deepcopy\_\_(), \_\_delattr\_\_(), \_\_gdoc\_\_(), \_\_gobject\_init\_\_(), \_\_hash\_\_(), \_\_new\_\_(), \_\_repr\_\_(), \_\_setattr\_\_(), chain(), connect(), connect\_after(), connect\_object(), connect\_object\_after(), disconnect(), disconnect\_by\_func(), emit(), emit\_stop\_by\_name(), freeze\_notify(), get\_data(), get\_properties(), get\_property(), handler\_block(), handler\_block\_by\_func(), handler\_disconnect(), handler\_is\_connected(), handler\_unblock(), handler\_unblock\_by\_func(), notify(), props(), set\_data(), set\_properties(), set\_property(), stop\_emission(), thaw\_notify(), weak\_ref()
\end{quote}

\large{\textbf{\textit{Inherited from object}}}

\begin{quote}
\_\_getattribute\_\_(), \_\_reduce\_\_(), \_\_reduce\_ex\_\_(), \_\_str\_\_()
\end{quote}

%%%%%%%%%%%%%%%%%%%%%%%%%%%%%%%%%%%%%%%%%%%%%%%%%%%%%%%%%%%%%%%%%%%%%%%%%%%
%%                              Properties                               %%
%%%%%%%%%%%%%%%%%%%%%%%%%%%%%%%%%%%%%%%%%%%%%%%%%%%%%%%%%%%%%%%%%%%%%%%%%%%

  \subsubsection{Properties}

    \vspace{-1cm}
\hspace{\varindent}\begin{longtable}{|p{\varnamewidth}|p{\vardescrwidth}|l}
\cline{1-2}
\cline{1-2} \centering \textbf{Name} & \centering \textbf{Description}& \\
\cline{1-2}
\endhead\cline{1-2}\multicolumn{3}{r}{\small\textit{continued on next page}}\\\endfoot\cline{1-2}
\endlastfoot\multicolumn{2}{|l|}{\textit{Inherited from ??.GObject}}\\
\multicolumn{2}{|p{\varwidth}|}{\raggedright \_\_grefcount\_\_}\\
\cline{1-2}
\multicolumn{2}{|l|}{\textit{Inherited from object}}\\
\multicolumn{2}{|p{\varwidth}|}{\raggedright \_\_class\_\_}\\
\cline{1-2}
\end{longtable}


%%%%%%%%%%%%%%%%%%%%%%%%%%%%%%%%%%%%%%%%%%%%%%%%%%%%%%%%%%%%%%%%%%%%%%%%%%%
%%                            Class Variables                            %%
%%%%%%%%%%%%%%%%%%%%%%%%%%%%%%%%%%%%%%%%%%%%%%%%%%%%%%%%%%%%%%%%%%%%%%%%%%%

  \subsubsection{Class Variables}

    \vspace{-1cm}
\hspace{\varindent}\begin{longtable}{|p{\varnamewidth}|p{\vardescrwidth}|l}
\cline{1-2}
\cline{1-2} \centering \textbf{Name} & \centering \textbf{Description}& \\
\cline{1-2}
\endhead\cline{1-2}\multicolumn{3}{r}{\small\textit{continued on next page}}\\\endfoot\cline{1-2}
\endlastfoot\raggedright \_\-\_\-g\-p\-r\-o\-p\-e\-r\-t\-i\-e\-s\-\_\-\_\- & \raggedright \textbf{Value:} 
{\tt \{"label":(gobject.TYPE\_STRING, "axis label", "The label o\texttt{...}}&\\
\cline{1-2}
\raggedright \_\-\_\-g\-t\-y\-p\-e\-\_\-\_\- & \raggedright \textbf{Value:} 
{\tt {\textless}GType pygtk\_chart+line\_chart+Axis (168640912){\textgreater}}&\\
\cline{1-2}
\multicolumn{2}{|l|}{\textit{Inherited from pygtk\_chart.chart\_object.ChartObject \textit{(Section \ref{pygtk_chart:chart_object:ChartObject})}}}\\
\multicolumn{2}{|p{\varwidth}|}{\raggedright \_\_gsignals\_\_}\\
\cline{1-2}
\end{longtable}

    \index{pygtk\_chart \textit{(package)}!pygtk\_chart.line\_chart \textit{(module)}!pygtk\_chart.line\_chart.Axis \textit{(class)}|)}

%%%%%%%%%%%%%%%%%%%%%%%%%%%%%%%%%%%%%%%%%%%%%%%%%%%%%%%%%%%%%%%%%%%%%%%%%%%
%%                           Class Description                           %%
%%%%%%%%%%%%%%%%%%%%%%%%%%%%%%%%%%%%%%%%%%%%%%%%%%%%%%%%%%%%%%%%%%%%%%%%%%%

    \index{pygtk\_chart \textit{(package)}!pygtk\_chart.line\_chart \textit{(module)}!pygtk\_chart.line\_chart.XAxis \textit{(class)}|(}
\subsection{Class XAxis}

    \label{pygtk_chart:line_chart:XAxis}
\begin{tabular}{cccccccccccc}
% Line for object, linespec=[False, False, False, False]
\multicolumn{2}{r}{\settowidth{\BCL}{object}\multirow{2}{\BCL}{object}}
&&
&&
&&
&&
  \\\cline{3-3}
  &&\multicolumn{1}{c|}{}
&&
&&
&&
&&
  \\
% Line for ??.GObject, linespec=[False, False, False]
\multicolumn{4}{r}{\settowidth{\BCL}{??.GObject}\multirow{2}{\BCL}{??.GObject}}
&&
&&
&&
  \\\cline{5-5}
  &&&&\multicolumn{1}{c|}{}
&&
&&
&&
  \\
% Line for pygtk\_chart.chart\_object.ChartObject, linespec=[False, False]
\multicolumn{6}{r}{\settowidth{\BCL}{pygtk\_chart.chart\_object.ChartObject}\multirow{2}{\BCL}{pygtk\_chart.chart\_object.ChartObject}}
&&
&&
  \\\cline{7-7}
  &&&&&&\multicolumn{1}{c|}{}
&&
&&
  \\
% Line for pygtk\_chart.line\_chart.Axis, linespec=[False]
\multicolumn{8}{r}{\settowidth{\BCL}{pygtk\_chart.line\_chart.Axis}\multirow{2}{\BCL}{pygtk\_chart.line\_chart.Axis}}
&&
  \\\cline{9-9}
  &&&&&&&&\multicolumn{1}{c|}{}
&&
  \\
&&&&&&&&\multicolumn{2}{l}{\textbf{pygtk\_chart.line\_chart.XAxis}}
\end{tabular}

This class represents the xaxis. It is used by the LineChart widget 
internally, there is no need to create an instance yourself.

(section) Properties

  The XAxis class inherits properties from Axis.

(section) Signals

  The XAxis class inherits signals from Axis.


%%%%%%%%%%%%%%%%%%%%%%%%%%%%%%%%%%%%%%%%%%%%%%%%%%%%%%%%%%%%%%%%%%%%%%%%%%%
%%                                Methods                                %%
%%%%%%%%%%%%%%%%%%%%%%%%%%%%%%%%%%%%%%%%%%%%%%%%%%%%%%%%%%%%%%%%%%%%%%%%%%%

  \subsubsection{Methods}

    \vspace{0.5ex}

\hspace{.8\funcindent}\begin{boxedminipage}{\funcwidth}

    \raggedright \textbf{\_\_init\_\_}(\textit{self}, \textit{range\_calc})

\setlength{\parskip}{2ex}
    x.\_\_init\_\_(...) initializes x; see x.\_\_class\_\_.\_\_doc\_\_ for 
    signature

\setlength{\parskip}{1ex}
      Overrides: object.\_\_init\_\_ 	extit{(inherited documentation)}

    \end{boxedminipage}

    \vspace{0.5ex}

\hspace{.8\funcindent}\begin{boxedminipage}{\funcwidth}

    \raggedright \textbf{draw}(\textit{self}, \textit{context}, \textit{rect}, \textit{yaxis})

    \vspace{-1.5ex}

    \rule{\textwidth}{0.5\fboxrule}
\setlength{\parskip}{2ex}
    This method is called by the parent Plot instance. It calls \_do\_draw.

\setlength{\parskip}{1ex}
      \textbf{Parameters}
      \vspace{-1ex}

      \begin{quote}
        \begin{Ventry}{xxxxxxx}

          \item[context]

          The context to draw on.

          \item[rect]

          A rectangle representing the charts area.

        \end{Ventry}

      \end{quote}

      Overrides: pygtk\_chart.chart\_object.ChartObject.draw

    \end{boxedminipage}


\large{\textbf{\textit{Inherited from pygtk\_chart.line\_chart.Axis\textit{(Section \ref{pygtk_chart:line_chart:Axis})}}}}

\begin{quote}
do\_get\_property(), do\_set\_property(), get\_label(), get\_logarithmic(), get\_position(), get\_show\_label(), get\_show\_tic\_labels(), get\_show\_tics(), get\_tic\_format\_function(), set\_label(), set\_logarithmic(), set\_position(), set\_show\_label(), set\_show\_tic\_labels(), set\_show\_tics(), set\_tic\_format\_function()
\end{quote}

\large{\textbf{\textit{Inherited from pygtk\_chart.chart\_object.ChartObject\textit{(Section \ref{pygtk_chart:chart_object:ChartObject})}}}}

\begin{quote}
get\_antialias(), get\_visible(), set\_antialias(), set\_visible()
\end{quote}

\large{\textbf{\textit{Inherited from ??.GObject}}}

\begin{quote}
\_\_cmp\_\_(), \_\_copy\_\_(), \_\_deepcopy\_\_(), \_\_delattr\_\_(), \_\_gdoc\_\_(), \_\_gobject\_init\_\_(), \_\_hash\_\_(), \_\_new\_\_(), \_\_repr\_\_(), \_\_setattr\_\_(), chain(), connect(), connect\_after(), connect\_object(), connect\_object\_after(), disconnect(), disconnect\_by\_func(), emit(), emit\_stop\_by\_name(), freeze\_notify(), get\_data(), get\_properties(), get\_property(), handler\_block(), handler\_block\_by\_func(), handler\_disconnect(), handler\_is\_connected(), handler\_unblock(), handler\_unblock\_by\_func(), notify(), props(), set\_data(), set\_properties(), set\_property(), stop\_emission(), thaw\_notify(), weak\_ref()
\end{quote}

\large{\textbf{\textit{Inherited from object}}}

\begin{quote}
\_\_getattribute\_\_(), \_\_reduce\_\_(), \_\_reduce\_ex\_\_(), \_\_str\_\_()
\end{quote}

%%%%%%%%%%%%%%%%%%%%%%%%%%%%%%%%%%%%%%%%%%%%%%%%%%%%%%%%%%%%%%%%%%%%%%%%%%%
%%                              Properties                               %%
%%%%%%%%%%%%%%%%%%%%%%%%%%%%%%%%%%%%%%%%%%%%%%%%%%%%%%%%%%%%%%%%%%%%%%%%%%%

  \subsubsection{Properties}

    \vspace{-1cm}
\hspace{\varindent}\begin{longtable}{|p{\varnamewidth}|p{\vardescrwidth}|l}
\cline{1-2}
\cline{1-2} \centering \textbf{Name} & \centering \textbf{Description}& \\
\cline{1-2}
\endhead\cline{1-2}\multicolumn{3}{r}{\small\textit{continued on next page}}\\\endfoot\cline{1-2}
\endlastfoot\multicolumn{2}{|l|}{\textit{Inherited from ??.GObject}}\\
\multicolumn{2}{|p{\varwidth}|}{\raggedright \_\_grefcount\_\_}\\
\cline{1-2}
\multicolumn{2}{|l|}{\textit{Inherited from object}}\\
\multicolumn{2}{|p{\varwidth}|}{\raggedright \_\_class\_\_}\\
\cline{1-2}
\end{longtable}


%%%%%%%%%%%%%%%%%%%%%%%%%%%%%%%%%%%%%%%%%%%%%%%%%%%%%%%%%%%%%%%%%%%%%%%%%%%
%%                            Class Variables                            %%
%%%%%%%%%%%%%%%%%%%%%%%%%%%%%%%%%%%%%%%%%%%%%%%%%%%%%%%%%%%%%%%%%%%%%%%%%%%

  \subsubsection{Class Variables}

    \vspace{-1cm}
\hspace{\varindent}\begin{longtable}{|p{\varnamewidth}|p{\vardescrwidth}|l}
\cline{1-2}
\cline{1-2} \centering \textbf{Name} & \centering \textbf{Description}& \\
\cline{1-2}
\endhead\cline{1-2}\multicolumn{3}{r}{\small\textit{continued on next page}}\\\endfoot\cline{1-2}
\endlastfoot\multicolumn{2}{|l|}{\textit{Inherited from pygtk\_chart.line\_chart.Axis \textit{(Section \ref{pygtk_chart:line_chart:Axis})}}}\\
\multicolumn{2}{|p{\varwidth}|}{\raggedright \_\_gproperties\_\_, \_\_gtype\_\_}\\
\cline{1-2}
\multicolumn{2}{|l|}{\textit{Inherited from pygtk\_chart.chart\_object.ChartObject \textit{(Section \ref{pygtk_chart:chart_object:ChartObject})}}}\\
\multicolumn{2}{|p{\varwidth}|}{\raggedright \_\_gsignals\_\_}\\
\cline{1-2}
\end{longtable}

    \index{pygtk\_chart \textit{(package)}!pygtk\_chart.line\_chart \textit{(module)}!pygtk\_chart.line\_chart.XAxis \textit{(class)}|)}

%%%%%%%%%%%%%%%%%%%%%%%%%%%%%%%%%%%%%%%%%%%%%%%%%%%%%%%%%%%%%%%%%%%%%%%%%%%
%%                           Class Description                           %%
%%%%%%%%%%%%%%%%%%%%%%%%%%%%%%%%%%%%%%%%%%%%%%%%%%%%%%%%%%%%%%%%%%%%%%%%%%%

    \index{pygtk\_chart \textit{(package)}!pygtk\_chart.line\_chart \textit{(module)}!pygtk\_chart.line\_chart.YAxis \textit{(class)}|(}
\subsection{Class YAxis}

    \label{pygtk_chart:line_chart:YAxis}
\begin{tabular}{cccccccccccc}
% Line for object, linespec=[False, False, False, False]
\multicolumn{2}{r}{\settowidth{\BCL}{object}\multirow{2}{\BCL}{object}}
&&
&&
&&
&&
  \\\cline{3-3}
  &&\multicolumn{1}{c|}{}
&&
&&
&&
&&
  \\
% Line for ??.GObject, linespec=[False, False, False]
\multicolumn{4}{r}{\settowidth{\BCL}{??.GObject}\multirow{2}{\BCL}{??.GObject}}
&&
&&
&&
  \\\cline{5-5}
  &&&&\multicolumn{1}{c|}{}
&&
&&
&&
  \\
% Line for pygtk\_chart.chart\_object.ChartObject, linespec=[False, False]
\multicolumn{6}{r}{\settowidth{\BCL}{pygtk\_chart.chart\_object.ChartObject}\multirow{2}{\BCL}{pygtk\_chart.chart\_object.ChartObject}}
&&
&&
  \\\cline{7-7}
  &&&&&&\multicolumn{1}{c|}{}
&&
&&
  \\
% Line for pygtk\_chart.line\_chart.Axis, linespec=[False]
\multicolumn{8}{r}{\settowidth{\BCL}{pygtk\_chart.line\_chart.Axis}\multirow{2}{\BCL}{pygtk\_chart.line\_chart.Axis}}
&&
  \\\cline{9-9}
  &&&&&&&&\multicolumn{1}{c|}{}
&&
  \\
&&&&&&&&\multicolumn{2}{l}{\textbf{pygtk\_chart.line\_chart.YAxis}}
\end{tabular}

This class represents the yaxis. It is used by the LineChart widget 
internally, there is no need to create an instance yourself.

(section) Properties

  The YAxis class inherits properties from Axis.

(section) Signals

  The YAxis class inherits signals from Axis.


%%%%%%%%%%%%%%%%%%%%%%%%%%%%%%%%%%%%%%%%%%%%%%%%%%%%%%%%%%%%%%%%%%%%%%%%%%%
%%                                Methods                                %%
%%%%%%%%%%%%%%%%%%%%%%%%%%%%%%%%%%%%%%%%%%%%%%%%%%%%%%%%%%%%%%%%%%%%%%%%%%%

  \subsubsection{Methods}

    \vspace{0.5ex}

\hspace{.8\funcindent}\begin{boxedminipage}{\funcwidth}

    \raggedright \textbf{\_\_init\_\_}(\textit{self}, \textit{range\_calc})

\setlength{\parskip}{2ex}
    x.\_\_init\_\_(...) initializes x; see x.\_\_class\_\_.\_\_doc\_\_ for 
    signature

\setlength{\parskip}{1ex}
      Overrides: object.\_\_init\_\_ 	extit{(inherited documentation)}

    \end{boxedminipage}

    \vspace{0.5ex}

\hspace{.8\funcindent}\begin{boxedminipage}{\funcwidth}

    \raggedright \textbf{draw}(\textit{self}, \textit{context}, \textit{rect}, \textit{xaxis})

    \vspace{-1.5ex}

    \rule{\textwidth}{0.5\fboxrule}
\setlength{\parskip}{2ex}
    This method is called by the parent Plot instance. It calls \_do\_draw.

\setlength{\parskip}{1ex}
      \textbf{Parameters}
      \vspace{-1ex}

      \begin{quote}
        \begin{Ventry}{xxxxxxx}

          \item[context]

          The context to draw on.

          \item[rect]

          A rectangle representing the charts area.

        \end{Ventry}

      \end{quote}

      Overrides: pygtk\_chart.chart\_object.ChartObject.draw

    \end{boxedminipage}


\large{\textbf{\textit{Inherited from pygtk\_chart.line\_chart.Axis\textit{(Section \ref{pygtk_chart:line_chart:Axis})}}}}

\begin{quote}
do\_get\_property(), do\_set\_property(), get\_label(), get\_logarithmic(), get\_position(), get\_show\_label(), get\_show\_tic\_labels(), get\_show\_tics(), get\_tic\_format\_function(), set\_label(), set\_logarithmic(), set\_position(), set\_show\_label(), set\_show\_tic\_labels(), set\_show\_tics(), set\_tic\_format\_function()
\end{quote}

\large{\textbf{\textit{Inherited from pygtk\_chart.chart\_object.ChartObject\textit{(Section \ref{pygtk_chart:chart_object:ChartObject})}}}}

\begin{quote}
get\_antialias(), get\_visible(), set\_antialias(), set\_visible()
\end{quote}

\large{\textbf{\textit{Inherited from ??.GObject}}}

\begin{quote}
\_\_cmp\_\_(), \_\_copy\_\_(), \_\_deepcopy\_\_(), \_\_delattr\_\_(), \_\_gdoc\_\_(), \_\_gobject\_init\_\_(), \_\_hash\_\_(), \_\_new\_\_(), \_\_repr\_\_(), \_\_setattr\_\_(), chain(), connect(), connect\_after(), connect\_object(), connect\_object\_after(), disconnect(), disconnect\_by\_func(), emit(), emit\_stop\_by\_name(), freeze\_notify(), get\_data(), get\_properties(), get\_property(), handler\_block(), handler\_block\_by\_func(), handler\_disconnect(), handler\_is\_connected(), handler\_unblock(), handler\_unblock\_by\_func(), notify(), props(), set\_data(), set\_properties(), set\_property(), stop\_emission(), thaw\_notify(), weak\_ref()
\end{quote}

\large{\textbf{\textit{Inherited from object}}}

\begin{quote}
\_\_getattribute\_\_(), \_\_reduce\_\_(), \_\_reduce\_ex\_\_(), \_\_str\_\_()
\end{quote}

%%%%%%%%%%%%%%%%%%%%%%%%%%%%%%%%%%%%%%%%%%%%%%%%%%%%%%%%%%%%%%%%%%%%%%%%%%%
%%                              Properties                               %%
%%%%%%%%%%%%%%%%%%%%%%%%%%%%%%%%%%%%%%%%%%%%%%%%%%%%%%%%%%%%%%%%%%%%%%%%%%%

  \subsubsection{Properties}

    \vspace{-1cm}
\hspace{\varindent}\begin{longtable}{|p{\varnamewidth}|p{\vardescrwidth}|l}
\cline{1-2}
\cline{1-2} \centering \textbf{Name} & \centering \textbf{Description}& \\
\cline{1-2}
\endhead\cline{1-2}\multicolumn{3}{r}{\small\textit{continued on next page}}\\\endfoot\cline{1-2}
\endlastfoot\multicolumn{2}{|l|}{\textit{Inherited from ??.GObject}}\\
\multicolumn{2}{|p{\varwidth}|}{\raggedright \_\_grefcount\_\_}\\
\cline{1-2}
\multicolumn{2}{|l|}{\textit{Inherited from object}}\\
\multicolumn{2}{|p{\varwidth}|}{\raggedright \_\_class\_\_}\\
\cline{1-2}
\end{longtable}


%%%%%%%%%%%%%%%%%%%%%%%%%%%%%%%%%%%%%%%%%%%%%%%%%%%%%%%%%%%%%%%%%%%%%%%%%%%
%%                            Class Variables                            %%
%%%%%%%%%%%%%%%%%%%%%%%%%%%%%%%%%%%%%%%%%%%%%%%%%%%%%%%%%%%%%%%%%%%%%%%%%%%

  \subsubsection{Class Variables}

    \vspace{-1cm}
\hspace{\varindent}\begin{longtable}{|p{\varnamewidth}|p{\vardescrwidth}|l}
\cline{1-2}
\cline{1-2} \centering \textbf{Name} & \centering \textbf{Description}& \\
\cline{1-2}
\endhead\cline{1-2}\multicolumn{3}{r}{\small\textit{continued on next page}}\\\endfoot\cline{1-2}
\endlastfoot\multicolumn{2}{|l|}{\textit{Inherited from pygtk\_chart.line\_chart.Axis \textit{(Section \ref{pygtk_chart:line_chart:Axis})}}}\\
\multicolumn{2}{|p{\varwidth}|}{\raggedright \_\_gproperties\_\_, \_\_gtype\_\_}\\
\cline{1-2}
\multicolumn{2}{|l|}{\textit{Inherited from pygtk\_chart.chart\_object.ChartObject \textit{(Section \ref{pygtk_chart:chart_object:ChartObject})}}}\\
\multicolumn{2}{|p{\varwidth}|}{\raggedright \_\_gsignals\_\_}\\
\cline{1-2}
\end{longtable}

    \index{pygtk\_chart \textit{(package)}!pygtk\_chart.line\_chart \textit{(module)}!pygtk\_chart.line\_chart.YAxis \textit{(class)}|)}

%%%%%%%%%%%%%%%%%%%%%%%%%%%%%%%%%%%%%%%%%%%%%%%%%%%%%%%%%%%%%%%%%%%%%%%%%%%
%%                           Class Description                           %%
%%%%%%%%%%%%%%%%%%%%%%%%%%%%%%%%%%%%%%%%%%%%%%%%%%%%%%%%%%%%%%%%%%%%%%%%%%%

    \index{pygtk\_chart \textit{(package)}!pygtk\_chart.line\_chart \textit{(module)}!pygtk\_chart.line\_chart.Grid \textit{(class)}|(}
\subsection{Class Grid}

    \label{pygtk_chart:line_chart:Grid}
\begin{tabular}{cccccccccc}
% Line for object, linespec=[False, False, False]
\multicolumn{2}{r}{\settowidth{\BCL}{object}\multirow{2}{\BCL}{object}}
&&
&&
&&
  \\\cline{3-3}
  &&\multicolumn{1}{c|}{}
&&
&&
&&
  \\
% Line for ??.GObject, linespec=[False, False]
\multicolumn{4}{r}{\settowidth{\BCL}{??.GObject}\multirow{2}{\BCL}{??.GObject}}
&&
&&
  \\\cline{5-5}
  &&&&\multicolumn{1}{c|}{}
&&
&&
  \\
% Line for pygtk\_chart.chart\_object.ChartObject, linespec=[False]
\multicolumn{6}{r}{\settowidth{\BCL}{pygtk\_chart.chart\_object.ChartObject}\multirow{2}{\BCL}{pygtk\_chart.chart\_object.ChartObject}}
&&
  \\\cline{7-7}
  &&&&&&\multicolumn{1}{c|}{}
&&
  \\
&&&&&&\multicolumn{2}{l}{\textbf{pygtk\_chart.line\_chart.Grid}}
\end{tabular}

A class representing the grid of the chart. It is used by the LineChart 
widget internally, there is no need to create an instance yourself.

(section) Properties

  The Grid class inherits properties from chart\_object.ChartObject. 
  Additional properties:

  \begin{itemize}
  \setlength{\parskip}{0.6ex}
    \item show-horizontal (sets whther to show horizontal grid lines, type: 
      boolean)

    \item show-vertical (sets whther to show vertical grid lines, type: 
      boolean)

    \item color (the color of the grid lines, type: gtk.gdk.Color)

    \item line-style-horizontal (the line style of the horizontal grid lines, 
      type: a line style constant)

    \item line-style-vertical (the line style of the vertical grid lines, type:
      a line style constant).

  \end{itemize}

(section) Signals

  The Grid class inherits signals from chart\_object.ChartObject.


%%%%%%%%%%%%%%%%%%%%%%%%%%%%%%%%%%%%%%%%%%%%%%%%%%%%%%%%%%%%%%%%%%%%%%%%%%%
%%                                Methods                                %%
%%%%%%%%%%%%%%%%%%%%%%%%%%%%%%%%%%%%%%%%%%%%%%%%%%%%%%%%%%%%%%%%%%%%%%%%%%%

  \subsubsection{Methods}

    \vspace{0.5ex}

\hspace{.8\funcindent}\begin{boxedminipage}{\funcwidth}

    \raggedright \textbf{\_\_init\_\_}(\textit{self}, \textit{range\_calc})

\setlength{\parskip}{2ex}
    x.\_\_init\_\_(...) initializes x; see x.\_\_class\_\_.\_\_doc\_\_ for 
    signature

\setlength{\parskip}{1ex}
      Overrides: object.\_\_init\_\_ 	extit{(inherited documentation)}

    \end{boxedminipage}

    \vspace{0.5ex}

\hspace{.8\funcindent}\begin{boxedminipage}{\funcwidth}

    \raggedright \textbf{do\_get\_property}(\textit{self}, \textit{property})

\setlength{\parskip}{2ex}
\setlength{\parskip}{1ex}
      Overrides: pygtk\_chart.chart\_object.ChartObject.do\_get\_property

    \end{boxedminipage}

    \vspace{0.5ex}

\hspace{.8\funcindent}\begin{boxedminipage}{\funcwidth}

    \raggedright \textbf{do\_set\_property}(\textit{self}, \textit{property}, \textit{value})

\setlength{\parskip}{2ex}
\setlength{\parskip}{1ex}
      Overrides: pygtk\_chart.chart\_object.ChartObject.do\_set\_property

    \end{boxedminipage}

    \label{pygtk_chart:line_chart:Grid:set_draw_horizontal_lines}
    \index{pygtk\_chart \textit{(package)}!pygtk\_chart.line\_chart \textit{(module)}!pygtk\_chart.line\_chart.Grid \textit{(class)}!pygtk\_chart.line\_chart.Grid.set\_draw\_horizontal\_lines \textit{(method)}}

    \vspace{0.5ex}

\hspace{.8\funcindent}\begin{boxedminipage}{\funcwidth}

    \raggedright \textbf{set\_draw\_horizontal\_lines}(\textit{self}, \textit{draw})

    \vspace{-1.5ex}

    \rule{\textwidth}{0.5\fboxrule}
\setlength{\parskip}{2ex}
    Set whether to draw horizontal grid lines.

\setlength{\parskip}{1ex}
      \textbf{Parameters}
      \vspace{-1ex}

      \begin{quote}
        \begin{Ventry}{xxxx}

          \item[draw]

            {\it (type=boolean.)}

        \end{Ventry}

      \end{quote}

    \end{boxedminipage}

    \label{pygtk_chart:line_chart:Grid:get_draw_horizontal_lines}
    \index{pygtk\_chart \textit{(package)}!pygtk\_chart.line\_chart \textit{(module)}!pygtk\_chart.line\_chart.Grid \textit{(class)}!pygtk\_chart.line\_chart.Grid.get\_draw\_horizontal\_lines \textit{(method)}}

    \vspace{0.5ex}

\hspace{.8\funcindent}\begin{boxedminipage}{\funcwidth}

    \raggedright \textbf{get\_draw\_horizontal\_lines}(\textit{self})

    \vspace{-1.5ex}

    \rule{\textwidth}{0.5\fboxrule}
\setlength{\parskip}{2ex}
    Returns True if horizontal grid lines are drawn.

\setlength{\parskip}{1ex}
      \textbf{Return Value}
    \vspace{-1ex}

      \begin{quote}
      boolean.

      \end{quote}

    \end{boxedminipage}

    \label{pygtk_chart:line_chart:Grid:set_draw_vertical_lines}
    \index{pygtk\_chart \textit{(package)}!pygtk\_chart.line\_chart \textit{(module)}!pygtk\_chart.line\_chart.Grid \textit{(class)}!pygtk\_chart.line\_chart.Grid.set\_draw\_vertical\_lines \textit{(method)}}

    \vspace{0.5ex}

\hspace{.8\funcindent}\begin{boxedminipage}{\funcwidth}

    \raggedright \textbf{set\_draw\_vertical\_lines}(\textit{self}, \textit{draw})

    \vspace{-1.5ex}

    \rule{\textwidth}{0.5\fboxrule}
\setlength{\parskip}{2ex}
    Set whether to draw vertical grid lines.

\setlength{\parskip}{1ex}
      \textbf{Parameters}
      \vspace{-1ex}

      \begin{quote}
        \begin{Ventry}{xxxx}

          \item[draw]

            {\it (type=boolean.)}

        \end{Ventry}

      \end{quote}

    \end{boxedminipage}

    \label{pygtk_chart:line_chart:Grid:get_draw_vertical_lines}
    \index{pygtk\_chart \textit{(package)}!pygtk\_chart.line\_chart \textit{(module)}!pygtk\_chart.line\_chart.Grid \textit{(class)}!pygtk\_chart.line\_chart.Grid.get\_draw\_vertical\_lines \textit{(method)}}

    \vspace{0.5ex}

\hspace{.8\funcindent}\begin{boxedminipage}{\funcwidth}

    \raggedright \textbf{get\_draw\_vertical\_lines}(\textit{self})

    \vspace{-1.5ex}

    \rule{\textwidth}{0.5\fboxrule}
\setlength{\parskip}{2ex}
    Returns True if vertical grid lines are drawn.

\setlength{\parskip}{1ex}
      \textbf{Return Value}
    \vspace{-1ex}

      \begin{quote}
      boolean.

      \end{quote}

    \end{boxedminipage}

    \label{pygtk_chart:line_chart:Grid:set_color}
    \index{pygtk\_chart \textit{(package)}!pygtk\_chart.line\_chart \textit{(module)}!pygtk\_chart.line\_chart.Grid \textit{(class)}!pygtk\_chart.line\_chart.Grid.set\_color \textit{(method)}}

    \vspace{0.5ex}

\hspace{.8\funcindent}\begin{boxedminipage}{\funcwidth}

    \raggedright \textbf{set\_color}(\textit{self}, \textit{color})

    \vspace{-1.5ex}

    \rule{\textwidth}{0.5\fboxrule}
\setlength{\parskip}{2ex}
    Set the color of the grid.

\setlength{\parskip}{1ex}
      \textbf{Parameters}
      \vspace{-1ex}

      \begin{quote}
        \begin{Ventry}{xxxxx}

          \item[color]

          The new color of the grid.

            {\it (type=gtk.gdk.Color)}

        \end{Ventry}

      \end{quote}

    \end{boxedminipage}

    \label{pygtk_chart:line_chart:Grid:get_color}
    \index{pygtk\_chart \textit{(package)}!pygtk\_chart.line\_chart \textit{(module)}!pygtk\_chart.line\_chart.Grid \textit{(class)}!pygtk\_chart.line\_chart.Grid.get\_color \textit{(method)}}

    \vspace{0.5ex}

\hspace{.8\funcindent}\begin{boxedminipage}{\funcwidth}

    \raggedright \textbf{get\_color}(\textit{self})

    \vspace{-1.5ex}

    \rule{\textwidth}{0.5\fboxrule}
\setlength{\parskip}{2ex}
    Returns the color of the grid.

\setlength{\parskip}{1ex}
      \textbf{Return Value}
    \vspace{-1ex}

      \begin{quote}
      gtk.gdk.Color.

      \end{quote}

    \end{boxedminipage}

    \label{pygtk_chart:line_chart:Grid:set_line_style_horizontal}
    \index{pygtk\_chart \textit{(package)}!pygtk\_chart.line\_chart \textit{(module)}!pygtk\_chart.line\_chart.Grid \textit{(class)}!pygtk\_chart.line\_chart.Grid.set\_line\_style\_horizontal \textit{(method)}}

    \vspace{0.5ex}

\hspace{.8\funcindent}\begin{boxedminipage}{\funcwidth}

    \raggedright \textbf{set\_line\_style\_horizontal}(\textit{self}, \textit{style})

    \vspace{-1.5ex}

    \rule{\textwidth}{0.5\fboxrule}
\setlength{\parskip}{2ex}
    Set the line style of the horizontal grid lines. style has to be one of
    these constants:

    \begin{itemize}
    \setlength{\parskip}{0.6ex}
      \item pygtk\_chart.LINE\_STYLE\_SOLID (default)

      \item pygtk\_chart.LINE\_STYLE\_DOTTED

      \item pygtk\_chart.LINE\_STYLE\_DASHED

      \item pygtk\_chart.LINE\_STYLE\_DASHED\_ASYMMETRIC.

    \end{itemize}

\setlength{\parskip}{1ex}
      \textbf{Parameters}
      \vspace{-1ex}

      \begin{quote}
        \begin{Ventry}{xxxxx}

          \item[style]

          the new line style

            {\it (type=one of the constants above.)}

        \end{Ventry}

      \end{quote}

    \end{boxedminipage}

    \label{pygtk_chart:line_chart:Grid:get_line_style_horizontal}
    \index{pygtk\_chart \textit{(package)}!pygtk\_chart.line\_chart \textit{(module)}!pygtk\_chart.line\_chart.Grid \textit{(class)}!pygtk\_chart.line\_chart.Grid.get\_line\_style\_horizontal \textit{(method)}}

    \vspace{0.5ex}

\hspace{.8\funcindent}\begin{boxedminipage}{\funcwidth}

    \raggedright \textbf{get\_line\_style\_horizontal}(\textit{self})

    \vspace{-1.5ex}

    \rule{\textwidth}{0.5\fboxrule}
\setlength{\parskip}{2ex}
    Returns ths current horizontal line style.

\setlength{\parskip}{1ex}
      \textbf{Return Value}
    \vspace{-1ex}

      \begin{quote}
      a line style constant.

      \end{quote}

    \end{boxedminipage}

    \label{pygtk_chart:line_chart:Grid:set_line_style_vertical}
    \index{pygtk\_chart \textit{(package)}!pygtk\_chart.line\_chart \textit{(module)}!pygtk\_chart.line\_chart.Grid \textit{(class)}!pygtk\_chart.line\_chart.Grid.set\_line\_style\_vertical \textit{(method)}}

    \vspace{0.5ex}

\hspace{.8\funcindent}\begin{boxedminipage}{\funcwidth}

    \raggedright \textbf{set\_line\_style\_vertical}(\textit{self}, \textit{style})

    \vspace{-1.5ex}

    \rule{\textwidth}{0.5\fboxrule}
\setlength{\parskip}{2ex}
    Set the line style of the vertical grid lines. style has to be one of 
    these constants:

    \begin{itemize}
    \setlength{\parskip}{0.6ex}
      \item pygtk\_chart.LINE\_STYLE\_SOLID (default)

      \item pygtk\_chart.LINE\_STYLE\_DOTTED

      \item pygtk\_chart.LINE\_STYLE\_DASHED

      \item pygtk\_chart.LINE\_STYLE\_DASHED\_ASYMMETRIC.

    \end{itemize}

\setlength{\parskip}{1ex}
      \textbf{Parameters}
      \vspace{-1ex}

      \begin{quote}
        \begin{Ventry}{xxxxx}

          \item[style]

          the new line style

            {\it (type=one of the constants above.)}

        \end{Ventry}

      \end{quote}

    \end{boxedminipage}

    \label{pygtk_chart:line_chart:Grid:get_line_style_vertical}
    \index{pygtk\_chart \textit{(package)}!pygtk\_chart.line\_chart \textit{(module)}!pygtk\_chart.line\_chart.Grid \textit{(class)}!pygtk\_chart.line\_chart.Grid.get\_line\_style\_vertical \textit{(method)}}

    \vspace{0.5ex}

\hspace{.8\funcindent}\begin{boxedminipage}{\funcwidth}

    \raggedright \textbf{get\_line\_style\_vertical}(\textit{self})

    \vspace{-1.5ex}

    \rule{\textwidth}{0.5\fboxrule}
\setlength{\parskip}{2ex}
    Returns ths current vertical line style.

\setlength{\parskip}{1ex}
      \textbf{Return Value}
    \vspace{-1ex}

      \begin{quote}
      a line style constant.

      \end{quote}

    \end{boxedminipage}


\large{\textbf{\textit{Inherited from pygtk\_chart.chart\_object.ChartObject\textit{(Section \ref{pygtk_chart:chart_object:ChartObject})}}}}

\begin{quote}
draw(), get\_antialias(), get\_visible(), set\_antialias(), set\_visible()
\end{quote}

\large{\textbf{\textit{Inherited from ??.GObject}}}

\begin{quote}
\_\_cmp\_\_(), \_\_copy\_\_(), \_\_deepcopy\_\_(), \_\_delattr\_\_(), \_\_gdoc\_\_(), \_\_gobject\_init\_\_(), \_\_hash\_\_(), \_\_new\_\_(), \_\_repr\_\_(), \_\_setattr\_\_(), chain(), connect(), connect\_after(), connect\_object(), connect\_object\_after(), disconnect(), disconnect\_by\_func(), emit(), emit\_stop\_by\_name(), freeze\_notify(), get\_data(), get\_properties(), get\_property(), handler\_block(), handler\_block\_by\_func(), handler\_disconnect(), handler\_is\_connected(), handler\_unblock(), handler\_unblock\_by\_func(), notify(), props(), set\_data(), set\_properties(), set\_property(), stop\_emission(), thaw\_notify(), weak\_ref()
\end{quote}

\large{\textbf{\textit{Inherited from object}}}

\begin{quote}
\_\_getattribute\_\_(), \_\_reduce\_\_(), \_\_reduce\_ex\_\_(), \_\_str\_\_()
\end{quote}

%%%%%%%%%%%%%%%%%%%%%%%%%%%%%%%%%%%%%%%%%%%%%%%%%%%%%%%%%%%%%%%%%%%%%%%%%%%
%%                              Properties                               %%
%%%%%%%%%%%%%%%%%%%%%%%%%%%%%%%%%%%%%%%%%%%%%%%%%%%%%%%%%%%%%%%%%%%%%%%%%%%

  \subsubsection{Properties}

    \vspace{-1cm}
\hspace{\varindent}\begin{longtable}{|p{\varnamewidth}|p{\vardescrwidth}|l}
\cline{1-2}
\cline{1-2} \centering \textbf{Name} & \centering \textbf{Description}& \\
\cline{1-2}
\endhead\cline{1-2}\multicolumn{3}{r}{\small\textit{continued on next page}}\\\endfoot\cline{1-2}
\endlastfoot\multicolumn{2}{|l|}{\textit{Inherited from ??.GObject}}\\
\multicolumn{2}{|p{\varwidth}|}{\raggedright \_\_grefcount\_\_}\\
\cline{1-2}
\multicolumn{2}{|l|}{\textit{Inherited from object}}\\
\multicolumn{2}{|p{\varwidth}|}{\raggedright \_\_class\_\_}\\
\cline{1-2}
\end{longtable}


%%%%%%%%%%%%%%%%%%%%%%%%%%%%%%%%%%%%%%%%%%%%%%%%%%%%%%%%%%%%%%%%%%%%%%%%%%%
%%                            Class Variables                            %%
%%%%%%%%%%%%%%%%%%%%%%%%%%%%%%%%%%%%%%%%%%%%%%%%%%%%%%%%%%%%%%%%%%%%%%%%%%%

  \subsubsection{Class Variables}

    \vspace{-1cm}
\hspace{\varindent}\begin{longtable}{|p{\varnamewidth}|p{\vardescrwidth}|l}
\cline{1-2}
\cline{1-2} \centering \textbf{Name} & \centering \textbf{Description}& \\
\cline{1-2}
\endhead\cline{1-2}\multicolumn{3}{r}{\small\textit{continued on next page}}\\\endfoot\cline{1-2}
\endlastfoot\raggedright \_\-\_\-g\-p\-r\-o\-p\-e\-r\-t\-i\-e\-s\-\_\-\_\- & \raggedright \textbf{Value:} 
{\tt \{"show-horizontal":(gobject.TYPE\_BOOLEAN, "show horizonta\texttt{...}}&\\
\cline{1-2}
\raggedright \_\-\_\-g\-t\-y\-p\-e\-\_\-\_\- & \raggedright \textbf{Value:} 
{\tt {\textless}GType pygtk\_chart+line\_chart+Grid (170084616){\textgreater}}&\\
\cline{1-2}
\multicolumn{2}{|l|}{\textit{Inherited from pygtk\_chart.chart\_object.ChartObject \textit{(Section \ref{pygtk_chart:chart_object:ChartObject})}}}\\
\multicolumn{2}{|p{\varwidth}|}{\raggedright \_\_gsignals\_\_}\\
\cline{1-2}
\end{longtable}

    \index{pygtk\_chart \textit{(package)}!pygtk\_chart.line\_chart \textit{(module)}!pygtk\_chart.line\_chart.Grid \textit{(class)}|)}

%%%%%%%%%%%%%%%%%%%%%%%%%%%%%%%%%%%%%%%%%%%%%%%%%%%%%%%%%%%%%%%%%%%%%%%%%%%
%%                           Class Description                           %%
%%%%%%%%%%%%%%%%%%%%%%%%%%%%%%%%%%%%%%%%%%%%%%%%%%%%%%%%%%%%%%%%%%%%%%%%%%%

    \index{pygtk\_chart \textit{(package)}!pygtk\_chart.line\_chart \textit{(module)}!pygtk\_chart.line\_chart.Graph \textit{(class)}|(}
\subsection{Class Graph}

    \label{pygtk_chart:line_chart:Graph}
\begin{tabular}{cccccccccc}
% Line for object, linespec=[False, False, False]
\multicolumn{2}{r}{\settowidth{\BCL}{object}\multirow{2}{\BCL}{object}}
&&
&&
&&
  \\\cline{3-3}
  &&\multicolumn{1}{c|}{}
&&
&&
&&
  \\
% Line for ??.GObject, linespec=[False, False]
\multicolumn{4}{r}{\settowidth{\BCL}{??.GObject}\multirow{2}{\BCL}{??.GObject}}
&&
&&
  \\\cline{5-5}
  &&&&\multicolumn{1}{c|}{}
&&
&&
  \\
% Line for pygtk\_chart.chart\_object.ChartObject, linespec=[False]
\multicolumn{6}{r}{\settowidth{\BCL}{pygtk\_chart.chart\_object.ChartObject}\multirow{2}{\BCL}{pygtk\_chart.chart\_object.ChartObject}}
&&
  \\\cline{7-7}
  &&&&&&\multicolumn{1}{c|}{}
&&
  \\
&&&&&&\multicolumn{2}{l}{\textbf{pygtk\_chart.line\_chart.Graph}}
\end{tabular}

This class represents a graph or the data you want to plot on your 
LineChart widget.

(section) Properties

  The Graph class inherits properties from chart\_object.ChartObject. 
  Additional properties:

  \begin{itemize}
  \setlength{\parskip}{0.6ex}
    \item name (a unique id for the graph, type: string, read only)

    \item title (the graph's title, type: string)

    \item color (the graph's color, type: gtk.gdk.Color)

    \item type (graph type, type: a graph type constant)

    \item point-size (radius of the datapoints in px, type: int in [1, 100])

    \item fill-to (set how to fill space under the graph, type: None, Graph or 
      float)

    \item fill-color (the color of the filling, type: gtk.gdk.Color)

    \item fill-opacity (the opacity of the filling, type: float in [0, 1])

    \item show-values (sets whether y values should be shown at the datapoints,
      type: boolean)

    \item show-title (sets whether ot show the graph's title, type: boolean)

    \item line-style (the graph's line style, type: a line style constant)

    \item point-style (the graph's datapoints' point style, type: a point style
      constant)

    \item clickable (sets whether datapoints are sensitive for clicks, type: 
      boolean)

    \item show-xerrors (sets whether x errors should be shown if error data is 
      available, type: boolean)

    \item show-yerrors (sets whether y errors should be shown if error data is 
      available, type: boolean).

  \end{itemize}

(section) Signals

  The Graph class inherits signals from chart\_object.ChartObject.


%%%%%%%%%%%%%%%%%%%%%%%%%%%%%%%%%%%%%%%%%%%%%%%%%%%%%%%%%%%%%%%%%%%%%%%%%%%
%%                                Methods                                %%
%%%%%%%%%%%%%%%%%%%%%%%%%%%%%%%%%%%%%%%%%%%%%%%%%%%%%%%%%%%%%%%%%%%%%%%%%%%

  \subsubsection{Methods}

    \vspace{0.5ex}

\hspace{.8\funcindent}\begin{boxedminipage}{\funcwidth}

    \raggedright \textbf{\_\_init\_\_}(\textit{self}, \textit{name}, \textit{title}, \textit{data})

    \vspace{-1.5ex}

    \rule{\textwidth}{0.5\fboxrule}
\setlength{\parskip}{2ex}
    Create a new graph instance. data should be a list of x,y pairs. If you
    want to provide error data for a datapoint, the tuple for that point 
    has to be (x, y, xerror, yerror). If you want only one error, set the 
    other to zero. You can mix datapoints with and without error data in 
    data.

\setlength{\parskip}{1ex}
      \textbf{Parameters}
      \vspace{-1ex}

      \begin{quote}
        \begin{Ventry}{xxxxx}

          \item[name]

          A unique name for the graph. This could be everything. It's just 
          a name used internally for identification. You need to know this 
          if you want to access or delete a graph from a chart.

            {\it (type=string)}

          \item[title]

          The graphs title. This can be drawn on the chart.

            {\it (type=string)}

          \item[data]

          This is the data you want to be visualized. For detail see 
          description above.

            {\it (type=list (see above))}

        \end{Ventry}

      \end{quote}

      Overrides: object.\_\_init\_\_

    \end{boxedminipage}

    \vspace{0.5ex}

\hspace{.8\funcindent}\begin{boxedminipage}{\funcwidth}

    \raggedright \textbf{do\_get\_property}(\textit{self}, \textit{property})

\setlength{\parskip}{2ex}
\setlength{\parskip}{1ex}
      Overrides: pygtk\_chart.chart\_object.ChartObject.do\_get\_property

    \end{boxedminipage}

    \vspace{0.5ex}

\hspace{.8\funcindent}\begin{boxedminipage}{\funcwidth}

    \raggedright \textbf{do\_set\_property}(\textit{self}, \textit{property}, \textit{value})

\setlength{\parskip}{2ex}
\setlength{\parskip}{1ex}
      Overrides: pygtk\_chart.chart\_object.ChartObject.do\_set\_property

    \end{boxedminipage}

    \label{pygtk_chart:line_chart:Graph:has_something_to_draw}
    \index{pygtk\_chart \textit{(package)}!pygtk\_chart.line\_chart \textit{(module)}!pygtk\_chart.line\_chart.Graph \textit{(class)}!pygtk\_chart.line\_chart.Graph.has\_something\_to\_draw \textit{(method)}}

    \vspace{0.5ex}

\hspace{.8\funcindent}\begin{boxedminipage}{\funcwidth}

    \raggedright \textbf{has\_something\_to\_draw}(\textit{self})

\setlength{\parskip}{2ex}
\setlength{\parskip}{1ex}
    \end{boxedminipage}

    \label{pygtk_chart:line_chart:Graph:get_x_range}
    \index{pygtk\_chart \textit{(package)}!pygtk\_chart.line\_chart \textit{(module)}!pygtk\_chart.line\_chart.Graph \textit{(class)}!pygtk\_chart.line\_chart.Graph.get\_x\_range \textit{(method)}}

    \vspace{0.5ex}

\hspace{.8\funcindent}\begin{boxedminipage}{\funcwidth}

    \raggedright \textbf{get\_x\_range}(\textit{self})

    \vspace{-1.5ex}

    \rule{\textwidth}{0.5\fboxrule}
\setlength{\parskip}{2ex}
    Get the the endpoints of the x interval.

\setlength{\parskip}{1ex}
      \textbf{Return Value}
    \vspace{-1ex}

      \begin{quote}
      pair of numbers

      \end{quote}

    \end{boxedminipage}

    \label{pygtk_chart:line_chart:Graph:get_y_range}
    \index{pygtk\_chart \textit{(package)}!pygtk\_chart.line\_chart \textit{(module)}!pygtk\_chart.line\_chart.Graph \textit{(class)}!pygtk\_chart.line\_chart.Graph.get\_y\_range \textit{(method)}}

    \vspace{0.5ex}

\hspace{.8\funcindent}\begin{boxedminipage}{\funcwidth}

    \raggedright \textbf{get\_y\_range}(\textit{self})

    \vspace{-1.5ex}

    \rule{\textwidth}{0.5\fboxrule}
\setlength{\parskip}{2ex}
    Get the the endpoints of the y interval.

\setlength{\parskip}{1ex}
      \textbf{Return Value}
    \vspace{-1ex}

      \begin{quote}
      pair of numbers

      \end{quote}

    \end{boxedminipage}

    \label{pygtk_chart:line_chart:Graph:get_name}
    \index{pygtk\_chart \textit{(package)}!pygtk\_chart.line\_chart \textit{(module)}!pygtk\_chart.line\_chart.Graph \textit{(class)}!pygtk\_chart.line\_chart.Graph.get\_name \textit{(method)}}

    \vspace{0.5ex}

\hspace{.8\funcindent}\begin{boxedminipage}{\funcwidth}

    \raggedright \textbf{get\_name}(\textit{self})

    \vspace{-1.5ex}

    \rule{\textwidth}{0.5\fboxrule}
\setlength{\parskip}{2ex}
    Get the name of the graph.

\setlength{\parskip}{1ex}
      \textbf{Return Value}
    \vspace{-1ex}

      \begin{quote}
      string

      \end{quote}

    \end{boxedminipage}

    \label{pygtk_chart:line_chart:Graph:get_title}
    \index{pygtk\_chart \textit{(package)}!pygtk\_chart.line\_chart \textit{(module)}!pygtk\_chart.line\_chart.Graph \textit{(class)}!pygtk\_chart.line\_chart.Graph.get\_title \textit{(method)}}

    \vspace{0.5ex}

\hspace{.8\funcindent}\begin{boxedminipage}{\funcwidth}

    \raggedright \textbf{get\_title}(\textit{self})

    \vspace{-1.5ex}

    \rule{\textwidth}{0.5\fboxrule}
\setlength{\parskip}{2ex}
    Returns the title of the graph.

\setlength{\parskip}{1ex}
      \textbf{Return Value}
    \vspace{-1ex}

      \begin{quote}
      string

      \end{quote}

    \end{boxedminipage}

    \label{pygtk_chart:line_chart:Graph:set_title}
    \index{pygtk\_chart \textit{(package)}!pygtk\_chart.line\_chart \textit{(module)}!pygtk\_chart.line\_chart.Graph \textit{(class)}!pygtk\_chart.line\_chart.Graph.set\_title \textit{(method)}}

    \vspace{0.5ex}

\hspace{.8\funcindent}\begin{boxedminipage}{\funcwidth}

    \raggedright \textbf{set\_title}(\textit{self}, \textit{title})

    \vspace{-1.5ex}

    \rule{\textwidth}{0.5\fboxrule}
\setlength{\parskip}{2ex}
    Set the title of the graph.

\setlength{\parskip}{1ex}
      \textbf{Parameters}
      \vspace{-1ex}

      \begin{quote}
        \begin{Ventry}{xxxxx}

          \item[title]

          The graph's new title.

            {\it (type=string)}

        \end{Ventry}

      \end{quote}

    \end{boxedminipage}

    \label{pygtk_chart:line_chart:Graph:set_range_calc}
    \index{pygtk\_chart \textit{(package)}!pygtk\_chart.line\_chart \textit{(module)}!pygtk\_chart.line\_chart.Graph \textit{(class)}!pygtk\_chart.line\_chart.Graph.set\_range\_calc \textit{(method)}}

    \vspace{0.5ex}

\hspace{.8\funcindent}\begin{boxedminipage}{\funcwidth}

    \raggedright \textbf{set\_range\_calc}(\textit{self}, \textit{range\_calc})

\setlength{\parskip}{2ex}
\setlength{\parskip}{1ex}
    \end{boxedminipage}

    \label{pygtk_chart:line_chart:Graph:get_color}
    \index{pygtk\_chart \textit{(package)}!pygtk\_chart.line\_chart \textit{(module)}!pygtk\_chart.line\_chart.Graph \textit{(class)}!pygtk\_chart.line\_chart.Graph.get\_color \textit{(method)}}

    \vspace{0.5ex}

\hspace{.8\funcindent}\begin{boxedminipage}{\funcwidth}

    \raggedright \textbf{get\_color}(\textit{self})

    \vspace{-1.5ex}

    \rule{\textwidth}{0.5\fboxrule}
\setlength{\parskip}{2ex}
    Returns the current color of the graph or COLOR\_AUTO.

\setlength{\parskip}{1ex}
      \textbf{Return Value}
    \vspace{-1ex}

      \begin{quote}
      gtk.gdk.Color or COLOR\_AUTO.

      \end{quote}

    \end{boxedminipage}

    \label{pygtk_chart:line_chart:Graph:set_color}
    \index{pygtk\_chart \textit{(package)}!pygtk\_chart.line\_chart \textit{(module)}!pygtk\_chart.line\_chart.Graph \textit{(class)}!pygtk\_chart.line\_chart.Graph.set\_color \textit{(method)}}

    \vspace{0.5ex}

\hspace{.8\funcindent}\begin{boxedminipage}{\funcwidth}

    \raggedright \textbf{set\_color}(\textit{self}, \textit{color})

    \vspace{-1.5ex}

    \rule{\textwidth}{0.5\fboxrule}
\setlength{\parskip}{2ex}
    Set the color of the graph. If set to COLOR\_AUTO, the color will be 
    choosen dynamicly.

\setlength{\parskip}{1ex}
      \textbf{Parameters}
      \vspace{-1ex}

      \begin{quote}
        \begin{Ventry}{xxxxx}

          \item[color]

          The new color of the graph.

            {\it (type=gtk.gdk.Color)}

        \end{Ventry}

      \end{quote}

    \end{boxedminipage}

    \label{pygtk_chart:line_chart:Graph:get_type}
    \index{pygtk\_chart \textit{(package)}!pygtk\_chart.line\_chart \textit{(module)}!pygtk\_chart.line\_chart.Graph \textit{(class)}!pygtk\_chart.line\_chart.Graph.get\_type \textit{(method)}}

    \vspace{0.5ex}

\hspace{.8\funcindent}\begin{boxedminipage}{\funcwidth}

    \raggedright \textbf{get\_type}(\textit{self})

    \vspace{-1.5ex}

    \rule{\textwidth}{0.5\fboxrule}
\setlength{\parskip}{2ex}
    Returns the type of the graph.

\setlength{\parskip}{1ex}
      \textbf{Return Value}
    \vspace{-1ex}

      \begin{quote}
      a type constant (see set\_type() for details)

      \end{quote}

    \end{boxedminipage}

    \label{pygtk_chart:line_chart:Graph:set_type}
    \index{pygtk\_chart \textit{(package)}!pygtk\_chart.line\_chart \textit{(module)}!pygtk\_chart.line\_chart.Graph \textit{(class)}!pygtk\_chart.line\_chart.Graph.set\_type \textit{(method)}}

    \vspace{0.5ex}

\hspace{.8\funcindent}\begin{boxedminipage}{\funcwidth}

    \raggedright \textbf{set\_type}(\textit{self}, \textit{type})

    \vspace{-1.5ex}

    \rule{\textwidth}{0.5\fboxrule}
\setlength{\parskip}{2ex}
    Set the type of the graph to one of these:

    \begin{itemize}
    \setlength{\parskip}{0.6ex}
      \item GRAPH\_POINTS: only show points

      \item GRAPH\_LINES: only draw lines

      \item GRAPH\_BOTH: draw points and lines, i.e. connect points with lines

    \end{itemize}

\setlength{\parskip}{1ex}
      \textbf{Parameters}
      \vspace{-1ex}

      \begin{quote}
        \begin{Ventry}{xxxx}

          \item[type]

          One of the constants above.

        \end{Ventry}

      \end{quote}

    \end{boxedminipage}

    \label{pygtk_chart:line_chart:Graph:get_point_size}
    \index{pygtk\_chart \textit{(package)}!pygtk\_chart.line\_chart \textit{(module)}!pygtk\_chart.line\_chart.Graph \textit{(class)}!pygtk\_chart.line\_chart.Graph.get\_point\_size \textit{(method)}}

    \vspace{0.5ex}

\hspace{.8\funcindent}\begin{boxedminipage}{\funcwidth}

    \raggedright \textbf{get\_point\_size}(\textit{self})

    \vspace{-1.5ex}

    \rule{\textwidth}{0.5\fboxrule}
\setlength{\parskip}{2ex}
    Returns the radius of the data points.

\setlength{\parskip}{1ex}
      \textbf{Return Value}
    \vspace{-1ex}

      \begin{quote}
      a poisitive integer

      \end{quote}

    \end{boxedminipage}

    \label{pygtk_chart:line_chart:Graph:set_point_size}
    \index{pygtk\_chart \textit{(package)}!pygtk\_chart.line\_chart \textit{(module)}!pygtk\_chart.line\_chart.Graph \textit{(class)}!pygtk\_chart.line\_chart.Graph.set\_point\_size \textit{(method)}}

    \vspace{0.5ex}

\hspace{.8\funcindent}\begin{boxedminipage}{\funcwidth}

    \raggedright \textbf{set\_point\_size}(\textit{self}, \textit{size})

    \vspace{-1.5ex}

    \rule{\textwidth}{0.5\fboxrule}
\setlength{\parskip}{2ex}
    Set the radius of the drawn points.

\setlength{\parskip}{1ex}
      \textbf{Parameters}
      \vspace{-1ex}

      \begin{quote}
        \begin{Ventry}{xxxx}

          \item[size]

          The new radius of the points.

            {\it (type=a positive integer in [1, 100])}

        \end{Ventry}

      \end{quote}

    \end{boxedminipage}

    \label{pygtk_chart:line_chart:Graph:get_fill_to}
    \index{pygtk\_chart \textit{(package)}!pygtk\_chart.line\_chart \textit{(module)}!pygtk\_chart.line\_chart.Graph \textit{(class)}!pygtk\_chart.line\_chart.Graph.get\_fill\_to \textit{(method)}}

    \vspace{0.5ex}

\hspace{.8\funcindent}\begin{boxedminipage}{\funcwidth}

    \raggedright \textbf{get\_fill\_to}(\textit{self})

    \vspace{-1.5ex}

    \rule{\textwidth}{0.5\fboxrule}
\setlength{\parskip}{2ex}
    The return value of this method depends on the filling under the graph.
    See set\_fill\_to() for details.

\setlength{\parskip}{1ex}
    \end{boxedminipage}

    \label{pygtk_chart:line_chart:Graph:set_fill_to}
    \index{pygtk\_chart \textit{(package)}!pygtk\_chart.line\_chart \textit{(module)}!pygtk\_chart.line\_chart.Graph \textit{(class)}!pygtk\_chart.line\_chart.Graph.set\_fill\_to \textit{(method)}}

    \vspace{0.5ex}

\hspace{.8\funcindent}\begin{boxedminipage}{\funcwidth}

    \raggedright \textbf{set\_fill\_to}(\textit{self}, \textit{fill\_to})

    \vspace{-1.5ex}

    \rule{\textwidth}{0.5\fboxrule}
\setlength{\parskip}{2ex}
    Use this method to specify how the space under the graph should be 
    filled. fill\_to has to be one of these:

    \begin{itemize}
    \setlength{\parskip}{0.6ex}
      \item None: dont't fill the space under the graph.

      \item int or float: fill the space to the value specified (setting 
        fill\_to=0 means filling the space between graph and xaxis).

      \item a Graph object: fill the space between this graph and the graph 
        given as the argument.

    \end{itemize}

    The color of the filling is the graph's color with 30\% opacity.

\setlength{\parskip}{1ex}
      \textbf{Parameters}
      \vspace{-1ex}

      \begin{quote}
        \begin{Ventry}{xxxxxxx}

          \item[fill\_to]

            {\it (type=one of the possibilities listed above.)}

        \end{Ventry}

      \end{quote}

    \end{boxedminipage}

    \label{pygtk_chart:line_chart:Graph:get_fill_color}
    \index{pygtk\_chart \textit{(package)}!pygtk\_chart.line\_chart \textit{(module)}!pygtk\_chart.line\_chart.Graph \textit{(class)}!pygtk\_chart.line\_chart.Graph.get\_fill\_color \textit{(method)}}

    \vspace{0.5ex}

\hspace{.8\funcindent}\begin{boxedminipage}{\funcwidth}

    \raggedright \textbf{get\_fill\_color}(\textit{self})

    \vspace{-1.5ex}

    \rule{\textwidth}{0.5\fboxrule}
\setlength{\parskip}{2ex}
    Returns the color that is used to fill space under the graph or 
    COLOR\_AUTO.

\setlength{\parskip}{1ex}
      \textbf{Return Value}
    \vspace{-1ex}

      \begin{quote}
      gtk.gdk.Color or COLOR\_AUTO.

      \end{quote}

    \end{boxedminipage}

    \label{pygtk_chart:line_chart:Graph:set_fill_color}
    \index{pygtk\_chart \textit{(package)}!pygtk\_chart.line\_chart \textit{(module)}!pygtk\_chart.line\_chart.Graph \textit{(class)}!pygtk\_chart.line\_chart.Graph.set\_fill\_color \textit{(method)}}

    \vspace{0.5ex}

\hspace{.8\funcindent}\begin{boxedminipage}{\funcwidth}

    \raggedright \textbf{set\_fill\_color}(\textit{self}, \textit{color})

    \vspace{-1.5ex}

    \rule{\textwidth}{0.5\fboxrule}
\setlength{\parskip}{2ex}
    Set which color should be used when filling the space under a graph. If
    color is COLOR\_AUTO, the graph's color will be used.

\setlength{\parskip}{1ex}
      \textbf{Parameters}
      \vspace{-1ex}

      \begin{quote}
        \begin{Ventry}{xxxxx}

          \item[color]

            {\it (type=gtk.gdk.Color or COLOR\_AUTO.)}

        \end{Ventry}

      \end{quote}

    \end{boxedminipage}

    \label{pygtk_chart:line_chart:Graph:get_fill_opacity}
    \index{pygtk\_chart \textit{(package)}!pygtk\_chart.line\_chart \textit{(module)}!pygtk\_chart.line\_chart.Graph \textit{(class)}!pygtk\_chart.line\_chart.Graph.get\_fill\_opacity \textit{(method)}}

    \vspace{0.5ex}

\hspace{.8\funcindent}\begin{boxedminipage}{\funcwidth}

    \raggedright \textbf{get\_fill\_opacity}(\textit{self})

    \vspace{-1.5ex}

    \rule{\textwidth}{0.5\fboxrule}
\setlength{\parskip}{2ex}
    Returns the opacity that is used to fill space under the graph.

\setlength{\parskip}{1ex}
    \end{boxedminipage}

    \label{pygtk_chart:line_chart:Graph:set_fill_opacity}
    \index{pygtk\_chart \textit{(package)}!pygtk\_chart.line\_chart \textit{(module)}!pygtk\_chart.line\_chart.Graph \textit{(class)}!pygtk\_chart.line\_chart.Graph.set\_fill\_opacity \textit{(method)}}

    \vspace{0.5ex}

\hspace{.8\funcindent}\begin{boxedminipage}{\funcwidth}

    \raggedright \textbf{set\_fill\_opacity}(\textit{self}, \textit{opacity})

    \vspace{-1.5ex}

    \rule{\textwidth}{0.5\fboxrule}
\setlength{\parskip}{2ex}
    Set which opacity should be used when filling the space under a graph. 
    The default is 0.3.

\setlength{\parskip}{1ex}
      \textbf{Parameters}
      \vspace{-1ex}

      \begin{quote}
        \begin{Ventry}{xxxxxxx}

          \item[opacity]

            {\it (type=float in [0, 1].)}

        \end{Ventry}

      \end{quote}

    \end{boxedminipage}

    \label{pygtk_chart:line_chart:Graph:get_show_values}
    \index{pygtk\_chart \textit{(package)}!pygtk\_chart.line\_chart \textit{(module)}!pygtk\_chart.line\_chart.Graph \textit{(class)}!pygtk\_chart.line\_chart.Graph.get\_show\_values \textit{(method)}}

    \vspace{0.5ex}

\hspace{.8\funcindent}\begin{boxedminipage}{\funcwidth}

    \raggedright \textbf{get\_show\_values}(\textit{self})

    \vspace{-1.5ex}

    \rule{\textwidth}{0.5\fboxrule}
\setlength{\parskip}{2ex}
    Returns True if y values are shown.

\setlength{\parskip}{1ex}
      \textbf{Return Value}
    \vspace{-1ex}

      \begin{quote}
      boolean

      \end{quote}

    \end{boxedminipage}

    \label{pygtk_chart:line_chart:Graph:set_show_values}
    \index{pygtk\_chart \textit{(package)}!pygtk\_chart.line\_chart \textit{(module)}!pygtk\_chart.line\_chart.Graph \textit{(class)}!pygtk\_chart.line\_chart.Graph.set\_show\_values \textit{(method)}}

    \vspace{0.5ex}

\hspace{.8\funcindent}\begin{boxedminipage}{\funcwidth}

    \raggedright \textbf{set\_show\_values}(\textit{self}, \textit{show})

    \vspace{-1.5ex}

    \rule{\textwidth}{0.5\fboxrule}
\setlength{\parskip}{2ex}
    Set whether the y values should be shown (only if graph type is 
    GRAPH\_POINTS or GRAPH\_BOTH).

\setlength{\parskip}{1ex}
      \textbf{Parameters}
      \vspace{-1ex}

      \begin{quote}
        \begin{Ventry}{xxxx}

          \item[show]

            {\it (type=boolean)}

        \end{Ventry}

      \end{quote}

    \end{boxedminipage}

    \label{pygtk_chart:line_chart:Graph:get_show_title}
    \index{pygtk\_chart \textit{(package)}!pygtk\_chart.line\_chart \textit{(module)}!pygtk\_chart.line\_chart.Graph \textit{(class)}!pygtk\_chart.line\_chart.Graph.get\_show\_title \textit{(method)}}

    \vspace{0.5ex}

\hspace{.8\funcindent}\begin{boxedminipage}{\funcwidth}

    \raggedright \textbf{get\_show\_title}(\textit{self})

    \vspace{-1.5ex}

    \rule{\textwidth}{0.5\fboxrule}
\setlength{\parskip}{2ex}
    Returns True if the title of the graph is shown.

\setlength{\parskip}{1ex}
      \textbf{Return Value}
    \vspace{-1ex}

      \begin{quote}
      boolean.

      \end{quote}

    \end{boxedminipage}

    \label{pygtk_chart:line_chart:Graph:set_show_title}
    \index{pygtk\_chart \textit{(package)}!pygtk\_chart.line\_chart \textit{(module)}!pygtk\_chart.line\_chart.Graph \textit{(class)}!pygtk\_chart.line\_chart.Graph.set\_show\_title \textit{(method)}}

    \vspace{0.5ex}

\hspace{.8\funcindent}\begin{boxedminipage}{\funcwidth}

    \raggedright \textbf{set\_show\_title}(\textit{self}, \textit{show})

    \vspace{-1.5ex}

    \rule{\textwidth}{0.5\fboxrule}
\setlength{\parskip}{2ex}
    Set whether to show the graph's title or not.

\setlength{\parskip}{1ex}
      \textbf{Parameters}
      \vspace{-1ex}

      \begin{quote}
        \begin{Ventry}{xxxx}

          \item[show]

            {\it (type=boolean.)}

        \end{Ventry}

      \end{quote}

    \end{boxedminipage}

    \label{pygtk_chart:line_chart:Graph:add_data}
    \index{pygtk\_chart \textit{(package)}!pygtk\_chart.line\_chart \textit{(module)}!pygtk\_chart.line\_chart.Graph \textit{(class)}!pygtk\_chart.line\_chart.Graph.add\_data \textit{(method)}}

    \vspace{0.5ex}

\hspace{.8\funcindent}\begin{boxedminipage}{\funcwidth}

    \raggedright \textbf{add\_data}(\textit{self}, \textit{data\_list})

    \vspace{-1.5ex}

    \rule{\textwidth}{0.5\fboxrule}
\setlength{\parskip}{2ex}
    Add data to the graph. data\_list should be a list of x,y pairs. If you
    want to provide error data for a datapoint, the tuple for that point 
    has to be (x, y, xerror, yerror). If you want only one error, set the 
    other to zero. You can mix datapoints with and without error data in 
    data\_list.

\setlength{\parskip}{1ex}
      \textbf{Parameters}
      \vspace{-1ex}

      \begin{quote}
        \begin{Ventry}{xxxxxxxxx}

          \item[data\_list]

            {\it (type=a list (see above).)}

        \end{Ventry}

      \end{quote}

    \end{boxedminipage}

    \vspace{0.5ex}

\hspace{.8\funcindent}\begin{boxedminipage}{\funcwidth}

    \raggedright \textbf{get\_data}(\textit{self})

    \vspace{-1.5ex}

    \rule{\textwidth}{0.5\fboxrule}
\setlength{\parskip}{2ex}
    Returns the data of the graph.

\setlength{\parskip}{1ex}
      \textbf{Return Value}
    \vspace{-1ex}

      \begin{quote}
      a list of x, y pairs.

      \end{quote}

      Overrides: ??.GObject.get\_data

    \end{boxedminipage}

    \label{pygtk_chart:line_chart:Graph:set_line_style}
    \index{pygtk\_chart \textit{(package)}!pygtk\_chart.line\_chart \textit{(module)}!pygtk\_chart.line\_chart.Graph \textit{(class)}!pygtk\_chart.line\_chart.Graph.set\_line\_style \textit{(method)}}

    \vspace{0.5ex}

\hspace{.8\funcindent}\begin{boxedminipage}{\funcwidth}

    \raggedright \textbf{set\_line\_style}(\textit{self}, \textit{style})

    \vspace{-1.5ex}

    \rule{\textwidth}{0.5\fboxrule}
\setlength{\parskip}{2ex}
    Set the line style that should be used for drawing the graph (if type 
    is line\_chart.GRAPH\_LINES or line\_chart.GRAPH\_BOTH). style has to 
    be one of these constants:

    \begin{itemize}
    \setlength{\parskip}{0.6ex}
      \item pygtk\_chart.LINE\_STYLE\_SOLID (default)

      \item pygtk\_chart.LINE\_STYLE\_DOTTED

      \item pygtk\_chart.LINE\_STYLE\_DASHED

      \item pygtk\_chart.LINE\_STYLE\_DASHED\_ASYMMETRIC.

    \end{itemize}

\setlength{\parskip}{1ex}
      \textbf{Parameters}
      \vspace{-1ex}

      \begin{quote}
        \begin{Ventry}{xxxxx}

          \item[style]

          the new line style

            {\it (type=one of the line style constants above.)}

        \end{Ventry}

      \end{quote}

    \end{boxedminipage}

    \label{pygtk_chart:line_chart:Graph:get_line_style}
    \index{pygtk\_chart \textit{(package)}!pygtk\_chart.line\_chart \textit{(module)}!pygtk\_chart.line\_chart.Graph \textit{(class)}!pygtk\_chart.line\_chart.Graph.get\_line\_style \textit{(method)}}

    \vspace{0.5ex}

\hspace{.8\funcindent}\begin{boxedminipage}{\funcwidth}

    \raggedright \textbf{get\_line\_style}(\textit{self})

    \vspace{-1.5ex}

    \rule{\textwidth}{0.5\fboxrule}
\setlength{\parskip}{2ex}
    Returns the current line style for the graph (see 
    \texttt{set\_line\_style} for details).

\setlength{\parskip}{1ex}
      \textbf{Return Value}
    \vspace{-1ex}

      \begin{quote}
      a line style constant.

      \end{quote}

    \end{boxedminipage}

    \label{pygtk_chart:line_chart:Graph:set_point_style}
    \index{pygtk\_chart \textit{(package)}!pygtk\_chart.line\_chart \textit{(module)}!pygtk\_chart.line\_chart.Graph \textit{(class)}!pygtk\_chart.line\_chart.Graph.set\_point\_style \textit{(method)}}

    \vspace{0.5ex}

\hspace{.8\funcindent}\begin{boxedminipage}{\funcwidth}

    \raggedright \textbf{set\_point\_style}(\textit{self}, \textit{style})

    \vspace{-1.5ex}

    \rule{\textwidth}{0.5\fboxrule}
\setlength{\parskip}{2ex}
    Set the point style that should be used when drawing the graph (if type
    is line\_chart.GRAPH\_POINTS or line\_chart.GRAPH\_BOTH). For style you
    can use one of these constants:

    \begin{itemize}
    \setlength{\parskip}{0.6ex}
      \item pygtk\_chart.POINT\_STYLE\_CIRCLE (default)

      \item pygtk\_chart.POINT\_STYLE\_SQUARE

      \item pygtk\_chart.POINT\_STYLE\_CROSS

      \item pygtk\_chart.POINT\_STYLE\_TRIANGLE\_UP

      \item pygtk\_chart.POINT\_STYLE\_TRIANGLE\_DOWN

      \item pygtk\_chart.POINT\_STYLE\_DIAMOND

    \end{itemize}

    style can also be a gtk.gdk.Pixbuf that should be used as point.

\setlength{\parskip}{1ex}
      \textbf{Parameters}
      \vspace{-1ex}

      \begin{quote}
        \begin{Ventry}{xxxxx}

          \item[style]

          the new point style

            {\it (type=one of the cosnatnts above or gtk.gdk.Pixbuf.)}

        \end{Ventry}

      \end{quote}

    \end{boxedminipage}

    \label{pygtk_chart:line_chart:Graph:get_point_style}
    \index{pygtk\_chart \textit{(package)}!pygtk\_chart.line\_chart \textit{(module)}!pygtk\_chart.line\_chart.Graph \textit{(class)}!pygtk\_chart.line\_chart.Graph.get\_point\_style \textit{(method)}}

    \vspace{0.5ex}

\hspace{.8\funcindent}\begin{boxedminipage}{\funcwidth}

    \raggedright \textbf{get\_point\_style}(\textit{self})

    \vspace{-1.5ex}

    \rule{\textwidth}{0.5\fboxrule}
\setlength{\parskip}{2ex}
    Returns the current point style. See \texttt{set\_point\_style} for 
    details.

\setlength{\parskip}{1ex}
      \textbf{Return Value}
    \vspace{-1ex}

      \begin{quote}
      a point style constant or gtk.gdk.Pixbuf.

      \end{quote}

    \end{boxedminipage}

    \label{pygtk_chart:line_chart:Graph:set_clickable}
    \index{pygtk\_chart \textit{(package)}!pygtk\_chart.line\_chart \textit{(module)}!pygtk\_chart.line\_chart.Graph \textit{(class)}!pygtk\_chart.line\_chart.Graph.set\_clickable \textit{(method)}}

    \vspace{0.5ex}

\hspace{.8\funcindent}\begin{boxedminipage}{\funcwidth}

    \raggedright \textbf{set\_clickable}(\textit{self}, \textit{clickable})

    \vspace{-1.5ex}

    \rule{\textwidth}{0.5\fboxrule}
\setlength{\parskip}{2ex}
    Set whether the datapoints of the graph should be clickable (only if 
    the datapoints are shown). If this is set to True, the LineChart will 
    emit the signal 'datapoint-clicked' when a datapoint was clicked.

\setlength{\parskip}{1ex}
      \textbf{Parameters}
      \vspace{-1ex}

      \begin{quote}
        \begin{Ventry}{xxxxxxxxx}

          \item[clickable]

            {\it (type=boolean.)}

        \end{Ventry}

      \end{quote}

    \end{boxedminipage}

    \label{pygtk_chart:line_chart:Graph:get_clickable}
    \index{pygtk\_chart \textit{(package)}!pygtk\_chart.line\_chart \textit{(module)}!pygtk\_chart.line\_chart.Graph \textit{(class)}!pygtk\_chart.line\_chart.Graph.get\_clickable \textit{(method)}}

    \vspace{0.5ex}

\hspace{.8\funcindent}\begin{boxedminipage}{\funcwidth}

    \raggedright \textbf{get\_clickable}(\textit{self})

    \vspace{-1.5ex}

    \rule{\textwidth}{0.5\fboxrule}
\setlength{\parskip}{2ex}
    Returns True if the datapoints of the graph are clickable.

\setlength{\parskip}{1ex}
      \textbf{Return Value}
    \vspace{-1ex}

      \begin{quote}
      boolean.

      \end{quote}

    \end{boxedminipage}

    \label{pygtk_chart:line_chart:Graph:set_show_xerrors}
    \index{pygtk\_chart \textit{(package)}!pygtk\_chart.line\_chart \textit{(module)}!pygtk\_chart.line\_chart.Graph \textit{(class)}!pygtk\_chart.line\_chart.Graph.set\_show\_xerrors \textit{(method)}}

    \vspace{0.5ex}

\hspace{.8\funcindent}\begin{boxedminipage}{\funcwidth}

    \raggedright \textbf{set\_show\_xerrors}(\textit{self}, \textit{show})

    \vspace{-1.5ex}

    \rule{\textwidth}{0.5\fboxrule}
\setlength{\parskip}{2ex}
    Use this method to set whether x-errorbars should be shown if error 
    data is available.

\setlength{\parskip}{1ex}
      \textbf{Parameters}
      \vspace{-1ex}

      \begin{quote}
        \begin{Ventry}{xxxx}

          \item[show]

            {\it (type=boolean.)}

        \end{Ventry}

      \end{quote}

    \end{boxedminipage}

    \label{pygtk_chart:line_chart:Graph:get_show_xerrors}
    \index{pygtk\_chart \textit{(package)}!pygtk\_chart.line\_chart \textit{(module)}!pygtk\_chart.line\_chart.Graph \textit{(class)}!pygtk\_chart.line\_chart.Graph.get\_show\_xerrors \textit{(method)}}

    \vspace{0.5ex}

\hspace{.8\funcindent}\begin{boxedminipage}{\funcwidth}

    \raggedright \textbf{get\_show\_xerrors}(\textit{self})

    \vspace{-1.5ex}

    \rule{\textwidth}{0.5\fboxrule}
\setlength{\parskip}{2ex}
    Returns True if x-errorbars should be drawn if error data is available.

\setlength{\parskip}{1ex}
      \textbf{Return Value}
    \vspace{-1ex}

      \begin{quote}
      boolean.

      \end{quote}

    \end{boxedminipage}

    \label{pygtk_chart:line_chart:Graph:set_show_yerrors}
    \index{pygtk\_chart \textit{(package)}!pygtk\_chart.line\_chart \textit{(module)}!pygtk\_chart.line\_chart.Graph \textit{(class)}!pygtk\_chart.line\_chart.Graph.set\_show\_yerrors \textit{(method)}}

    \vspace{0.5ex}

\hspace{.8\funcindent}\begin{boxedminipage}{\funcwidth}

    \raggedright \textbf{set\_show\_yerrors}(\textit{self}, \textit{show})

    \vspace{-1.5ex}

    \rule{\textwidth}{0.5\fboxrule}
\setlength{\parskip}{2ex}
    Use this method to set whether y-errorbars should be shown if error 
    data is available.

\setlength{\parskip}{1ex}
      \textbf{Parameters}
      \vspace{-1ex}

      \begin{quote}
        \begin{Ventry}{xxxx}

          \item[show]

            {\it (type=boolean.)}

        \end{Ventry}

      \end{quote}

    \end{boxedminipage}

    \label{pygtk_chart:line_chart:Graph:get_show_yerrors}
    \index{pygtk\_chart \textit{(package)}!pygtk\_chart.line\_chart \textit{(module)}!pygtk\_chart.line\_chart.Graph \textit{(class)}!pygtk\_chart.line\_chart.Graph.get\_show\_yerrors \textit{(method)}}

    \vspace{0.5ex}

\hspace{.8\funcindent}\begin{boxedminipage}{\funcwidth}

    \raggedright \textbf{get\_show\_yerrors}(\textit{self})

    \vspace{-1.5ex}

    \rule{\textwidth}{0.5\fboxrule}
\setlength{\parskip}{2ex}
    Returns True if y-errorbars should be drawn if error data is available.

\setlength{\parskip}{1ex}
      \textbf{Return Value}
    \vspace{-1ex}

      \begin{quote}
      boolean.

      \end{quote}

    \end{boxedminipage}


\large{\textbf{\textit{Inherited from pygtk\_chart.chart\_object.ChartObject\textit{(Section \ref{pygtk_chart:chart_object:ChartObject})}}}}

\begin{quote}
draw(), get\_antialias(), get\_visible(), set\_antialias(), set\_visible()
\end{quote}

\large{\textbf{\textit{Inherited from ??.GObject}}}

\begin{quote}
\_\_cmp\_\_(), \_\_copy\_\_(), \_\_deepcopy\_\_(), \_\_delattr\_\_(), \_\_gdoc\_\_(), \_\_gobject\_init\_\_(), \_\_hash\_\_(), \_\_new\_\_(), \_\_repr\_\_(), \_\_setattr\_\_(), chain(), connect(), connect\_after(), connect\_object(), connect\_object\_after(), disconnect(), disconnect\_by\_func(), emit(), emit\_stop\_by\_name(), freeze\_notify(), get\_properties(), get\_property(), handler\_block(), handler\_block\_by\_func(), handler\_disconnect(), handler\_is\_connected(), handler\_unblock(), handler\_unblock\_by\_func(), notify(), props(), set\_data(), set\_properties(), set\_property(), stop\_emission(), thaw\_notify(), weak\_ref()
\end{quote}

\large{\textbf{\textit{Inherited from object}}}

\begin{quote}
\_\_getattribute\_\_(), \_\_reduce\_\_(), \_\_reduce\_ex\_\_(), \_\_str\_\_()
\end{quote}

%%%%%%%%%%%%%%%%%%%%%%%%%%%%%%%%%%%%%%%%%%%%%%%%%%%%%%%%%%%%%%%%%%%%%%%%%%%
%%                              Properties                               %%
%%%%%%%%%%%%%%%%%%%%%%%%%%%%%%%%%%%%%%%%%%%%%%%%%%%%%%%%%%%%%%%%%%%%%%%%%%%

  \subsubsection{Properties}

    \vspace{-1cm}
\hspace{\varindent}\begin{longtable}{|p{\varnamewidth}|p{\vardescrwidth}|l}
\cline{1-2}
\cline{1-2} \centering \textbf{Name} & \centering \textbf{Description}& \\
\cline{1-2}
\endhead\cline{1-2}\multicolumn{3}{r}{\small\textit{continued on next page}}\\\endfoot\cline{1-2}
\endlastfoot\multicolumn{2}{|l|}{\textit{Inherited from ??.GObject}}\\
\multicolumn{2}{|p{\varwidth}|}{\raggedright \_\_grefcount\_\_}\\
\cline{1-2}
\multicolumn{2}{|l|}{\textit{Inherited from object}}\\
\multicolumn{2}{|p{\varwidth}|}{\raggedright \_\_class\_\_}\\
\cline{1-2}
\end{longtable}


%%%%%%%%%%%%%%%%%%%%%%%%%%%%%%%%%%%%%%%%%%%%%%%%%%%%%%%%%%%%%%%%%%%%%%%%%%%
%%                            Class Variables                            %%
%%%%%%%%%%%%%%%%%%%%%%%%%%%%%%%%%%%%%%%%%%%%%%%%%%%%%%%%%%%%%%%%%%%%%%%%%%%

  \subsubsection{Class Variables}

    \vspace{-1cm}
\hspace{\varindent}\begin{longtable}{|p{\varnamewidth}|p{\vardescrwidth}|l}
\cline{1-2}
\cline{1-2} \centering \textbf{Name} & \centering \textbf{Description}& \\
\cline{1-2}
\endhead\cline{1-2}\multicolumn{3}{r}{\small\textit{continued on next page}}\\\endfoot\cline{1-2}
\endlastfoot\raggedright \_\-\_\-g\-p\-r\-o\-p\-e\-r\-t\-i\-e\-s\-\_\-\_\- & \raggedright \textbf{Value:} 
{\tt \{"name":(gobject.TYPE\_STRING, "graph id", "The graph's un\texttt{...}}&\\
\cline{1-2}
\raggedright \_\-\_\-g\-t\-y\-p\-e\-\_\-\_\- & \raggedright \textbf{Value:} 
{\tt {\textless}GType pygtk\_chart+line\_chart+Graph (170159304){\textgreater}}&\\
\cline{1-2}
\multicolumn{2}{|l|}{\textit{Inherited from pygtk\_chart.chart\_object.ChartObject \textit{(Section \ref{pygtk_chart:chart_object:ChartObject})}}}\\
\multicolumn{2}{|p{\varwidth}|}{\raggedright \_\_gsignals\_\_}\\
\cline{1-2}
\end{longtable}

    \index{pygtk\_chart \textit{(package)}!pygtk\_chart.line\_chart \textit{(module)}!pygtk\_chart.line\_chart.Graph \textit{(class)}|)}

%%%%%%%%%%%%%%%%%%%%%%%%%%%%%%%%%%%%%%%%%%%%%%%%%%%%%%%%%%%%%%%%%%%%%%%%%%%
%%                           Class Description                           %%
%%%%%%%%%%%%%%%%%%%%%%%%%%%%%%%%%%%%%%%%%%%%%%%%%%%%%%%%%%%%%%%%%%%%%%%%%%%

    \index{pygtk\_chart \textit{(package)}!pygtk\_chart.line\_chart \textit{(module)}!pygtk\_chart.line\_chart.Legend \textit{(class)}|(}
\subsection{Class Legend}

    \label{pygtk_chart:line_chart:Legend}
\begin{tabular}{cccccccccc}
% Line for object, linespec=[False, False, False]
\multicolumn{2}{r}{\settowidth{\BCL}{object}\multirow{2}{\BCL}{object}}
&&
&&
&&
  \\\cline{3-3}
  &&\multicolumn{1}{c|}{}
&&
&&
&&
  \\
% Line for ??.GObject, linespec=[False, False]
\multicolumn{4}{r}{\settowidth{\BCL}{??.GObject}\multirow{2}{\BCL}{??.GObject}}
&&
&&
  \\\cline{5-5}
  &&&&\multicolumn{1}{c|}{}
&&
&&
  \\
% Line for pygtk\_chart.chart\_object.ChartObject, linespec=[False]
\multicolumn{6}{r}{\settowidth{\BCL}{pygtk\_chart.chart\_object.ChartObject}\multirow{2}{\BCL}{pygtk\_chart.chart\_object.ChartObject}}
&&
  \\\cline{7-7}
  &&&&&&\multicolumn{1}{c|}{}
&&
  \\
&&&&&&\multicolumn{2}{l}{\textbf{pygtk\_chart.line\_chart.Legend}}
\end{tabular}

This class represents a legend on a line chart.

(section) Properties

  The Legend class inherits properties from chart\_object.ChartObject. 
  Additional properties:

  \begin{itemize}
  \setlength{\parskip}{0.6ex}
    \item position (the legend's position on the chart, type: a corner position
      constant).

  \end{itemize}

(section) Signals

  The Legend class inherits signals from chart\_object.ChartObject.


%%%%%%%%%%%%%%%%%%%%%%%%%%%%%%%%%%%%%%%%%%%%%%%%%%%%%%%%%%%%%%%%%%%%%%%%%%%
%%                                Methods                                %%
%%%%%%%%%%%%%%%%%%%%%%%%%%%%%%%%%%%%%%%%%%%%%%%%%%%%%%%%%%%%%%%%%%%%%%%%%%%

  \subsubsection{Methods}

    \vspace{0.5ex}

\hspace{.8\funcindent}\begin{boxedminipage}{\funcwidth}

    \raggedright \textbf{\_\_init\_\_}(\textit{self})

\setlength{\parskip}{2ex}
    x.\_\_init\_\_(...) initializes x; see x.\_\_class\_\_.\_\_doc\_\_ for 
    signature

\setlength{\parskip}{1ex}
      Overrides: object.\_\_init\_\_ 	extit{(inherited documentation)}

    \end{boxedminipage}

    \vspace{0.5ex}

\hspace{.8\funcindent}\begin{boxedminipage}{\funcwidth}

    \raggedright \textbf{do\_get\_property}(\textit{self}, \textit{property})

\setlength{\parskip}{2ex}
\setlength{\parskip}{1ex}
      Overrides: pygtk\_chart.chart\_object.ChartObject.do\_get\_property

    \end{boxedminipage}

    \vspace{0.5ex}

\hspace{.8\funcindent}\begin{boxedminipage}{\funcwidth}

    \raggedright \textbf{do\_set\_property}(\textit{self}, \textit{property}, \textit{value})

\setlength{\parskip}{2ex}
\setlength{\parskip}{1ex}
      Overrides: pygtk\_chart.chart\_object.ChartObject.do\_set\_property

    \end{boxedminipage}

    \label{pygtk_chart:line_chart:Legend:set_position}
    \index{pygtk\_chart \textit{(package)}!pygtk\_chart.line\_chart \textit{(module)}!pygtk\_chart.line\_chart.Legend \textit{(class)}!pygtk\_chart.line\_chart.Legend.set\_position \textit{(method)}}

    \vspace{0.5ex}

\hspace{.8\funcindent}\begin{boxedminipage}{\funcwidth}

    \raggedright \textbf{set\_position}(\textit{self}, \textit{position})

    \vspace{-1.5ex}

    \rule{\textwidth}{0.5\fboxrule}
\setlength{\parskip}{2ex}
    Set the position of the legend. position has to be one of these 
    position constants:

    \begin{itemize}
    \setlength{\parskip}{0.6ex}
      \item line\_chart.POSITION\_TOP\_RIGHT (default)

      \item line\_chart.POSITION\_BOTTOM\_RIGHT

      \item line\_chart.POSITION\_BOTTOM\_LEFT

      \item line\_chart.POSITION\_TOP\_LEFT

    \end{itemize}

\setlength{\parskip}{1ex}
      \textbf{Parameters}
      \vspace{-1ex}

      \begin{quote}
        \begin{Ventry}{xxxxxxxx}

          \item[position]

          the legend's position

            {\it (type=one of the constants above.)}

        \end{Ventry}

      \end{quote}

    \end{boxedminipage}

    \label{pygtk_chart:line_chart:Legend:get_position}
    \index{pygtk\_chart \textit{(package)}!pygtk\_chart.line\_chart \textit{(module)}!pygtk\_chart.line\_chart.Legend \textit{(class)}!pygtk\_chart.line\_chart.Legend.get\_position \textit{(method)}}

    \vspace{0.5ex}

\hspace{.8\funcindent}\begin{boxedminipage}{\funcwidth}

    \raggedright \textbf{get\_position}(\textit{self})

    \vspace{-1.5ex}

    \rule{\textwidth}{0.5\fboxrule}
\setlength{\parskip}{2ex}
    Returns the position of the legend. See \texttt{set\_position} for 
    details.

\setlength{\parskip}{1ex}
      \textbf{Return Value}
    \vspace{-1ex}

      \begin{quote}
      a position constant.

      \end{quote}

    \end{boxedminipage}


\large{\textbf{\textit{Inherited from pygtk\_chart.chart\_object.ChartObject\textit{(Section \ref{pygtk_chart:chart_object:ChartObject})}}}}

\begin{quote}
draw(), get\_antialias(), get\_visible(), set\_antialias(), set\_visible()
\end{quote}

\large{\textbf{\textit{Inherited from ??.GObject}}}

\begin{quote}
\_\_cmp\_\_(), \_\_copy\_\_(), \_\_deepcopy\_\_(), \_\_delattr\_\_(), \_\_gdoc\_\_(), \_\_gobject\_init\_\_(), \_\_hash\_\_(), \_\_new\_\_(), \_\_repr\_\_(), \_\_setattr\_\_(), chain(), connect(), connect\_after(), connect\_object(), connect\_object\_after(), disconnect(), disconnect\_by\_func(), emit(), emit\_stop\_by\_name(), freeze\_notify(), get\_data(), get\_properties(), get\_property(), handler\_block(), handler\_block\_by\_func(), handler\_disconnect(), handler\_is\_connected(), handler\_unblock(), handler\_unblock\_by\_func(), notify(), props(), set\_data(), set\_properties(), set\_property(), stop\_emission(), thaw\_notify(), weak\_ref()
\end{quote}

\large{\textbf{\textit{Inherited from object}}}

\begin{quote}
\_\_getattribute\_\_(), \_\_reduce\_\_(), \_\_reduce\_ex\_\_(), \_\_str\_\_()
\end{quote}

%%%%%%%%%%%%%%%%%%%%%%%%%%%%%%%%%%%%%%%%%%%%%%%%%%%%%%%%%%%%%%%%%%%%%%%%%%%
%%                              Properties                               %%
%%%%%%%%%%%%%%%%%%%%%%%%%%%%%%%%%%%%%%%%%%%%%%%%%%%%%%%%%%%%%%%%%%%%%%%%%%%

  \subsubsection{Properties}

    \vspace{-1cm}
\hspace{\varindent}\begin{longtable}{|p{\varnamewidth}|p{\vardescrwidth}|l}
\cline{1-2}
\cline{1-2} \centering \textbf{Name} & \centering \textbf{Description}& \\
\cline{1-2}
\endhead\cline{1-2}\multicolumn{3}{r}{\small\textit{continued on next page}}\\\endfoot\cline{1-2}
\endlastfoot\multicolumn{2}{|l|}{\textit{Inherited from ??.GObject}}\\
\multicolumn{2}{|p{\varwidth}|}{\raggedright \_\_grefcount\_\_}\\
\cline{1-2}
\multicolumn{2}{|l|}{\textit{Inherited from object}}\\
\multicolumn{2}{|p{\varwidth}|}{\raggedright \_\_class\_\_}\\
\cline{1-2}
\end{longtable}


%%%%%%%%%%%%%%%%%%%%%%%%%%%%%%%%%%%%%%%%%%%%%%%%%%%%%%%%%%%%%%%%%%%%%%%%%%%
%%                            Class Variables                            %%
%%%%%%%%%%%%%%%%%%%%%%%%%%%%%%%%%%%%%%%%%%%%%%%%%%%%%%%%%%%%%%%%%%%%%%%%%%%

  \subsubsection{Class Variables}

    \vspace{-1cm}
\hspace{\varindent}\begin{longtable}{|p{\varnamewidth}|p{\vardescrwidth}|l}
\cline{1-2}
\cline{1-2} \centering \textbf{Name} & \centering \textbf{Description}& \\
\cline{1-2}
\endhead\cline{1-2}\multicolumn{3}{r}{\small\textit{continued on next page}}\\\endfoot\cline{1-2}
\endlastfoot\raggedright \_\-\_\-g\-p\-r\-o\-p\-e\-r\-t\-i\-e\-s\-\_\-\_\- & \raggedright \textbf{Value:} 
{\tt \{"position":(gobject.TYPE\_INT, "legend position", "Positi\texttt{...}}&\\
\cline{1-2}
\raggedright \_\-\_\-g\-t\-y\-p\-e\-\_\-\_\- & \raggedright \textbf{Value:} 
{\tt {\textless}GType pygtk\_chart+line\_chart+Legend (170094128){\textgreater}}&\\
\cline{1-2}
\multicolumn{2}{|l|}{\textit{Inherited from pygtk\_chart.chart\_object.ChartObject \textit{(Section \ref{pygtk_chart:chart_object:ChartObject})}}}\\
\multicolumn{2}{|p{\varwidth}|}{\raggedright \_\_gsignals\_\_}\\
\cline{1-2}
\end{longtable}

    \index{pygtk\_chart \textit{(package)}!pygtk\_chart.line\_chart \textit{(module)}!pygtk\_chart.line\_chart.Legend \textit{(class)}|)}
    \index{pygtk\_chart \textit{(package)}!pygtk\_chart.line\_chart \textit{(module)}|)}
