%
% API Documentation for pygtkChart
% Module pygtk_chart.pie_chart
%
% Generated by epydoc 3.0.1
% [Mon Sep 28 11:06:52 2009]
%

%%%%%%%%%%%%%%%%%%%%%%%%%%%%%%%%%%%%%%%%%%%%%%%%%%%%%%%%%%%%%%%%%%%%%%%%%%%
%%                          Module Description                           %%
%%%%%%%%%%%%%%%%%%%%%%%%%%%%%%%%%%%%%%%%%%%%%%%%%%%%%%%%%%%%%%%%%%%%%%%%%%%

    \index{pygtk\_chart \textit{(package)}!pygtk\_chart.pie\_chart \textit{(module)}|(}
\section{Module pygtk\_chart.pie\_chart}

    \label{pygtk_chart:pie_chart}
Contains the PieChart widget.

Author: Sven Festersen (sven@sven-festersen.de)


%%%%%%%%%%%%%%%%%%%%%%%%%%%%%%%%%%%%%%%%%%%%%%%%%%%%%%%%%%%%%%%%%%%%%%%%%%%
%%                               Functions                               %%
%%%%%%%%%%%%%%%%%%%%%%%%%%%%%%%%%%%%%%%%%%%%%%%%%%%%%%%%%%%%%%%%%%%%%%%%%%%

  \subsection{Functions}

    \label{pygtk_chart:pie_chart:draw_sector}
    \index{pygtk\_chart \textit{(package)}!pygtk\_chart.pie\_chart \textit{(module)}!pygtk\_chart.pie\_chart.draw\_sector \textit{(function)}}

    \vspace{0.5ex}

\hspace{.8\funcindent}\begin{boxedminipage}{\funcwidth}

    \raggedright \textbf{draw\_sector}(\textit{context}, \textit{cx}, \textit{cy}, \textit{radius}, \textit{angle}, \textit{angle\_offset})

\setlength{\parskip}{2ex}
\setlength{\parskip}{1ex}
    \end{boxedminipage}


%%%%%%%%%%%%%%%%%%%%%%%%%%%%%%%%%%%%%%%%%%%%%%%%%%%%%%%%%%%%%%%%%%%%%%%%%%%
%%                           Class Description                           %%
%%%%%%%%%%%%%%%%%%%%%%%%%%%%%%%%%%%%%%%%%%%%%%%%%%%%%%%%%%%%%%%%%%%%%%%%%%%

    \index{pygtk\_chart \textit{(package)}!pygtk\_chart.pie\_chart \textit{(module)}!pygtk\_chart.pie\_chart.PieArea \textit{(class)}|(}
\subsection{Class PieArea}

    \label{pygtk_chart:pie_chart:PieArea}
\begin{tabular}{cccccccccccc}
% Line for object, linespec=[False, False, False, False]
\multicolumn{2}{r}{\settowidth{\BCL}{object}\multirow{2}{\BCL}{object}}
&&
&&
&&
&&
  \\\cline{3-3}
  &&\multicolumn{1}{c|}{}
&&
&&
&&
&&
  \\
% Line for ??.GObject, linespec=[False, False, False]
\multicolumn{4}{r}{\settowidth{\BCL}{??.GObject}\multirow{2}{\BCL}{??.GObject}}
&&
&&
&&
  \\\cline{5-5}
  &&&&\multicolumn{1}{c|}{}
&&
&&
&&
  \\
% Line for pygtk\_chart.chart\_object.ChartObject, linespec=[False, False]
\multicolumn{6}{r}{\settowidth{\BCL}{pygtk\_chart.chart\_object.ChartObject}\multirow{2}{\BCL}{pygtk\_chart.chart\_object.ChartObject}}
&&
&&
  \\\cline{7-7}
  &&&&&&\multicolumn{1}{c|}{}
&&
&&
  \\
% Line for pygtk\_chart.chart.Area, linespec=[False]
\multicolumn{8}{r}{\settowidth{\BCL}{pygtk\_chart.chart.Area}\multirow{2}{\BCL}{pygtk\_chart.chart.Area}}
&&
  \\\cline{9-9}
  &&&&&&&&\multicolumn{1}{c|}{}
&&
  \\
&&&&&&&&\multicolumn{2}{l}{\textbf{pygtk\_chart.pie\_chart.PieArea}}
\end{tabular}

This class represents the sector of a pie chart.

(section) Properties

  The PieArea class inherits properties from chart.Area.

(section) Signals

  The PieArea class inherits signals from chart.Area.


%%%%%%%%%%%%%%%%%%%%%%%%%%%%%%%%%%%%%%%%%%%%%%%%%%%%%%%%%%%%%%%%%%%%%%%%%%%
%%                                Methods                                %%
%%%%%%%%%%%%%%%%%%%%%%%%%%%%%%%%%%%%%%%%%%%%%%%%%%%%%%%%%%%%%%%%%%%%%%%%%%%

  \subsubsection{Methods}

    \vspace{0.5ex}

\hspace{.8\funcindent}\begin{boxedminipage}{\funcwidth}

    \raggedright \textbf{\_\_init\_\_}(\textit{self}, \textit{name}, \textit{value}, \textit{title}={\tt \texttt{'}\texttt{}\texttt{'}})

\setlength{\parskip}{2ex}
    x.\_\_init\_\_(...) initializes x; see x.\_\_class\_\_.\_\_doc\_\_ for 
    signature

\setlength{\parskip}{1ex}
      Overrides: object.\_\_init\_\_ 	extit{(inherited documentation)}

    \end{boxedminipage}


\large{\textbf{\textit{Inherited from pygtk\_chart.chart.Area\textit{(Section \ref{pygtk_chart:chart:Area})}}}}

\begin{quote}
do\_get\_property(), do\_set\_property(), get\_color(), get\_highlighted(), get\_label(), get\_value(), set\_color(), set\_highlighted(), set\_label(), set\_value()
\end{quote}

\large{\textbf{\textit{Inherited from pygtk\_chart.chart\_object.ChartObject\textit{(Section \ref{pygtk_chart:chart_object:ChartObject})}}}}

\begin{quote}
draw(), get\_antialias(), get\_visible(), set\_antialias(), set\_visible()
\end{quote}

\large{\textbf{\textit{Inherited from ??.GObject}}}

\begin{quote}
\_\_cmp\_\_(), \_\_copy\_\_(), \_\_deepcopy\_\_(), \_\_delattr\_\_(), \_\_gdoc\_\_(), \_\_gobject\_init\_\_(), \_\_hash\_\_(), \_\_new\_\_(), \_\_repr\_\_(), \_\_setattr\_\_(), chain(), connect(), connect\_after(), connect\_object(), connect\_object\_after(), disconnect(), disconnect\_by\_func(), emit(), emit\_stop\_by\_name(), freeze\_notify(), get\_data(), get\_properties(), get\_property(), handler\_block(), handler\_block\_by\_func(), handler\_disconnect(), handler\_is\_connected(), handler\_unblock(), handler\_unblock\_by\_func(), notify(), props(), set\_data(), set\_properties(), set\_property(), stop\_emission(), thaw\_notify(), weak\_ref()
\end{quote}

\large{\textbf{\textit{Inherited from object}}}

\begin{quote}
\_\_getattribute\_\_(), \_\_reduce\_\_(), \_\_reduce\_ex\_\_(), \_\_str\_\_()
\end{quote}

%%%%%%%%%%%%%%%%%%%%%%%%%%%%%%%%%%%%%%%%%%%%%%%%%%%%%%%%%%%%%%%%%%%%%%%%%%%
%%                              Properties                               %%
%%%%%%%%%%%%%%%%%%%%%%%%%%%%%%%%%%%%%%%%%%%%%%%%%%%%%%%%%%%%%%%%%%%%%%%%%%%

  \subsubsection{Properties}

    \vspace{-1cm}
\hspace{\varindent}\begin{longtable}{|p{\varnamewidth}|p{\vardescrwidth}|l}
\cline{1-2}
\cline{1-2} \centering \textbf{Name} & \centering \textbf{Description}& \\
\cline{1-2}
\endhead\cline{1-2}\multicolumn{3}{r}{\small\textit{continued on next page}}\\\endfoot\cline{1-2}
\endlastfoot\multicolumn{2}{|l|}{\textit{Inherited from ??.GObject}}\\
\multicolumn{2}{|p{\varwidth}|}{\raggedright \_\_grefcount\_\_}\\
\cline{1-2}
\multicolumn{2}{|l|}{\textit{Inherited from object}}\\
\multicolumn{2}{|p{\varwidth}|}{\raggedright \_\_class\_\_}\\
\cline{1-2}
\end{longtable}


%%%%%%%%%%%%%%%%%%%%%%%%%%%%%%%%%%%%%%%%%%%%%%%%%%%%%%%%%%%%%%%%%%%%%%%%%%%
%%                            Class Variables                            %%
%%%%%%%%%%%%%%%%%%%%%%%%%%%%%%%%%%%%%%%%%%%%%%%%%%%%%%%%%%%%%%%%%%%%%%%%%%%

  \subsubsection{Class Variables}

    \vspace{-1cm}
\hspace{\varindent}\begin{longtable}{|p{\varnamewidth}|p{\vardescrwidth}|l}
\cline{1-2}
\cline{1-2} \centering \textbf{Name} & \centering \textbf{Description}& \\
\cline{1-2}
\endhead\cline{1-2}\multicolumn{3}{r}{\small\textit{continued on next page}}\\\endfoot\cline{1-2}
\endlastfoot\multicolumn{2}{|l|}{\textit{Inherited from pygtk\_chart.chart.Area \textit{(Section \ref{pygtk_chart:chart:Area})}}}\\
\multicolumn{2}{|p{\varwidth}|}{\raggedright \_\_gproperties\_\_, \_\_gtype\_\_}\\
\cline{1-2}
\multicolumn{2}{|l|}{\textit{Inherited from pygtk\_chart.chart\_object.ChartObject \textit{(Section \ref{pygtk_chart:chart_object:ChartObject})}}}\\
\multicolumn{2}{|p{\varwidth}|}{\raggedright \_\_gsignals\_\_}\\
\cline{1-2}
\end{longtable}

    \index{pygtk\_chart \textit{(package)}!pygtk\_chart.pie\_chart \textit{(module)}!pygtk\_chart.pie\_chart.PieArea \textit{(class)}|)}

%%%%%%%%%%%%%%%%%%%%%%%%%%%%%%%%%%%%%%%%%%%%%%%%%%%%%%%%%%%%%%%%%%%%%%%%%%%
%%                           Class Description                           %%
%%%%%%%%%%%%%%%%%%%%%%%%%%%%%%%%%%%%%%%%%%%%%%%%%%%%%%%%%%%%%%%%%%%%%%%%%%%

    \index{pygtk\_chart \textit{(package)}!pygtk\_chart.pie\_chart \textit{(module)}!pygtk\_chart.pie\_chart.PieChart \textit{(class)}|(}
\subsection{Class PieChart}

    \label{pygtk_chart:pie_chart:PieChart}
\begin{tabular}{cccccccccccccccc}
% Line for object, linespec=[False, False, False, False, False, False]
\multicolumn{2}{r}{\settowidth{\BCL}{object}\multirow{2}{\BCL}{object}}
&&
&&
&&
&&
&&
&&
  \\\cline{3-3}
  &&\multicolumn{1}{c|}{}
&&
&&
&&
&&
&&
&&
  \\
% Line for ??.GObject, linespec=[False, False, False, False, False]
\multicolumn{4}{r}{\settowidth{\BCL}{??.GObject}\multirow{2}{\BCL}{??.GObject}}
&&
&&
&&
&&
&&
  \\\cline{5-5}
  &&&&\multicolumn{1}{c|}{}
&&
&&
&&
&&
&&
  \\
% Line for gtk.Object, linespec=[False, False, False, False]
\multicolumn{6}{r}{\settowidth{\BCL}{gtk.Object}\multirow{2}{\BCL}{gtk.Object}}
&&
&&
&&
&&
  \\\cline{7-7}
  &&&&&&\multicolumn{1}{c|}{}
&&
&&
&&
&&
  \\
% Line for object, linespec=[False, False, True, False, False, False]
\multicolumn{2}{r}{\settowidth{\BCL}{object}\multirow{2}{\BCL}{object}}
&&
&&
&&\multicolumn{1}{|c}{}
&&
&&
&&
  \\\cline{3-3}
  &&\multicolumn{1}{c|}{}
&&
&&
&\multicolumn{1}{|c}{}&
&&
&&
&&
  \\
% Line for gobject.GInterface, linespec=[False, True, False, False, False]
\multicolumn{4}{r}{\settowidth{\BCL}{gobject.GInterface}\multirow{2}{\BCL}{gobject.GInterface}}
&&
&&\multicolumn{1}{|c}{}
&&
&&
&&
  \\\cline{5-5}
  &&&&\multicolumn{1}{c|}{}
&&
&\multicolumn{1}{|c}{}&
&&
&&
&&
  \\
% Line for atk.ImplementorIface, linespec=[True, False, False, False]
\multicolumn{6}{r}{\settowidth{\BCL}{atk.ImplementorIface}\multirow{2}{\BCL}{atk.ImplementorIface}}
&&\multicolumn{1}{|c}{}
&&
&&
&&
  \\\cline{7-7}
  &&&&&&\multicolumn{1}{c|}{}
&\multicolumn{1}{|c}{}&
&&
&&
&&
  \\
% Line for object, linespec=[False, False, True, False, False, False]
\multicolumn{2}{r}{\settowidth{\BCL}{object}\multirow{2}{\BCL}{object}}
&&
&&
&&\multicolumn{1}{|c}{}
&&
&&
&&
  \\\cline{3-3}
  &&\multicolumn{1}{c|}{}
&&
&&
&\multicolumn{1}{|c}{}&
&&
&&
&&
  \\
% Line for gobject.GInterface, linespec=[False, True, False, False, False]
\multicolumn{4}{r}{\settowidth{\BCL}{gobject.GInterface}\multirow{2}{\BCL}{gobject.GInterface}}
&&
&&\multicolumn{1}{|c}{}
&&
&&
&&
  \\\cline{5-5}
  &&&&\multicolumn{1}{c|}{}
&&
&\multicolumn{1}{|c}{}&
&&
&&
&&
  \\
% Line for gtk.Buildable, linespec=[True, False, False, False]
\multicolumn{6}{r}{\settowidth{\BCL}{gtk.Buildable}\multirow{2}{\BCL}{gtk.Buildable}}
&&\multicolumn{1}{|c}{}
&&
&&
&&
  \\\cline{7-7}
  &&&&&&\multicolumn{1}{c|}{}
&\multicolumn{1}{|c}{}&
&&
&&
&&
  \\
% Line for gtk.Widget, linespec=[False, False, False]
\multicolumn{8}{r}{\settowidth{\BCL}{gtk.Widget}\multirow{2}{\BCL}{gtk.Widget}}
&&
&&
&&
  \\\cline{9-9}
  &&&&&&&&\multicolumn{1}{c|}{}
&&
&&
&&
  \\
% Line for gtk.DrawingArea, linespec=[False, False]
\multicolumn{10}{r}{\settowidth{\BCL}{gtk.DrawingArea}\multirow{2}{\BCL}{gtk.DrawingArea}}
&&
&&
  \\\cline{11-11}
  &&&&&&&&&&\multicolumn{1}{c|}{}
&&
&&
  \\
% Line for pygtk\_chart.chart.Chart, linespec=[False]
\multicolumn{12}{r}{\settowidth{\BCL}{pygtk\_chart.chart.Chart}\multirow{2}{\BCL}{pygtk\_chart.chart.Chart}}
&&
  \\\cline{13-13}
  &&&&&&&&&&&&\multicolumn{1}{c|}{}
&&
  \\
&&&&&&&&&&&&\multicolumn{2}{l}{\textbf{pygtk\_chart.pie\_chart.PieChart}}
\end{tabular}

This is the pie chart class.

(section) Properties

  The PieChart class inherits properties from chart.Chart. Additional 
  properties:

  \begin{itemize}
  \setlength{\parskip}{0.6ex}
    \item rotate (the angle that the pie chart should be rotated by in degrees,
      type: int in [0, 360])

    \item draw-shadow (sets whther to draw a shadow under the pie chart, type: 
      boolean)

    \item draw-labels (sets whether to draw area labels, type: boolean)

    \item show-percentage (sets whether to show percentage in area labels, 
      type: boolean)

    \item show-values (sets whether to show values in area labels, type: 
      boolean)

    \item enable-scroll (sets whether the pie chart can be rotated by scrolling
      with the mouse wheel, type: boolean)

    \item enable-mouseover (sets whether a mouse over effect should be added to
      areas, type: boolean).

  \end{itemize}

(section) Signals

  The PieChart class inherits signals from chart.Chart. Additional signals:

  \begin{itemize}
  \setlength{\parskip}{0.6ex}
    \item area-clicked (emitted when an area is clicked)

  \end{itemize}

  callback signature: def callback(piechart, area).


%%%%%%%%%%%%%%%%%%%%%%%%%%%%%%%%%%%%%%%%%%%%%%%%%%%%%%%%%%%%%%%%%%%%%%%%%%%
%%                                Methods                                %%
%%%%%%%%%%%%%%%%%%%%%%%%%%%%%%%%%%%%%%%%%%%%%%%%%%%%%%%%%%%%%%%%%%%%%%%%%%%

  \subsubsection{Methods}

    \vspace{0.5ex}

\hspace{.8\funcindent}\begin{boxedminipage}{\funcwidth}

    \raggedright \textbf{\_\_init\_\_}(\textit{self})

\setlength{\parskip}{2ex}
    x.\_\_init\_\_(...) initializes x; see x.\_\_class\_\_.\_\_doc\_\_ for 
    signature

\setlength{\parskip}{1ex}
      Overrides: object.\_\_init\_\_ 	extit{(inherited documentation)}

    \end{boxedminipage}

    \vspace{0.5ex}

\hspace{.8\funcindent}\begin{boxedminipage}{\funcwidth}

    \raggedright \textbf{do\_get\_property}(\textit{self}, \textit{property})

\setlength{\parskip}{2ex}
\setlength{\parskip}{1ex}
      Overrides: pygtk\_chart.chart.Chart.do\_get\_property

    \end{boxedminipage}

    \vspace{0.5ex}

\hspace{.8\funcindent}\begin{boxedminipage}{\funcwidth}

    \raggedright \textbf{do\_set\_property}(\textit{self}, \textit{property}, \textit{value})

\setlength{\parskip}{2ex}
\setlength{\parskip}{1ex}
      Overrides: pygtk\_chart.chart.Chart.do\_set\_property

    \end{boxedminipage}

    \vspace{0.5ex}

\hspace{.8\funcindent}\begin{boxedminipage}{\funcwidth}

    \raggedright \textbf{draw}(\textit{self}, \textit{context})

    \vspace{-1.5ex}

    \rule{\textwidth}{0.5\fboxrule}
\setlength{\parskip}{2ex}
    Draw the widget. This method is called automatically. Don't call it 
    yourself. If you want to force a redrawing of the widget, call the 
    queue\_draw() method.

\setlength{\parskip}{1ex}
      \textbf{Parameters}
      \vspace{-1ex}

      \begin{quote}
        \begin{Ventry}{xxxxxxx}

          \item[context]

          The context to draw on.

            {\it (type=cairo.Context)}

        \end{Ventry}

      \end{quote}

      Overrides: gtk.Widget.draw

    \end{boxedminipage}

    \label{pygtk_chart:pie_chart:PieChart:add_area}
    \index{pygtk\_chart \textit{(package)}!pygtk\_chart.pie\_chart \textit{(module)}!pygtk\_chart.pie\_chart.PieChart \textit{(class)}!pygtk\_chart.pie\_chart.PieChart.add\_area \textit{(method)}}

    \vspace{0.5ex}

\hspace{.8\funcindent}\begin{boxedminipage}{\funcwidth}

    \raggedright \textbf{add\_area}(\textit{self}, \textit{area})

\setlength{\parskip}{2ex}
\setlength{\parskip}{1ex}
    \end{boxedminipage}

    \label{pygtk_chart:pie_chart:PieChart:get_pie_area}
    \index{pygtk\_chart \textit{(package)}!pygtk\_chart.pie\_chart \textit{(module)}!pygtk\_chart.pie\_chart.PieChart \textit{(class)}!pygtk\_chart.pie\_chart.PieChart.get\_pie\_area \textit{(method)}}

    \vspace{0.5ex}

\hspace{.8\funcindent}\begin{boxedminipage}{\funcwidth}

    \raggedright \textbf{get\_pie\_area}(\textit{self}, \textit{name})

    \vspace{-1.5ex}

    \rule{\textwidth}{0.5\fboxrule}
\setlength{\parskip}{2ex}
    Returns the PieArea with the id 'name' if it exists, None otherwise.

\setlength{\parskip}{1ex}
      \textbf{Parameters}
      \vspace{-1ex}

      \begin{quote}
        \begin{Ventry}{xxxx}

          \item[name]

          the id of a PieArea

            {\it (type=string)}

        \end{Ventry}

      \end{quote}

      \textbf{Return Value}
    \vspace{-1ex}

      \begin{quote}
      a PieArea or None.

      \end{quote}

    \end{boxedminipage}

    \label{pygtk_chart:pie_chart:PieChart:set_rotate}
    \index{pygtk\_chart \textit{(package)}!pygtk\_chart.pie\_chart \textit{(module)}!pygtk\_chart.pie\_chart.PieChart \textit{(class)}!pygtk\_chart.pie\_chart.PieChart.set\_rotate \textit{(method)}}

    \vspace{0.5ex}

\hspace{.8\funcindent}\begin{boxedminipage}{\funcwidth}

    \raggedright \textbf{set\_rotate}(\textit{self}, \textit{angle})

    \vspace{-1.5ex}

    \rule{\textwidth}{0.5\fboxrule}
\setlength{\parskip}{2ex}
    Set the rotation angle of the PieChart in degrees.

\setlength{\parskip}{1ex}
      \textbf{Parameters}
      \vspace{-1ex}

      \begin{quote}
        \begin{Ventry}{xxxxx}

          \item[angle]

          angle in degrees 0 - 360

            {\it (type=integer.)}

        \end{Ventry}

      \end{quote}

    \end{boxedminipage}

    \label{pygtk_chart:pie_chart:PieChart:get_rotate}
    \index{pygtk\_chart \textit{(package)}!pygtk\_chart.pie\_chart \textit{(module)}!pygtk\_chart.pie\_chart.PieChart \textit{(class)}!pygtk\_chart.pie\_chart.PieChart.get\_rotate \textit{(method)}}

    \vspace{0.5ex}

\hspace{.8\funcindent}\begin{boxedminipage}{\funcwidth}

    \raggedright \textbf{get\_rotate}(\textit{self})

    \vspace{-1.5ex}

    \rule{\textwidth}{0.5\fboxrule}
\setlength{\parskip}{2ex}
    Get the current rotation angle in degrees.

\setlength{\parskip}{1ex}
      \textbf{Return Value}
    \vspace{-1ex}

      \begin{quote}
      integer.

      \end{quote}

    \end{boxedminipage}

    \label{pygtk_chart:pie_chart:PieChart:set_draw_shadow}
    \index{pygtk\_chart \textit{(package)}!pygtk\_chart.pie\_chart \textit{(module)}!pygtk\_chart.pie\_chart.PieChart \textit{(class)}!pygtk\_chart.pie\_chart.PieChart.set\_draw\_shadow \textit{(method)}}

    \vspace{0.5ex}

\hspace{.8\funcindent}\begin{boxedminipage}{\funcwidth}

    \raggedright \textbf{set\_draw\_shadow}(\textit{self}, \textit{draw})

    \vspace{-1.5ex}

    \rule{\textwidth}{0.5\fboxrule}
\setlength{\parskip}{2ex}
    Set whether to draw the pie chart's shadow.

\setlength{\parskip}{1ex}
      \textbf{Parameters}
      \vspace{-1ex}

      \begin{quote}
        \begin{Ventry}{xxxx}

          \item[draw]

            {\it (type=boolean.)}

        \end{Ventry}

      \end{quote}

    \end{boxedminipage}

    \label{pygtk_chart:pie_chart:PieChart:get_draw_shadow}
    \index{pygtk\_chart \textit{(package)}!pygtk\_chart.pie\_chart \textit{(module)}!pygtk\_chart.pie\_chart.PieChart \textit{(class)}!pygtk\_chart.pie\_chart.PieChart.get\_draw\_shadow \textit{(method)}}

    \vspace{0.5ex}

\hspace{.8\funcindent}\begin{boxedminipage}{\funcwidth}

    \raggedright \textbf{get\_draw\_shadow}(\textit{self})

    \vspace{-1.5ex}

    \rule{\textwidth}{0.5\fboxrule}
\setlength{\parskip}{2ex}
    Returns True if pie chart currently has a shadow.

\setlength{\parskip}{1ex}
      \textbf{Return Value}
    \vspace{-1ex}

      \begin{quote}
      boolean.

      \end{quote}

    \end{boxedminipage}

    \label{pygtk_chart:pie_chart:PieChart:set_draw_labels}
    \index{pygtk\_chart \textit{(package)}!pygtk\_chart.pie\_chart \textit{(module)}!pygtk\_chart.pie\_chart.PieChart \textit{(class)}!pygtk\_chart.pie\_chart.PieChart.set\_draw\_labels \textit{(method)}}

    \vspace{0.5ex}

\hspace{.8\funcindent}\begin{boxedminipage}{\funcwidth}

    \raggedright \textbf{set\_draw\_labels}(\textit{self}, \textit{draw})

    \vspace{-1.5ex}

    \rule{\textwidth}{0.5\fboxrule}
\setlength{\parskip}{2ex}
    Set whether to draw the labels of the pie areas.

\setlength{\parskip}{1ex}
      \textbf{Parameters}
      \vspace{-1ex}

      \begin{quote}
        \begin{Ventry}{xxxx}

          \item[draw]

            {\it (type=boolean.)}

        \end{Ventry}

      \end{quote}

    \end{boxedminipage}

    \label{pygtk_chart:pie_chart:PieChart:get_draw_labels}
    \index{pygtk\_chart \textit{(package)}!pygtk\_chart.pie\_chart \textit{(module)}!pygtk\_chart.pie\_chart.PieChart \textit{(class)}!pygtk\_chart.pie\_chart.PieChart.get\_draw\_labels \textit{(method)}}

    \vspace{0.5ex}

\hspace{.8\funcindent}\begin{boxedminipage}{\funcwidth}

    \raggedright \textbf{get\_draw\_labels}(\textit{self})

    \vspace{-1.5ex}

    \rule{\textwidth}{0.5\fboxrule}
\setlength{\parskip}{2ex}
    Returns True if area labels are shown.

\setlength{\parskip}{1ex}
      \textbf{Return Value}
    \vspace{-1ex}

      \begin{quote}
      boolean.

      \end{quote}

    \end{boxedminipage}

    \label{pygtk_chart:pie_chart:PieChart:set_show_percentage}
    \index{pygtk\_chart \textit{(package)}!pygtk\_chart.pie\_chart \textit{(module)}!pygtk\_chart.pie\_chart.PieChart \textit{(class)}!pygtk\_chart.pie\_chart.PieChart.set\_show\_percentage \textit{(method)}}

    \vspace{0.5ex}

\hspace{.8\funcindent}\begin{boxedminipage}{\funcwidth}

    \raggedright \textbf{set\_show\_percentage}(\textit{self}, \textit{show})

    \vspace{-1.5ex}

    \rule{\textwidth}{0.5\fboxrule}
\setlength{\parskip}{2ex}
    Set whether to show the percentage an area has in its label.

\setlength{\parskip}{1ex}
      \textbf{Parameters}
      \vspace{-1ex}

      \begin{quote}
        \begin{Ventry}{xxxx}

          \item[show]

            {\it (type=boolean.)}

        \end{Ventry}

      \end{quote}

    \end{boxedminipage}

    \label{pygtk_chart:pie_chart:PieChart:get_show_percentage}
    \index{pygtk\_chart \textit{(package)}!pygtk\_chart.pie\_chart \textit{(module)}!pygtk\_chart.pie\_chart.PieChart \textit{(class)}!pygtk\_chart.pie\_chart.PieChart.get\_show\_percentage \textit{(method)}}

    \vspace{0.5ex}

\hspace{.8\funcindent}\begin{boxedminipage}{\funcwidth}

    \raggedright \textbf{get\_show\_percentage}(\textit{self})

    \vspace{-1.5ex}

    \rule{\textwidth}{0.5\fboxrule}
\setlength{\parskip}{2ex}
    Returns True if percentages are shown.

\setlength{\parskip}{1ex}
      \textbf{Return Value}
    \vspace{-1ex}

      \begin{quote}
      boolean.

      \end{quote}

    \end{boxedminipage}

    \label{pygtk_chart:pie_chart:PieChart:set_enable_scroll}
    \index{pygtk\_chart \textit{(package)}!pygtk\_chart.pie\_chart \textit{(module)}!pygtk\_chart.pie\_chart.PieChart \textit{(class)}!pygtk\_chart.pie\_chart.PieChart.set\_enable\_scroll \textit{(method)}}

    \vspace{0.5ex}

\hspace{.8\funcindent}\begin{boxedminipage}{\funcwidth}

    \raggedright \textbf{set\_enable\_scroll}(\textit{self}, \textit{scroll})

    \vspace{-1.5ex}

    \rule{\textwidth}{0.5\fboxrule}
\setlength{\parskip}{2ex}
    Set whether the pie chart can be rotated by scrolling with the mouse 
    wheel.

\setlength{\parskip}{1ex}
      \textbf{Parameters}
      \vspace{-1ex}

      \begin{quote}
        \begin{Ventry}{xxxxxx}

          \item[scroll]

            {\it (type=boolean.)}

        \end{Ventry}

      \end{quote}

    \end{boxedminipage}

    \label{pygtk_chart:pie_chart:PieChart:get_enable_scroll}
    \index{pygtk\_chart \textit{(package)}!pygtk\_chart.pie\_chart \textit{(module)}!pygtk\_chart.pie\_chart.PieChart \textit{(class)}!pygtk\_chart.pie\_chart.PieChart.get\_enable\_scroll \textit{(method)}}

    \vspace{0.5ex}

\hspace{.8\funcindent}\begin{boxedminipage}{\funcwidth}

    \raggedright \textbf{get\_enable\_scroll}(\textit{self})

    \vspace{-1.5ex}

    \rule{\textwidth}{0.5\fboxrule}
\setlength{\parskip}{2ex}
    Returns True if the user can rotate the pie chart by scrolling.

\setlength{\parskip}{1ex}
      \textbf{Return Value}
    \vspace{-1ex}

      \begin{quote}
      boolean.

      \end{quote}

    \end{boxedminipage}

    \label{pygtk_chart:pie_chart:PieChart:set_enable_mouseover}
    \index{pygtk\_chart \textit{(package)}!pygtk\_chart.pie\_chart \textit{(module)}!pygtk\_chart.pie\_chart.PieChart \textit{(class)}!pygtk\_chart.pie\_chart.PieChart.set\_enable\_mouseover \textit{(method)}}

    \vspace{0.5ex}

\hspace{.8\funcindent}\begin{boxedminipage}{\funcwidth}

    \raggedright \textbf{set\_enable\_mouseover}(\textit{self}, \textit{mouseover})

    \vspace{-1.5ex}

    \rule{\textwidth}{0.5\fboxrule}
\setlength{\parskip}{2ex}
    Set whether a mouseover effect should be shown when the pointer enters 
    a pie area.

\setlength{\parskip}{1ex}
      \textbf{Parameters}
      \vspace{-1ex}

      \begin{quote}
        \begin{Ventry}{xxxxxxxxx}

          \item[mouseover]

            {\it (type=boolean.)}

        \end{Ventry}

      \end{quote}

    \end{boxedminipage}

    \label{pygtk_chart:pie_chart:PieChart:get_enable_mouseover}
    \index{pygtk\_chart \textit{(package)}!pygtk\_chart.pie\_chart \textit{(module)}!pygtk\_chart.pie\_chart.PieChart \textit{(class)}!pygtk\_chart.pie\_chart.PieChart.get\_enable\_mouseover \textit{(method)}}

    \vspace{0.5ex}

\hspace{.8\funcindent}\begin{boxedminipage}{\funcwidth}

    \raggedright \textbf{get\_enable\_mouseover}(\textit{self})

    \vspace{-1.5ex}

    \rule{\textwidth}{0.5\fboxrule}
\setlength{\parskip}{2ex}
    Returns True if the mouseover effect is enabled.

\setlength{\parskip}{1ex}
      \textbf{Return Value}
    \vspace{-1ex}

      \begin{quote}
      boolean.

      \end{quote}

    \end{boxedminipage}

    \label{pygtk_chart:pie_chart:PieChart:set_show_values}
    \index{pygtk\_chart \textit{(package)}!pygtk\_chart.pie\_chart \textit{(module)}!pygtk\_chart.pie\_chart.PieChart \textit{(class)}!pygtk\_chart.pie\_chart.PieChart.set\_show\_values \textit{(method)}}

    \vspace{0.5ex}

\hspace{.8\funcindent}\begin{boxedminipage}{\funcwidth}

    \raggedright \textbf{set\_show\_values}(\textit{self}, \textit{show})

    \vspace{-1.5ex}

    \rule{\textwidth}{0.5\fboxrule}
\setlength{\parskip}{2ex}
    Set whether the area's value should be shown in its label.

\setlength{\parskip}{1ex}
      \textbf{Parameters}
      \vspace{-1ex}

      \begin{quote}
        \begin{Ventry}{xxxx}

          \item[show]

            {\it (type=boolean.)}

        \end{Ventry}

      \end{quote}

    \end{boxedminipage}

    \label{pygtk_chart:pie_chart:PieChart:get_show_values}
    \index{pygtk\_chart \textit{(package)}!pygtk\_chart.pie\_chart \textit{(module)}!pygtk\_chart.pie\_chart.PieChart \textit{(class)}!pygtk\_chart.pie\_chart.PieChart.get\_show\_values \textit{(method)}}

    \vspace{0.5ex}

\hspace{.8\funcindent}\begin{boxedminipage}{\funcwidth}

    \raggedright \textbf{get\_show\_values}(\textit{self})

    \vspace{-1.5ex}

    \rule{\textwidth}{0.5\fboxrule}
\setlength{\parskip}{2ex}
    Returns True if the value of a pie area is shown in its label.

\setlength{\parskip}{1ex}
      \textbf{Return Value}
    \vspace{-1ex}

      \begin{quote}
      boolean.

      \end{quote}

    \end{boxedminipage}


\large{\textbf{\textit{Inherited from pygtk\_chart.chart.Chart\textit{(Section \ref{pygtk_chart:chart:Chart})}}}}

\begin{quote}
draw\_basics(), export\_png(), export\_svg(), get\_padding(), set\_padding()
\end{quote}

\large{\textbf{\textit{Inherited from gtk.DrawingArea}}}

\begin{quote}
size()
\end{quote}

\large{\textbf{\textit{Inherited from gtk.Widget}}}

\begin{quote}
activate(), add\_accelerator(), add\_events(), add\_mnemonic\_label(), can\_activate\_accel(), child\_focus(), child\_notify(), class\_path(), create\_pango\_context(), create\_pango\_layout(), destroy(), do\_button\_press\_event(), do\_button\_release\_event(), do\_can\_activate\_accel(), do\_client\_event(), do\_composited\_changed(), do\_configure\_event(), do\_delete\_event(), do\_destroy\_event(), do\_direction\_changed(), do\_drag\_begin(), do\_drag\_data\_delete(), do\_drag\_data\_get(), do\_drag\_data\_received(), do\_drag\_drop(), do\_drag\_end(), do\_drag\_leave(), do\_drag\_motion(), do\_enter\_notify\_event(), do\_event(), do\_expose\_event(), do\_focus(), do\_focus\_in\_event(), do\_focus\_out\_event(), do\_get\_accessible(), do\_grab\_broken\_event(), do\_grab\_focus(), do\_grab\_notify(), do\_hide(), do\_hide\_all(), do\_hierarchy\_changed(), do\_key\_press\_event(), do\_key\_release\_event(), do\_leave\_notify\_event(), do\_map(), do\_map\_event(), do\_mnemonic\_activate(), do\_motion\_notify\_event(), do\_no\_expose\_event(), do\_parent\_set(), do\_popup\_menu(), do\_property\_notify\_event(), do\_proximity\_in\_event(), do\_proximity\_out\_event(), do\_realize(), do\_screen\_changed(), do\_scroll\_event(), do\_selection\_clear\_event(), do\_selection\_get(), do\_selection\_notify\_event(), do\_selection\_received(), do\_selection\_request\_event(), do\_show(), do\_show\_all(), do\_show\_help(), do\_size\_allocate(), do\_size\_request(), do\_state\_changed(), do\_style\_set(), do\_unmap(), do\_unmap\_event(), do\_unrealize(), do\_visibility\_notify\_event(), do\_window\_state\_event(), drag\_begin(), drag\_check\_threshold(), drag\_dest\_add\_image\_targets(), drag\_dest\_add\_text\_targets(), drag\_dest\_add\_uri\_targets(), drag\_dest\_find\_target(), drag\_dest\_get\_target\_list(), drag\_dest\_get\_track\_motion(), drag\_dest\_set(), drag\_dest\_set\_proxy(), drag\_dest\_set\_target\_list(), drag\_dest\_set\_track\_motion(), drag\_dest\_unset(), drag\_get\_data(), drag\_highlight(), drag\_source\_add\_image\_targets(), drag\_source\_add\_text\_targets(), drag\_source\_add\_uri\_targets(), drag\_source\_get\_target\_list(), drag\_source\_set(), drag\_source\_set\_icon(), drag\_source\_set\_icon\_name(), drag\_source\_set\_icon\_pixbuf(), drag\_source\_set\_icon\_stock(), drag\_source\_set\_target\_list(), drag\_source\_unset(), drag\_unhighlight(), ensure\_style(), error\_bell(), event(), freeze\_child\_notify(), get\_accessible(), get\_action(), get\_activate\_signal(), get\_allocation(), get\_ancestor(), get\_child\_requisition(), get\_child\_visible(), get\_clipboard(), get\_colormap(), get\_composite\_name(), get\_direction(), get\_display(), get\_events(), get\_extension\_events(), get\_has\_tooltip(), get\_modifier\_style(), get\_name(), get\_no\_show\_all(), get\_pango\_context(), get\_parent(), get\_parent\_window(), get\_pointer(), get\_root\_window(), get\_screen(), get\_settings(), get\_size\_request(), get\_snapshot(), get\_style(), get\_tooltip\_markup(), get\_tooltip\_text(), get\_tooltip\_window(), get\_toplevel(), get\_visual(), get\_window(), grab\_add(), grab\_default(), grab\_focus(), grab\_remove(), has\_screen(), hide(), hide\_all(), hide\_on\_delete(), input\_shape\_combine\_mask(), intersect(), is\_ancestor(), is\_composited(), is\_focus(), keynav\_failed(), list\_mnemonic\_labels(), map(), menu\_get\_for\_attach\_widget(), mnemonic\_activate(), modify\_base(), modify\_bg(), modify\_cursor(), modify\_fg(), modify\_font(), modify\_style(), modify\_text(), path(), queue\_clear(), queue\_clear\_area(), queue\_draw(), queue\_draw\_area(), queue\_resize(), queue\_resize\_no\_redraw(), rc\_get\_style(), realize(), region\_intersect(), remove\_accelerator(), remove\_mnemonic\_label(), render\_icon(), reparent(), reset\_rc\_styles(), reset\_shapes(), selection\_add\_target(), selection\_add\_targets(), selection\_clear\_targets(), selection\_convert(), selection\_owner\_set(), selection\_remove\_all(), send\_expose(), set\_accel\_path(), set\_activate\_signal(), set\_app\_paintable(), set\_child\_visible(), set\_colormap(), set\_composite\_name(), set\_direction(), set\_double\_buffered(), set\_events(), set\_extension\_events(), set\_has\_tooltip(), set\_name(), set\_no\_show\_all(), set\_parent(), set\_parent\_window(), set\_redraw\_on\_allocate(), set\_scroll\_adjustments(), set\_sensitive(), set\_set\_scroll\_adjustments\_signal(), set\_size\_request(), set\_state(), set\_style(), set\_tooltip\_markup(), set\_tooltip\_text(), set\_tooltip\_window(), set\_uposition(), set\_usize(), shape\_combine\_mask(), show(), show\_all(), show\_now(), size\_allocate(), size\_request(), style\_get\_property(), thaw\_child\_notify(), translate\_coordinates(), trigger\_tooltip\_query(), unmap(), unparent(), unrealize()
\end{quote}

\large{\textbf{\textit{Inherited from gtk.Object}}}

\begin{quote}
do\_destroy(), flags(), remove\_data(), remove\_no\_notify(), set\_flags(), unset\_flags()
\end{quote}

\large{\textbf{\textit{Inherited from ??.GObject}}}

\begin{quote}
\_\_cmp\_\_(), \_\_copy\_\_(), \_\_deepcopy\_\_(), \_\_delattr\_\_(), \_\_gdoc\_\_(), \_\_gobject\_init\_\_(), \_\_hash\_\_(), \_\_new\_\_(), \_\_repr\_\_(), \_\_setattr\_\_(), chain(), connect(), connect\_after(), connect\_object(), connect\_object\_after(), disconnect(), disconnect\_by\_func(), emit(), emit\_stop\_by\_name(), freeze\_notify(), get\_data(), get\_properties(), get\_property(), handler\_block(), handler\_block\_by\_func(), handler\_disconnect(), handler\_is\_connected(), handler\_unblock(), handler\_unblock\_by\_func(), notify(), props(), set\_data(), set\_properties(), set\_property(), stop\_emission(), thaw\_notify(), weak\_ref()
\end{quote}

\large{\textbf{\textit{Inherited from atk.ImplementorIface}}}

\begin{quote}
ref\_accessible()
\end{quote}

\large{\textbf{\textit{Inherited from gtk.Buildable}}}

\begin{quote}
add\_child(), construct\_child(), do\_add\_child(), do\_construct\_child(), do\_get\_internal\_child(), do\_parser\_finished(), do\_set\_name(), get\_internal\_child(), parser\_finished()
\end{quote}

\large{\textbf{\textit{Inherited from object}}}

\begin{quote}
\_\_getattribute\_\_(), \_\_reduce\_\_(), \_\_reduce\_ex\_\_(), \_\_str\_\_()
\end{quote}

%%%%%%%%%%%%%%%%%%%%%%%%%%%%%%%%%%%%%%%%%%%%%%%%%%%%%%%%%%%%%%%%%%%%%%%%%%%
%%                              Properties                               %%
%%%%%%%%%%%%%%%%%%%%%%%%%%%%%%%%%%%%%%%%%%%%%%%%%%%%%%%%%%%%%%%%%%%%%%%%%%%

  \subsubsection{Properties}

    \vspace{-1cm}
\hspace{\varindent}\begin{longtable}{|p{\varnamewidth}|p{\vardescrwidth}|l}
\cline{1-2}
\cline{1-2} \centering \textbf{Name} & \centering \textbf{Description}& \\
\cline{1-2}
\endhead\cline{1-2}\multicolumn{3}{r}{\small\textit{continued on next page}}\\\endfoot\cline{1-2}
\endlastfoot\multicolumn{2}{|l|}{\textit{Inherited from gtk.Widget}}\\
\multicolumn{2}{|p{\varwidth}|}{\raggedright allocation, name, parent, requisition, saved\_state, state, style, window}\\
\cline{1-2}
\multicolumn{2}{|l|}{\textit{Inherited from ??.GObject}}\\
\multicolumn{2}{|p{\varwidth}|}{\raggedright \_\_grefcount\_\_}\\
\cline{1-2}
\multicolumn{2}{|l|}{\textit{Inherited from object}}\\
\multicolumn{2}{|p{\varwidth}|}{\raggedright \_\_class\_\_}\\
\cline{1-2}
\end{longtable}


%%%%%%%%%%%%%%%%%%%%%%%%%%%%%%%%%%%%%%%%%%%%%%%%%%%%%%%%%%%%%%%%%%%%%%%%%%%
%%                            Class Variables                            %%
%%%%%%%%%%%%%%%%%%%%%%%%%%%%%%%%%%%%%%%%%%%%%%%%%%%%%%%%%%%%%%%%%%%%%%%%%%%

  \subsubsection{Class Variables}

    \vspace{-1cm}
\hspace{\varindent}\begin{longtable}{|p{\varnamewidth}|p{\vardescrwidth}|l}
\cline{1-2}
\cline{1-2} \centering \textbf{Name} & \centering \textbf{Description}& \\
\cline{1-2}
\endhead\cline{1-2}\multicolumn{3}{r}{\small\textit{continued on next page}}\\\endfoot\cline{1-2}
\endlastfoot\raggedright \_\-\_\-g\-p\-r\-o\-p\-e\-r\-t\-i\-e\-s\-\_\-\_\- & \raggedright \textbf{Value:} 
{\tt \{"rotate":(gobject.TYPE\_INT, "rotation", "The angle to ro\texttt{...}}&\\
\cline{1-2}
\raggedright \_\-\_\-g\-s\-i\-g\-n\-a\-l\-s\-\_\-\_\- & \raggedright \textbf{Value:} 
{\tt \{"area-clicked":(gobject.SIGNAL\_RUN\_LAST, gobject.TYPE\_NO\texttt{...}}&\\
\cline{1-2}
\raggedright \_\-\_\-g\-t\-y\-p\-e\-\_\-\_\- & \raggedright \textbf{Value:} 
{\tt {\textless}GType pygtk\_chart+pie\_chart+PieChart (171517840){\textgreater}}&\\
\cline{1-2}
\end{longtable}

    \index{pygtk\_chart \textit{(package)}!pygtk\_chart.pie\_chart \textit{(module)}!pygtk\_chart.pie\_chart.PieChart \textit{(class)}|)}
    \index{pygtk\_chart \textit{(package)}!pygtk\_chart.pie\_chart \textit{(module)}|)}
